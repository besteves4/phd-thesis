\section{Gaps and challenges}
\label{sec:challenges}

This Section discusses the challenges of having a transparent, legally-aligned Solid ecosystem based on the literature review described in this Chapter. The following issues were identified based on the performed analysis:

\begin{enumerate}
    \item [Ch1.] \textbf{Identity of Solid actors and their roles is unknown} -- Solid users lack awareness of the entities responsible for providing and/or developing their Pods, the applications they utilise, WebIDs, or other server infrastructure. Additionally, the majority of applications or services fail to give contact details or information about their data protection officer.
    \item [Ch2.] \textbf{No metadata about Solid infrastructure} -- Solid users lack information regarding the Solid specification their Pod is operating on, the services installed within it, or the location of the servers where Pods are hosted. Additionally, there is no record of this information kept in the Pod for convenient reference by the user.\footnote{An incomplete catalogue of Pod providers is published at \url{https://solidproject.org/users/get-a-pod}, detailing the hosting service used for the Pods (although no specifics on the entities behind them are provided) and, occasionally, the country of hosting (though lacking a privacy policy for the storage service).}
    \item [Ch3.] \textbf{Availability/Discovery of categories of data} -- In order for Solid applications to access data within Pods at a granular level, i.e., by data type, they require knowledge of its existence and storage location within the Pod. Additionally, for smooth interoperability, it is essential to document the schemas, formats, or shapes for data recognised or supported by applications, services, or Pods.
    \item [Ch4.] \textbf{Pod and applications providers do not provide information on their data processing practices} -- The majority of providers and developers offering Pod-related services fail to give human and/or machine-readable privacy notices or specify the data they require for operation. It is imperative to document this information within the Pod itself to enable users to retain a record of data requests. This ensures that users have a reference in case data is utilised in a manner not authorised by them.
    \item [Ch5.] \textbf{Users cannot express their privacy policies} -- Solid users lack the means to articulate their privacy preferences and requirements, as well as to oversee incoming data requests or manage existing agreements regarding data usage.
    \item [Ch6.] \textbf{No logging or record-keeping} -- There is no recorded provenance metadata in the user Pod for accountability purposes. For instance, users do not maintain consent records or information about who has accessed their data, how it is being utilised, or any alterations to data policies.
    \item [Ch7.] \textbf{No legal compliance checks} -- Currently there are no Solid-based tools for Solid users to address legal obligations, like granting/revoking consent or exercising rights under the GDPR. Additionally, no tools are available for the authorities conducting investigations to access required auditing information.
    \item [Ch8.] \textbf{Societal and business needs} -- Beyond legal requirements, user studies still need to be performed to understand what type of policies fulfil the needs and expectations of users. Furthermore, a similar exercise needs to be performed for companies in order to understand how they can function and adapt their business for such decentralised data-sharing environments.
\end{enumerate}