\section{Impact}
\label{sec:impact}

The outcomes of this Thesis highlight the advantages and possibilities of a joint technological and legal approach to personal data management.
While past approaches have nearly exclusively focused on the technological prevention of legally undesirable behaviours, the scientific, technological, and societal solutions derived from this Thesis' work enable a much wider perspective on the problem space.
The contributions of the Thesis are carefully supported and validated by legal research and also challenge this research domain, e.g., by highlighting the shortcomings of using consent as a legal ground.

As such, this work has the potential for a strong, interdisciplinary, scientific, technological, and societal impact, which can serve as the foundation of important techo-legal research to come.
This statement is supported by the several published RDF vocabularies derived from this Thesis' contributions, which are finding adoption with industry practitioners.
The active participation and contribution to W3C specification processes also further highlights the aforementioned impact in standardisation activities.