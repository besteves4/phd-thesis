\chapter{Conclusions}
\label{chap:conclusions}

In a world where AI-based technologies are taking over and distrust in data-consuming services is at its highest point, BigTech companies prefer to deal with the consequences of their unlawful practices than provide the necessary tools to data subjects to make the right decisions over the processing of their personal data.
Endowed with enhanced interoperability and transparency features, the decentralised Semantic Web aims to aid data subjects in taking control of the publication and movement of their personal data.
As such, legally-aligned vocabularies and services were produced to support policy-based access to data in decentralised settings, providing accountability and enhanced transparency to people looking at regaining trust in Web services. 
To this end, Section~\ref{sec:fulfilment} concludes the Thesis with a discussion on the extent to which the research objectives have been fulfilled through the contributions described in Chapters~\ref{chap:vocabularies} to~\ref{chap:dga}.
Section~\ref{sec:future_work} provides lines of future work arising from the research presented within this Thesis.
Finally, Section~\ref{sec:impact} discusses the scientific, technological, and societal impact of the Thesis' contributions.

\section{Fulfilment of research objectives}
\label{sec:fulfilment}
\section{Future work}
\label{sec:future_work}

% Regarding future work, as described in Section 6.3, there are still several open problems in the processing of temporal information. Besides targeting the context-dependent temporal expressions described there, future work envisaged is outlined below.

\paragraph{beyond consent}

\paragraph{new data protection laws} within and outside EU and how they are interconnected, which ones apply in what scenarios

\paragraph{delegation}

\paragraph{data reuse}

\paragraph{more interfaces}

\paragraph{web agents}

\paragraph{usage control}

\paragraph{contextualised data}

\paragraph{data spaces}
\section{Impact}
\label{sec:impact}

The outcomes of this Thesis highlight the advantages and possibilities of a joint technological and legal approach to personal data management.
While past approaches have nearly exclusively focused on the technological prevention of legally undesirable behaviours, the scientific, technological, and societal solutions derived from this Thesis' work enable a much wider perspective on the problem space.
The contributions of the Thesis are carefully supported and validated by legal research and also challenge this research domain, e.g., by highlighting the shortcomings of using consent as a legal ground.

As such, this work has the potential for a strong, interdisciplinary, scientific, technological, and societal impact, which can serve as the foundation of important techo-legal research to come.
This statement is supported by the several published RDF vocabularies derived from this Thesis' contributions, which are finding adoption with industry practitioners.
The active participation and contribution to W3C specification processes also further highlights the aforementioned impact in standardisation activities.