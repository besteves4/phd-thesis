\section{The emergence of decentralised PIMS}
\label{sec:motivation_legal}

As previously mentioned in Section~\ref{sec:def_data_protection_law}, the governance of data flows, and in particular of \textit{personal} data flows, has been a topic of discussion since the early 1970s and 1980s, when the Fair Information Practice Principles (FIPPs)~\citep{cate_failure_2006} and Convention 108~\citep{council_of_europe_convention_1981} were first created, to GDPR and subsequent personal data-related regulations being developed in countries such as Brazil or India~\citep{bradford_brussels_2019}.
Most of these instruments rely on the existence of an accountable entity that is responsible for establishing the purpose of processing personal data from a natural person, who has rights that must be respected for said processing to be considered compliant with the law.
This model has been the most prevalent since most personal data are stored in large centralised databases under the control of only a certain number of Big tech companies, however, it does not account for cases where the processing is shared among different entities which have distinct purposes or rely on unsuitable legal bases, or the information overload that prevents individuals from actually understanding what they are consenting to~\citep{benshahar_more_2014}.
As such, new data governance systems that assist individuals in having more understanding and control over their data and trust in data processing services, such as \textit{data cooperatives}, \textit{data trusts}, \textit{data commons} or \textit{personal data sovereignty} schemes, are being proposed~\citep{viljoen_relational_2021,craglia_digitranscope_2021}.
For instance, data trusts may give more emphasis on ensuring that data subjects have a good understanding of the purposes for which the personal data is being used and that these are explained clearly and transparently, however, they do not offer control over those purposes in the same way data cooperatives do, where data subjects can participate in purpose and rule decision making.
Moreover, these systems are even starting to be regulated, such as the new requirements on data intermediation services described in the DGA~\citeyearpar{noauthor_regulation_2022}.

In this context, the emergence of decentralised PIMS for the Web, such as the personal datastores model promoted by Solid and studied in this Thesis, has earned many advocates in the last years.
In particular, when it comes to trust, the usefulness and ease of use of digital personal datastores have been proven to be an important factor in increasing citizens' trust in personal data-handling services by allowing them to share their sensitive data for the `public good' while maintaining a sense of control over their data~\citep{mariani_explaining_2021}.
Moreover, while these decentralised solutions are not without their faults, as has been shown by blockchain-related scandals in the financial services industry~\citep{zetzsche_ico_2019}, their Semantic Web-based counterparts have been gaining a large number of adopters recently as such systems can actually allow its users to choose who can access their data and, therefore, actually shift the power balance in favour of the individuals.
By detaching the storage of data from the data processing services and promoting the usage of Web standards, individuals can move their data between storage providers, use the same data across different services and choose which services and applications best suit their preferences and needs without being locked out of the access to their data~\citep{verbrugge_towards_2021,ilves_roadmap_2019}.
This user-managed access to data represents a considerable change from the current \textit{status quo}, where individuals must usually accept an application's privacy policy in order to use it, while personal datastores present the next step towards having an actual negotiation of privacy terms between individuals and data processing entities.
Such systems are also promoted by the EDPS as a mechanism to enable personal data sovereignty where ``Individuals, service providers and applications would need to authenticate to access a personal storage centre'' in an interoperable manner~\citep{european_data_protection_supervisor_techdispatch_2021}.
Additionally, the European data spaces initiative launched by the~\cite{european_commission_communication_2020} follows the same spirit by encouraging the development of infrastructures for data holders and data users to share and reuse data across different services while respecting European data protection law.
However, it should be acknowledged that such data spaces are still focused on encouraging industrial data sharing, for large institutions to gain economic and societal benefits, often at a sectorial level, rather than empowering data subjects individually.
Nevertheless, they are still an improvement over current systems where the data controller is purely motivated by profit for their organisation alone.

While personal datastores' developers have as their main banner that data subjects are `controllers' of their data, this view is incompatible with most data protection-related regulations as \textit{``most [...] legal systems are structured around the identification of an accountable entity''}~\citep{chomczyk_penedo_selfsovereign_2021} which is given duties in order to ensure that their data processing activities do not affect data subjects' fundamental rights.
In addition, personal data processing activities often involve a complex web of parties that share control of the usage, storage and collection of data for distinct and shared purposes, a fact that makes the compliance with the information requirements described in GDPR's Articles 12 to 14 quite challenging and difficult to implement~\citep{lovato_more_2023} -- compliance with such requirements is usually dealt with by providing lengthy and complex notices which are not easy to understand and place a significant burden on data subjects as they have to deal with at least one notice for every personal data processing service they use~\citep{terpstra_improving_2019,linden_privacy_2020}.
Although these notices usually address the information required by the law, in no way do they fulfil the `informed' character of consent, as prescribed in GDPR's Article 4.11, as it is utterly challenging for data subjects to understand the plethora of terms and conditions of all the personal data handling services that are used nowadays, from smartphone applications to personalised streaming of content, and to give consent in a freely, specific, informed and unambiguous way~\citep{mohan_analyzing_2019}.
Beyond consent management, notices are also an inefficient way for data subjects to exercise their rights as they are limited to describing rights and where to exercising them, without focusing on providing data subjects with actually tools \textit{``for facilitating the exercise of the data subject’s rights [...], including mechanisms to request and, if applicable, obtain, free of charge, in particular, access to and rectification or erasure of personal data and the exercise of the right to object''}, as mentioned in GDPR's Recital 59~\citeyearpar{noauthor_regulation_2016}.

As such, recently, there has been legal work in identifying the different roles and responsibilities that distinct entities occupy in decentralised systems and how said systems can be used to facilitate the exercise of data subjects' rights, fulfil the data protection principles of privacy by design and by default and improve the clarity and transparency of personal data handling processes in contrast to the existing landscape~\citep{janssen_personal_2020}, as described in Section~\ref{sec:sota_solid_data_protection}.
Nevertheless, work still needs to be done to align such decentralised systems with the legal requirements, in particular, related to the identification and enforcement of a lawful basis and transparent purpose that justify the access to data.  
Particularly, in this Thesis, the focus is positioned on how to obtain valid consent in Solid, according to the GDPR, while promoting the usage of automation to improve the current information overload felt by data subjects in terms of consent management.
Accordingly, in this Chapter, the introduction of a semantic policy layer, based on the vocabularies described in Chapter~\ref{chap:vocabularies}, for providing the necessary information to obtain informed and valid GDPR consent is studied from a technical and legal angle.

% FROM THE PAPER: In particular, we focus on the processing of health data (considered a special category of personal data under the GDPR) for biomedical research purposes.}

It is likewise important to distinguish between what can be technologically or legally enforced.
Although technically a certain app or service can be restricted to only access certain data, by being allowed to read data from a personal datastore, it can also copy it, even if within a decentralised setting this is not necessary at all.
Thereupon, the realm of law comes into play.
Although the wishes of the data subjects, as stated by the preferences they have stored in their personal datastore, have a role to play in the negotiation of privacy terms between data subjects and data controllers~\citep{verborgh_paradigm_2017}, their legal value is still up to debate, as these are quite new technological tools which are still to be argued and tested in the court of law.

As such, in the next Section, the usage of policies as a precursor of consent is further studied.