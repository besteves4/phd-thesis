\section{The Emergence of Decentralised PIMS}
\label{sec:motivation_legal}

As previously mentioned in Section \ref{sec:def_data_protection_law}, the governance of data flows, and in particular of \textit{personal} data flows, has been a topic of discussion since the early 1970s and 1980s, when the Fair Information Practice Principles (FIPPs) \citep{cate_failure_2006} and Convention 108 \citep{council_of_europe_convention_1981} were first created, to GDPR and subsequent personal data-related regulations being developed in countries such as Brazil or India \citep{bradford_brussels_2019}.
Most of these instruments rely on the existence of an accountable entity that is responsible for establishing the purpose of processing personal data from a natural person, who has rights that must be respected for said processing to be considered compliant with the law.
This model has been the most prevalent since most personal data are stored in large centralised databases under the control of only a certain number of Big tech companies, however, it does not account for cases where the processing is shared among different entities which have distinct purposes or rely on unsuitable legal bases, or the information overload that prevents individuals from actually understanding what they are consenting to \citep{benshahar_more_2014}.
As such, new data governance systems that assist individuals in having more control over their data, such as \textit{data cooperatives}, \textit{data trusts}, \textit{data commons} or \textit{personal data sovereignty} schemes, are being proposed \citep{viljoen_relational_2021,craglia_digitranscope_2021} and are even starting to be regulated, such as the new requirements on data intermediation services described in the DGA \citeyearpar{noauthor_regulation_2022}.

In this context, the emergence of decentralised PIMS for the Web, such as the personal datastores model promoted by Solid and studied in this Thesis, has earned many advocates in the last years.
While these decentralised solutions are not without their faults, as has been shown by blockchain-related scandals in the financial services industry \citep{zetzsche_ico_2019}, its Semantic Web-based counterparts have been gaining a large number of adopters recently as such systems can actually allow its users to choose who can access their data and, therefore, actually shift the power balance in favor of the individuals.
By detaching the storage of data from the data processing services and promoting the usage of Web standards, individuals can move their data between storage providers, use the same data across different services and choose which services and applications best suit their preferences without being locked out of the access to their data \citep{verbrugge_towards_2021}.
This user-managed access to data represents a considerable change from the current \textit{status quo}, where individuals must usually accept an application's privacy policy in order to use it, while personal datastores present the next step towards having an actual negotiation of privacy terms between individuals and data processing entities.
Additionally, the European data spaces initiative launched by the European Commission \citep{european_commission_communication_2020} follows the same spirit by encouraging the development of infrastructures for data holders and data users to share and reuse data across different services, while respecting European data protection law.

While personal datastores' developers have as their main banner that data subjects are `controllers' of their data, this view is incompatible with most data protection-related regulations as \textit{``most [...] legal systems are structured around the identification of an accountable entity''} \citep{chomczyk_penedo_selfsovereign_2021} which is given duties in order to ensure that their data processing activities do not affect data subjects' fundamental rights.
As such, recently, there has been legal work in identifying entities' roles and responsibilities in said decentralised systems and how said systems can be used to facilitate the exercise of data subjects' rights, fulfil the data protection principles of privacy by design and by default and improve the transparency of personal data handling processes in contrast to the existing landscape \cite{janssen_personal_2020}, as described in Section~\ref{sec:sota_solid_data_protection}.
Nevertheless, work still needs to be done to align such decentralised systems with the legal requirements, in particular related to the identification enforcement of the legal bases and purposes that justify the access to data.  
Particularly, in this Thesis, the focus is positioned on how to obtain valid consent, according to the GDPR, while promoting the usage of automation to improve the current information overload felt by data subjects in terms of consent management.