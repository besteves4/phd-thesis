\section{Ethical challenges of controlling data and reclaiming control over it}
\label{sec:ethical_challenges}

Advancements in data-driven innovations are poised to drive further economic and societal progress~\citep{jacobides_platforms_2019}.
The analysis, sharing, and reuse of data have led to transformative changes in business models and government processes, enabling them to capitalise on these practices. 
As discussed in the previous Sections, these changes propelled policy initiatives implemented by various governments globally.
In particular, the EU is actively engaged in this transformation, exemplified by the \cite{european_commission_communication_2020} commitment\footnote{The European Commission's strategy and related documents are available at \url{https://ec.europa.eu/info/strategy/priorities-2019-2024/europe-fit-digital-age_en} (accessed on 10 March 2024)} to shaping ``A Europe fit for the Digital Age''. 
Whether it is a prominent Big Tech firm headquartered in the United States, a major data intermediary in the EU, or a state-controlled entity in China, contemporary data practices face scrutiny from diverse sectors of society, spanning individuals, non-governmental organisations (NGOs), academics, and governmental bodies.
Such distrust in digital services has been called into question~\citep{waldman_industry_2021}, prompting individuals to ponder who should they trust their data with.

Amidst this trust crisis, technology has emerged as a potential solution, in particular self-sovereign PIMS \citep{chomczyk_penedo_selfsovereign_2021}, as discussed in Section~\ref{sec:motivation_legal}.
These models empower users to directly control their data, dictating the terms of access and usage, and have been gaining the support of policymakers, in particular in Europe, with the European Commission supporting the creation of common European data spaces~\citeyearpar{noauthor_commission_2022}.
Moreover, it could be argued that the EU is strategically investing in these technologies to foster more democratic and participatory data practices, and enhance confidence in data-intensive operations by advocating for technologically robust systems that reduce reliance on the reputation of individual firms, thus mitigating power imbalances between data subjects and controllers~\citep{european_commission_communication_2020}.

The literature exploring the concept of trust is extensive, yet complex due to varying interpretations.
\cite{de_filippi_blockchain_2020} distinguished trust from confidence, noting that trust is rooted in personal vulnerability and risk-taking, while confidence is based on internalised expectations stemming from knowledge or past experiences.
As such, in this Section, the interest of data subjects in technologies that provide insights into how their information is integrated into real personal data handling processes is studied as a vehicle of trust, given their general apprehension regarding the processing actions of data controllers over personal data.
As visible in the previous Sections, the personal data regulatory framework in the EU is designed to address imbalances or vulnerabilities between multiple parties by revealing potential risks and resulting harms, aiming to leverage consent as a catalyst for the data-driven economy \citep{chomczyk_penedo_towards_2022}. 
Simultaneously, they aim to furnish essential information to individuals making decisions, facilitating informed choices \cite{benshahar_more_2014}.
Furthermore, from an ethical standpoint, several norms emerge that should guide the conduct of individuals with whom information is shared to ensure trustworthiness.
These norms encompass sincerity, competence, and the appropriateness of the entrusted task \citep{hawley_how_2019}.

Considering the myriad of factors influencing both trust and confidence, the analysis in this Section focuses on (i) transparency as a crucial prerequisite for the functioning of decentralised PIMS, (ii) the relevant EU regulatory framework on personal data, and (iii) an ethical debate concerning data control, as outlined in \citeauthor{bodo_mediated_2021}'s framework for mediated technological trust.
The emphasis on transparency stems from three primary considerations:
\begin{itemize}
    \item from a regulatory viewpoint, transparency stands as a fundamental principle within personal data protection regimes, often integrated alongside lawfulness and fairness, as exemplified in GDPR's Article 5.1(a);
    \item transparency encompasses both its \textit{ex-ante} and \textit{ex-post} components, with the latter including the issue of explainability \citep{felzmann_transparency_2019};
    \item transparency offers the potential to demystify the `black box' nature of many AI systems, enabling the identification of potential biases towards vulnerable populations \citep{pasquale_black_2015}.
\end{itemize}

As illustrated by case law from supervisory authorities, the intricate nature of data processing activities has proven challenging for data controllers to articulate in straightforward terms, especially when relying on limited attention resources from data subjects \citep{european_data_protection_board_guidelines_2020}.
The dearth of actionable information, to understand data handling practices, poses a risk to fostering trust among involved parties.
As a result, individuals are endeavoring to reassert control over their data and restrict its usage by such entities, also by looking at new data governance schemes such as PIMS or other data intermediaries \citep{craglia_digitranscope_2021,papagiannakopoulou_leveraging_2014}. 

As such, the concept of `control' gains particular importance as users require someone to trust in order to reclaim control over their data in the digital era.
Emerging data governance models are coupled with legal frameworks to assist data subjects in asserting their agency.
For instance, in the data cooperative model (which is regulated by the DGA), cooperatives act as trustees overseeing data on behalf of data subjects, thus enabling data subjects to maintain democratic control over their data. 
In such governance frameworks, establishing a relationship of trust between cooperatives managing data and data subjects is paramount.
In certain instances, trustees may need to consult with data subjects, providing agreements and contracts to inform them. 
Meanwhile, data subjects can articulate their preferences and determine how to share their data and for what purposes~\citep{craglia_digitranscope_2021}.

Data cooperatives and other intermediaries (will) play a pivotal role in empowering data subjects to maintain control over their data and reassert their ethical standing in the digital era.
Specifically, personal data sovereignty offers a significant return to more democratic and egalitarian governance, allowing individuals to reclaim control over their personal data \citep{craglia_digitranscope_2021, giannopoulou_digital_2023}.
In theory, these systems should restore personal autonomy and uphold classical liberal values by fostering trust-based relationships. %TODO: add citation
Furthermore, drawing from our current democratic experiences can offer valuable lessons to avoid repeating the same mistakes made in the past two centuries.
During this time, a substantial portion of the population, particularly in the Global South, suffered from neglected rights due to inadequate governance safeguards. %TODO: add citation
For instance, democratic failures in Latin America over the last 50 years, stemming from regime changes, economic crises, or environmental catastrophes, have led to the absence of robust governance mechanisms to address such challenges.
One illustrative example is the impact of the last Argentinian military dictatorship, which significantly altered the identities of numerous individuals who were abducted as children and placed with new families, effectively erasing their true identities.
In response, collective organisations emerged to address this injustice, recognising the vulnerable position these individuals were placed in and their limited ability to resist and reclaim their true identities~\citep{gesteira_mas_2014}.

Despite the critical role of trust in upholding the autonomy and agency of data subjects \citep{benshahar_more_2014}, the methods currently employed to foster trust remain contentious, and unresolved societal issues persist in digital services and emerging digital intermediaries \citep{carovano_regulating_2023}.
Given the practical nature of the issues at hand, including how to practically approach trust, establish trust relationships between data subjects and data intermediaries, and identify the necessary conditions for fostering trust, a public Think-In event was organised in the context of the PROTECT project.
In these events, individuals were convened to explore the implications of governing personal data spaces through decentralised PIMS or trusted data intermediaries.
With the ``citizens' Think-In'' approach, there is a public discussion focused on the opinion of individuals, which encourages direct participation from attendees.
In particular, through small-scale group discussions, a Think-In offers a platform for individuals from diverse backgrounds to deliberate and exchange views on current societal issues stemming from advancements in Science, Technology, Engineering, and Mathematics (STEM) fields\footnote{Information regarding the organised PROTECT Think-Ins and respective results is available at \url{https://w3id.org/people/besteves/phd/thinkin} (accessed on 11 March 2024)}. %TODO: add citation of think-ins

While the comprehensive outcomes of the Think-In process will not be included in this Thesis as a contribution, it is worth noting that the general public exhibited sensitivity toward the ethical considerations regarding whom to trust and the significance of transparency in such contexts.
Citizens emphasised the importance of preventing the GDPR from turning into a mere `tick-box' compliance exercise, similar to the current format of privacy notices which result from deploying template privacy notices for distinct data processing activities.
Furthermore, there was a call for increased disclosure and oversight concerning the practical and beneficial utilisation of personal data, highlighting the importance of meaningful transparency in fostering trust among parties involved in such sensitive data exchanges.

To conclude, the insights derived from the Citizens' Think-In discussion offer a valuable foundation for considering the integration of transparency into data access agreement terms for personal data vaults, presented in both machine-readable and human-readable formats.
As such, the proposed vocabulary work, described in Chapter~\ref{chap:vocabularies}, represents a first step to offer said transparency for data subjects, containing both machine-readable and human-readable descriptions of concepts.
This approach enables data subjects to better comprehend and manage the expression of policy terms, and empowers data controllers and data subjects to navigate the intricate data-sharing landscape of the platform economy with greater control vested in the data subject.