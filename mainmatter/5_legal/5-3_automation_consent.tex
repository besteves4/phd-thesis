\section{Can consent be automated?}
\label{sec:automation_consent}

To discuss consent automation, first one should look into the rationale of why it is such an important requirement in personal data protection law.
As per \cite{jarovsky_improving_2018}, the main rationale behind consent is to retain human autonomy and to enable data subjects to have agency regarding the processing of their data.
To achieve that, data subjects must (i) comprehend the circumstances that surround the processing of their data, (ii) decide which is their optimal choice among a variety of options, and (iii) express their choice, while knowing that they can change it at any point in the future.
However, if there are no technical and/or organisational measures in place to preserve individual autonomy in this process, meaningful, freely given consent cannot be achieved due to \textit{``issues of cognitive limitations, information overload, information insufficiency, lack of intervenability and lack of free choice''} \citep{jarovsky_improving_2018}.
\cite{solove_privacy_2012} also outlined a few shortcomings in the self-management of privacy, distinguishing between cognitive limitations related to human decision-making abilities and structural limitations that prevent an adequate cost-benefit analysis of consenting to simultaneous personal data processing activities. 
As such, presenting Solid users with a consent dialogue with the result of the matching for each access request that comes in will result in similar scalability issues for the users \citep{mcdonald_cost_2008}.

One possible solution to the previously identified consent-related issues is to automate some aspects of giving consent \citep{baarslag_automated_2017}.
However, this solution has often been criticised due to the complexity surrounding current personal data processing activities on the Web, which might compromise the validity of consent \citep{jarovsky_improving_2018,solove_murky_2023}.
Consenting is context-dependent and encompasses weighing the risks, likelihood of harms and benefits of several variables involved in a personal data processing activity.
As such, it is difficult to imagine how an automated system can weigh all the arguments in favour and against the processing of personal data, while maintaining the interests and autonomy of the data subject at the center of the decision making algorithm.

Thus, in this Section, the setting of OAC user policies in advance and the matching of such policies with requests for data access is analysed to check if such a system is sufficient to comply with the legal requirements for expressing valid consent.
Consenting is usually a binary choice -- the data subject either agrees with the conditions set by the data controller to process their data or they do not.
Nevertheless, different privacy laws implement this choice in distinct manners: in the United States the `opt-out' choice predominates, i.e., \textit{``organizations post a notice of their privacy practices and people are deemed to consent if they continue to do business with the organization or fail to opt out''}, while in the EU the `opt-in' option prevails, i.e., \textit{``people must voluntarily and affirmatively consent''} \citep{solove_murky_2023}.
As such, the latter involves (i) the data controller requesting consent and (ii) the data subject accepting or rejecting it.
If users set their policies in advance, this order is inverted.
Even though the GDPR does not regulate the interaction between data subjects and software to assist them in expressing consent, as previously mentioned, Recital 32 \citeyearpar{noauthor_regulation_2016} suggests the usage of \textit{``technical settings''} to indicate \textit{``acceptance of the proposed processing of his or her personal data''}.

Moreover, two levels of automation, both triggered by a data request, can be considered: (i) the result of the matching, between user policies and data request, is presented to the user for him/her to consent, or (ii) access to data is given automatically if the data request matches with the user's policies.
The former -- the consent dialogue, based on the policy matching algorithm -- improves the transparency of the processing activity and helps the data subject to make an informed choice, while the latter assists with the issues related to information overload and scalability.

\subsection{Expressing specific consent}
\label{sec:specific_consent}

% TODO: explain function creep and right to informational self-determination in the footnotes and add references
In this Section, the specific character of consent is going to be analysed to understand whether OAC can be used to automate access to data in Solid Pods in a GDPR-aligned manner.
According to Article 4.11 \citeyearpar{noauthor_regulation_2016}, consent must express a specific \textit{``indication of the data subject’s wishes''} that \textit{``signifies agreement to the processing of personal data relating to him or her''}.
However, the wording ``indication of wishes'' is rather vague in the sense that such wishes might be related to the categories of personal data, the purpose for processing, the processing operations, the identity of data controller(s) and/or third party recipients, or their interconnection.
Moreover, the European Court of Justice mentions in Case C-61/19 Orange Romania \citeyearpar{noauthor_orange_2020} states that the data subject’s wishes \textit{``must relate specifically to the processing of the data in question and cannot be inferred from an indication of the data subject’s wishes for other purposes''}.
The EDPB also included guidance on the specificity of consent in its consent guidelines \citep{european_data_protection_board_guidelines_2020}, in particular related to (i) using purpose as a safeguard against `function creep'\footnote{\cite{koops_concept_2021} defines `function creep' as \textit{``an imperceptibly transformative and therewith contestable change in a data-processing system's proper activity''} or, in simpler terms, \textit{``the expansion of a system or technology beyond its original purposes''}.}, (ii) the granularity of consent requests, and (iii) the requirement to provide information related to consent separately from other data processing matters.
Moreover, pursuant to Recital 42 \citeyearpar{noauthor_regulation_2016}, for consent to be informed, the data subject should be aware of, at least, the purpose for the personal data processing and the identity of the controller(s). 
However, the level of detail in which the purpose must be described is not further prescribed in the regulation.
According to \cite{kosta_consent_2013}, the specificity of consent is fulfilled when the relation between personal data and its processing, as well as all other conditions surrounding the processing activities, are explained.
Furthermore, consenting should be as specific as needed for safeguarding the data subject's right to informational self-determination\footnote{The right to informational self-determination was first formulated in German law and it has had a profound impact in European data protection law as it asserts that an individual should have the authority \textit{``to decide fundamentally for herself, when and within what limits personal data may be disclosed, [...]''} \citep{vivarelli_crisis_2020}.}.
As such, the following analysis will be focused on the purpose of processing personal data, as well as on the identity of the data controller, and how they relate to the specific character of consent.

\paragraph{Distinguishing the processing operation from the purpose for processing}

The GDPR explicitly mentions that not only the purpose but also the processing operation needs to be specific to have valid consent.
Article 6.1(a) \citeyearpar{noauthor_regulation_2016} states that \textit{``Processing shall be lawful only if [...] the data subject has given consent to the processing of his or her personal data for one or more specific purposes''}, while Recital 43 \citeyearpar{noauthor_regulation_2016} pinpoints that \textit{``Consent is presumed not to be freely given if it does not allow separate consent to be given to different personal data processing operations despite it being appropriate in the individual case''}.
Hence it is important to distinguish between both.
While processing operations refer to the actions performed over data -- personal data in the case of GDPR --, the purpose expresses the motive or objective of the data controller for processing personal data.
This also means that several processing operations might be needed to reach a purpose and, on the other hand, distinct purposes can be reached through the same operation, with use of data being a case in point since it has a very broad scope.
As such, \textit{``collection, recording, organisation, structuring, storage, adaptation or alteration, retrieval, consultation, use, disclosure by transmission, dissemination or otherwise making available, alignment or combination, restriction, erasure or destruction''} are examples of processing operations set out on Article 4.2 \citeyearpar{noauthor_regulation_2016}, while Article 5.1(e) \citeyearpar{noauthor_regulation_2016} mentions purposes such as \textit{``archiving purposes in the public interest, scientific or historical research purposes or statistical purposes''}.
Moreover, the distinction between both is also made explicit in Recital 32, which states that \textit{``Consent should cover all processing activities carried out for the same purpose or purposes. When the processing has multiple purposes, consent should be given for all of them''}.
As such, this provision can be interpreted as follows: (i) if a processing operation is used to reach more than one purpose, then consent must be obtained for each purpose, and (ii) if multiple processing operations are needed to reach a single purpose, then consent must be obtained for each processing operation.

Moreover, WP 29, in its opinion on the definition of consent, affirmed that \textit{``There is a requirement of granularity of the consent with regard to the different elements that constitute the data processing: it can not be held to cover `all the legitimate purposes' followed by the data controller''} \citep{article_29_data_protection_working_party_opinion_2011}, i.e., consent should be specific in relation to a purpose.
The relation between processing operations and purposes is also further commented on by WP 29 on these guidelines -- \textit{``it should be sufficient in principle for data controllers to obtain consent only once for different operations if they fall within the reasonable expectations of the data subject''} \citep{article_29_data_protection_working_party_opinion_2011}, however, no further guidance is given on what constitutes `reasonable expectations of the data subject'.
These can be identified, for instance through user studies, however, the result of such studies would only be statistically relevant to the average data subject and not to the particular data subject who has to give consent.
Additionally, the contextual integrity theory of privacy developed by Helen Nissenbaum \citep{nissenbaum_privacy_2004} could also be applied to determine the contextual nature of the processing operation.
Nissenbaum states that privacy should be considered a right that the individual has over the appropriate, context-based flow of their personal information according to context-specific social norms.
As such, the context and norms governing the exchange of personal data should be used to calculate whether the data subject's given consent is specific or not.
% TODO: develop more on the contextual integrity theory of privacy developed by Helen Nissenbaum
On the other hand, there is no clear guidance on what should be the granularity of the processing operations.
Furthermore, these guidelines also provide an example of how consent fails to be specific -- data that is collected for the purpose of providing movie recommendations cannot be used to provide targeted advertisements as the former is more specific than the latter.

However, in a later guidance document, WP 29 also mentions that there are no tools to assess the specificity of data processing elements such as the processing operation or the purpose
\citep{article_29_data_protection_working_party_article_2016}.
Furthermore, in its opinion on electronic health records \citep{article_29_data_protection_working_party_working_2007}, WP 29 had already advanced that \textit{```Specific' consent must relate to a well-defined, concrete situation in which the processing of medical data is envisaged. Therefore a `general agreement' of the data subject e.g. to the collection of his medical data for an EHR and to subsequent transfers of these medical data of the past and of the future to health professionals involved in treatment would not constitute consent''}.
This document also reasons that if the purpose for processing changes at some point in time, then the data subject must be notified to re-consent to the new personal data processing activity and provided with information related to the repercussions of rejecting to consent to such changes.

At this point, it can be discussed whether any change in the purpose, however small, means that consent must be given again by the data subject.
Perhaps in the case of a minor change, re-consent could be considered unnecessary, however, the criterion to measure such a change is not clearly defined in the law. 
As such, it is essential to analyse the matching of offers and requests, for access to data stored in Solid Pods, to understand if the specific character of consent is respected.
A policy matching algorithm based on OAC, such as the one detailed in Chapter~\ref{chap:matching}, functions on the basis of subsumption between the data requests and the user policies defined in advance.
Thus, by hypothesis, user policies can be broader than data requests, which can be used to doubt the specificity of the consent.
However, since OAC policies can also include the usage of prohibitions, these can be used to narrow the scope of the data subject consent, making it more specific.

%FROM THE PAPER: The reasoner that implements the matching transforms a preference, expressed through a positive statement (the purpose)\added{,} and one or more negative ones (the prohibitions) into a choice between the available options as it \replaced{materializes}{materialises} the preferences and produces legal effects towards third parties (the data controller).

\paragraph{Applying the purpose limitation principle to assess the specific character of consent}

GDPR's Article 5.1(b) \citeyearpar{noauthor_regulation_2016} specifies the `purpose limitation' principle which states that personal data should be \textit{``collected for specified, explicit and legitimate purposes and not further processed in a manner that is incompatible with those purposes''}.
As such, to fulfil this principle, two requirements should be taken into account: (a) the purpose specification requirement and (b) the non-incompatibility requirement \citep{koning_purpose_2020}. 
Previously, the WP 29, in its opinion on the purpose limitation \citep{article_29_data_protection_working_party_opinion_2013}, stated that all contextual information should be taken into consideration to determine the actual purpose of the personal data handling activity, including the \textit{``common understanding and
reasonable expectations of the data subjects''}.
Furthermore, Article 8 on the `Protection of personal data' of the Charter of Fundamental Rights of the European Union \citeyearpar{noauthor_charter_2000} also states that \textit{``[s]uch data must be processed fairly for specified purposes and on the basis of the consent of the person concerned or some other legitimate basis laid down by law''}.
In essence, with regard to the processing of personal data, the purpose element relates to the `why' the processing is/will be occurring, while the `how' is related to the processing activity operated over the data \citep{koning_purpose_2020}.

The requirement for the purpose of processing to be specific, in the same manner as the requirement for consent to be specific, is connected with control, self-determination, and autonomy \citep{koning_purpose_2020}. 
Additionally, to comply with other principles, such as the `data minimisation' and `storage limitation' principles, the purpose for processing should also be considered \citep{koning_purpose_2020}.
Moreover, the motivation for consent -- and by consequence for the purpose -- to be specific is related to avoiding the broadening or fading of the purpose, which can result in the misuse of personal data by data controllers and recipients and loss of control by data subjects \citep{european_data_protection_board_guidelines_2020}.
Without control, users fail to preserve their autonomy and cannot exercise agency over the processing of their data \citep{jarovsky_improving_2018}.
As such, purpose specification also serves as a measure to mitigate power and information asymmetries.
In the absence of such a measure, the power balance leans towards the data controller as it can use the data in its possession according to its interests.

Moreover, the EDPB also discusses how the specific character of consent can be used to mitigate the risk of function creep.
According to \cite{koops_concept_2021}, the `creep' component is related to the imperceptibility of the change, which deprives the data subject of the chance to oppose the change and assess its consequences.
With the proposed OAC-based system, the risk of function creep is mitigated since the purpose is specifically instantiated in the request for data.
Furthermore, this information can also be directly consulted by the data subjects as the OAC system proposes to store user policies, data requests, and data agreements in their Pod, enabling them to exercise their rights, e.g., the right to withdraw consent.
% TODO: connect with the chapter of DUODRL
Finally, when it comes to the granularity of the purpose for processing, the system proposed in this Thesis can also be easily extended to include a domain-specific taxonomy of purposes, e.g., health, medical, and biomedical research purposes.
% enabling the interface designer to cluster them in categories with different level\added{s} of specificity, as required by the specific context.

\paragraph{Identifying the data controller}

Using OAC, data subjects can specify explicit policies for specific data controllers that are identified at the time these policies are established. 
Nevertheless, to grant such authorisation in advance, data subjects need to receive information when initially accessing a Solid application or have access to it elsewhere for review, for instance, through metadata available on an app store.
Presently, this crucial information is absent from Solid protocol implementations, as depicted in Figure \ref{fig:css}, which displays the current authorisation dialogue presented to Solid users who use the CSS. 
While the app's name and ID are displayed, no additional information such as contact details, policies, or links to policies, is provided.
Furthermore, there is no official Solid app store that includes metadata about the providers and/or developers of these applications.

In this scenario, an access control policy can vary in specificity and in the way it is modelled.
One approach is to mandate that the data subject identifies the data controller by providing their name and contact information, in order to allow them access to personal data.
This approach does not present issues in terms of specificity, although the drawback is that the data subject would need to consent to each new data controller individually.
On the other hand, another option could be to authorise controllers based on specific criteria, such as industry, country of incorporation, or sector.
In this approach, the data subject would establish a set of restrictions for data controllers without specifying concrete entities.
While this option offers flexibility and allows for automation, it is questionable whether it can be considered valid consent as the specificity criterion is not respected.

As such, the initial challenge is related to the moment when the data subject becomes aware of the data controllers' identity.
In the first approach, the data controller's identity is disclosed beforehand, while in the second, the data subject establishes the criteria that a requester must meet and the specific identity of the data controller only becomes accessible when the policy matching algorithm determines that the requester meets these criteria.
This last option allows for automation -- the data subject does not explicitly acknowledge or approve the identity and contact details before granting access, but this information is accessible for review within the data subject's Pod.
Both options are possible with an OAC-based system, but involve different policy modelling and matching features.

By taking a look at GDPR's Recital 42 \citeyearpar{noauthor_regulation_2016}, it is clear that informed consent requires that the data subject is \textit{``aware at least of the identity of the controller and the purposes of the processing for which the personal data are intended''}.
What's more, the \cite{european_data_protection_board_guidelines_2020} specified that if consent is relied upon by multiple controllers, including controllers to which the data was transferred, all should be identified.
Comparably, the \cite{article_29_data_protection_working_party_guidelines_2018} highlights the importance of disclosing the controller's identity by stating that a change of identity should \textbf{always} be communicated to the data subjects, e.g., through a privacy notice, similarly to a change in purpose or how data subjects can exercise their rights.
As such, it can be clearly noted that the specific nature of consent and its informed character are closely linked.
Consequently, consent is likely invalid if the data subject is unaware of the identity of the entity processing their personal data at the time of consenting.

As outlined in Section \ref{sec:oac_notice_automation}, data subjects must receive information about the processing of their data at the point of data collection or during the initial interaction between them and the data controller.
If the decision to grant access is automatically made by the matching algorithm, without the involvement of the data subject, this scenario is unlikely to meet the requirements for valid consent as the data subject is not given the opportunity to make a choice.
However, if the legal basis is not consent, the result of the matching algorithm could serve as an informational mechanism and automation can be allowed.

Indeed, Recital 39 \citeyearpar{noauthor_regulation_2016}, which discusses the lawfulness, fairness, and transparency principle, emphasises that transparency in particular should be reflected in the \textit{``information to the data subjects on the identity of the controller and the purposes of the processing and further information to ensure fair and transparent processing''}.
Transparency is also highlighted in Recital 58 \citeyearpar{noauthor_regulation_2016}, which emphasises its significance in contexts where the involvement of multiple actors and technological complexity makes it challenging for data subjects to understand who is processing their personal data and under which conditions.
As such, both Recitals indicate that disclosing the identity of data controllers is a crucial aspect of fulfilling information and transparency requirements.

\paragraph{Distinguishing the data controller from the recipients of personal data}

GDPR's Articles 13 and 14 \citeyearpar{noauthor_regulation_2016} outline a set of information items that must be shared with the data subject.
Among them, \textit{``the identity and the contact details of the controller and, where applicable, of the controller's representative''} and \textit{``the recipients or categories of recipients of the personal data''} are clearly separated in Articles 13.1(a) and 14.1(a), and Article 13.1(e) and 14.1(e), respectively.
This differentiation implies that, under specific circumstances, recipients may be identified by category rather than by specific identity details.
As such, in this Section, the following legal questions are debated: (i) what distinguishes data controllers from recipients, particularly in a decentralized environment, (ii) are requesters considered data controllers or recipients, and (iii) what is the significance of this distinction.

According to the GDPR (Article 4.9 \citeyearpar{noauthor_regulation_2016}), a recipient is \textit{``a natural or legal person [...] to which the personal data are disclosed, whether a third party or not''}, while a third party is, according to Article 4.10 \citeyearpar{noauthor_regulation_2016}, \textit{``a natural or legal person, [...] other than the data subject, controller, processor and persons who, under the direct authority of the controller or processor, are authorised to process personal data''}.
Moreover, the EDPB's guidelines related to the concepts of controller and processor \citep{european_data_protection_board_guidelines_cp_2020} state that, unlike the defined roles of controller and processor, the GDPR does not establish distinct obligations or responsibilities for recipients and third parties.
These roles are considered relative concepts, describing a relationship with a controller or processor from a particular standpoint.
For instance, when a controller or processor shares data with another entity, regardless of the recipient's role as a controller or processor, they are classified as a recipient.
Previously, the WP 29 guidance on transparency \citep{article_29_data_protection_working_party_guidelines_2018} stipulated that controllers must provide information about the recipients or categories of recipients of personal data.
Additionally, to uphold the principle of fairness, controllers are obligated to furnish data subjects with information on recipients that is most relevant and meaningful to them, which \textit{``in practice, [...] will generally be the named recipients, so that data subjects know exactly who has their personal data''}.
However, WP 29 also acknowledges the option of disclosing the categories of recipients to data subjects.
This would require these categories to be identified as specifically as possible, delineating the type of recipient by referencing their activities, industry, sector, sub-sector, and location.

As for the data controllers, due to their crucial role in the provision of data subject rights, should \textbf{always} be explicitly identified.
Recital 39 \citeyearpar{noauthor_regulation_2016} underlines the importance of transparent communication by stating that individuals should be informed about the risks, rules, safeguards, and rights associated with the processing of their personal data.
Hence, the alignment of user policies with actual data requests empowers individuals to assert their rights despite data subjects not expressly acknowledging or agreeing to the identity and contact details of requesters, since this information is accessible in their Pod, enabling them to exercise their data subject rights of access, rectification, erasure, or consent withdrawal as outlined in Articles 15, 16, 17, and 7(3) of the GDPR \citeyearpar{noauthor_regulation_2016}, respectively.

Thus, the stringent information criteria in GDPR's Articles 13 and 14 \citeyearpar{noauthor_regulation_2016}, requiring the identification of data controllers by name and contact details, appear to be tailored to the current web landscape, where only a handful of entities collect data and subsequently share it with other parties.
Despite that, within the ecosystem facilitated by Solid, data exchange occurs directly between the data subject (via the Solid Pod) and an unspecified number of app providers, developers, and/or other users.
This poses significant challenges in identifying all these entities by name and contact details, in particular for access automation.
In addition, \cite{vogel_stretching_2022} examines this issue within the context of the Data Governance Act and contends that it presents a barrier to offering intermediary services as well.

In summary of this Section, it is probable that consenting to the processing of personal data without explicitly identifying the providers of Solid services, particularly in terms of their identity and contact information, will not meet the criteria for valid consent under the GDPR.
Nevertheless, according to certain interpretations and referencing the WP 29 guidance, Pod-stored policies might function as an informational mechanism that facilitates compliance with Articles 13 and 14 \citeyearpar{noauthor_regulation_2016}.
Furthermore, through registries of entities, e.g., documented in Solid Pods through the PLASMA vocabulary developed in Section \ref{sec:plasma}, information regarding the identity and contact information of both controllers and recipients, as well as other parties, can be kept and consulted by the data subject at any time.

\subsection{Is already-given consent valid for compatible purposes?}
\label{sec:consent_compatibility}

In this Section, the validity of consent is discussed concerning the compatibility of purposes.
To give an example, if a research participant indicates a preference to have their data processed for research on Alzheimer's disease, a type of degenerative disease, can this consent also be applied to a request for utilising their data for research on dementia, also a type of degenerative disease?

As previously outlined in this Section, the proposed matching algorithm running over OAC policies currently relies on subsumption.
This means that if a user policy grants access for purpose A, and a data request for purpose B (a subclass of A) is made, then access should be allowed.
The same principle applies to other matching operations, such as matching on processing operations or categories of personal data.
To illustrate it with the example introduced above, if a participant's user policy permits their data to be used for research on Alzheimer's disease but a request is made for research on degenerative diseases, access is denied since the user policy's purpose is more specific than the request. Conversely, if the user policy allows data usage for research on degenerative diseases and a request specifically for Alzheimer's research is made, access is granted as the request's purpose is more specific.
Another benefit of using OAC is its capability to express prohibitions, enhancing the specificity of consent.
Users can authorise data access for medical research while prohibiting it for specific areas within medical research, such as genetic engineering research.

Nevertheless, OAC currently lacks consideration for the matching of ``compatible purposes''.
For instance, if a participant's user policy permits data usage for Alzheimer's research, and a request is made to utilise the data for dementia research, should access be granted based on compatible purposes?
Since dementia, like Alzheimer's, falls under the category of degenerative diseases, it can be argued that access should indeed be permitted.
As such, introducing a ``compatibility matching'' algorithm to Solid would enhance the model, and in particular the automation of access to data, however, it is crucial to assess the legal implications of this addition.

Firstly, it must be discussed how one can determine the compatibility of purposes and whether users desire such a model to complement the policy matching algorithm that governs access to their data.
To facilitate this matching, particularly in the context of biomedical research, the example use case developed in Chapter~\ref{chap:matching} on policies for health data-sharing can be leveraged.
This development specifies a taxonomy of health-related research purposes, links to other ontologies containing disease taxonomies, and utilises OAC's matching algorithm.
Nonetheless, for the algorithm to assess compatibility, this information must be integrated into the used taxonomies of purposes.
This could be achieved, for example, by incorporating a triple statement in the purpose taxonomy indicating that \texttt{:purposeX :isCompatible :purposeY}.

Within GDPR, the concept of purpose compatibility, or the \textit{``further processing [...] shall [...] not be considered to be incompatible with the initial purposes''} non-incompatibility principle, is addressed in the second aspect of the purpose limitation principle outlined in Article 5.1(b).
The criteria for evaluating compatibility are further detailed in Article 6.4 \citeyearpar{noauthor_regulation_2016} and quoted below:

\begin{enumerate}
    \item [(a)] \textit{``any link between the purposes for which the personal data have been collected and the purposes of the intended further processing;''}
    \item [(b)] \textit{``the context in which the personal data have been collected, in particular regarding the relationship between data subjects and the controller;''}
    \item [(c)] \textit{``the nature of the personal data, in particular, whether special categories of personal data are processed [...];''}
    \item [(d)] \textit{``the possible consequences of the intended further processing for data subjects;''}
    \item [(e)] \textit{``the existence of appropriate safeguards, which may include encryption or pseudonymisation.''}
\end{enumerate}

From a technical viewpoint, the initial two criteria, namely (a) the association between purposes for use and secondary reuse of data and (b) the processing context, could potentially be evaluated through automated methods, as previously discussed.
However, the evaluation required for the third and fourth criteria entails linking the nature of the data with the potential ramifications of its usage for the data subjects.
Regarding the final criterion, automated verification of such safeguards can be partially conducted through certifications.
Nonetheless, determining the adequacy of such measures presents a challenge for automation, as it requires a risk assessment related to data subjects' rights and interests.

Secondly, regarding the legal hurdles associated with employing these criteria to match user policies with data requests from third parties, the non-compatibility requirement stands as a form of usage limitation.
This requirement prohibits the processing of personal data for purposes that are incompatible with the purpose specified at the time of data collection.
Moreover, the requirements for compatibility and usage of an appropriate legal basis are conditions that must be met together, i.e., while purpose compatibility is important, it cannot compensate for a lack of legal ground for processing.
Therefore, if the purposes are found to be compatible but there is no lawful basis for processing, either renewed consent must be obtained or an alternative legal basis must be identified.
In the example used throughout this Section, even if the data subject previously consented to the use of their data for Alzheimer's disease research purposes and the matching algorithm confirmed the compatibility of purposes, this alone would not be sufficient to grant access to the resources in the Pod.
Thus, expressing consent for a specific compatible purpose (such as research for dementia) would also be required.
Additionally, as will be discussed in the next Section, if special categories of data are being processed, an exception under Article 9 of the GDPR \citeyearpar{noauthor_regulation_2016} must also be identified and recorded in consent notices.

Finally, while it cannot substitute for the lack of consent, the assessment of purpose compatibility is still beneficial since it serves as a prerequisite for reusing data for new purposes under different legal bases.
% worth mentioning that if additional triple assertion of compatibiliy can be included in oac implementation, a key point is what degree of transparency and control does the subject have over such changes. if it is in the remit only of the solid provider or trust intermediary, rather than the request or, then this may be an acceptable means for enriching the expression of context and purpose with newly needed knowledge.

\subsection{The intricate boundary between expressing and delegating consent}
\label{sec:consent_delegating}

This Section examines whether an OAC-based system enables data subjects to either express or delegate consent.
Moreover, the question of who is liable in the event of errors in such decentralised data-sharing environments is also debated.

The proposed OAC-based matching algorithm converts a preference regarding personal data processing -- \textit{``An individual’s preferred outcome for a specific privacy-related situation''} -- into a decision -- \textit{``What an individual chooses to do in a specific privacy-related situation among available options''} \citep{colnago_is_2022}.
This aligns with the work of \cite{colnago_is_2022} which primarily focuses on employing these concepts in empirical studies concerning attitudes toward privacy.
However, this differentiation could also be valuable in discussing the proposed matching of users policies with data requests within the framework of Solid.

When establishing their preferences through the OAC profile, data subjects delineate specific conditions under which their personal data may be accessed.
This also implies that they understand that the practical outcome of the matching is determined through a series of actions adhering to those conditions, including subsumption and exclusion.
Thus, consent is extended not only to the categories of data, purposes, and entities acting as data controllers but also to the mechanism that facilitates the transformation of these preferences into choices.

Distinct authors have also explored the delegation of consent from individuals to other entities.
\cite{boers_broad_2015} coined the term \textit{``consent for governance''} by arguing that consent, in particular within the realm of biobanking, should be centered on the governance structure of a biobank rather than on the specific details of individual studies.
\cite{le_metayer_automated_2009} examined the concept of automated consent from the perspective of shifting from consenting to the use of personal data to consenting to the use of a privacy agent, \textit{``a dedicated software that would work as a surrogate and automatically manage consent on behalf of the data subject''}.
\cite{sheehan_can_2011} explores the notion of ethical consent and distinguishes between first-order and second-order decisions.
Second-order decisions are intrinsically different from first-order decisions in that the decision-maker's focus lies on the decision-making process rather than the content of the choice.
Sheehan illustrates this with an example involving ordering food in a restaurant.
He describes a scenario where a group of friends dine together, and before the waiter takes their orders, one of them briefly steps away and asks another person at the table to place an order on their behalf.
When making decisions about delegating decision-making, the individual selects based on factors such as trust in their companions, their knowledge of their taste in food, and the information they possess about the approximate amount they wish to spend.

When applying the differentiation between first- and second-order decisions to OAC, the key inquiry revolves around whether consenting to access conditions and the matching algorithm constitutes a first- or second-order decision.
For instance, if a data subject consents to research for the public interest, this concept is further interpreted by the matching algorithm and the ultimate decision to approve a request becomes detailed and specific.
However, this decision is not directly made by the data subject but rather indirectly, by delegating it to an OAC-based system.
An enhancement suggestion for OAC involves not merely augmenting the ontology with more information, but rather enabling the algorithm to `learn' to make such inferences.
However, in this scenario, consideration has to be given to GDPR's  Article 22 \citeyearpar{noauthor_regulation_2016} concerning automated decision-making.

In cases where consent is delegated rather than directly given by the data subject, it becomes essential to examine the impact of this consent arrangement among the data subject, the agent, and the app provider.
It can be argued that the app provider cannot solely rely on the results of the matching algorithm to demonstrate the obtaining of valid consent.
From the standpoint of private law, this matter is addressed within the framework of mandate agreements.
% TODO: add more information on mandate agreements
According to the `appearance principle', under specific conditions, the intentions expressed by the agent are legally binding over the ones of the data subject.
If the choices made by the agent do not accurately represent the data subject's intentions, any discrepancies are resolved between the data subject and the agent \citep{le_metayer_automated_2009}.
This approach offers legal certainty within contract law.
However, the data subject has more protection under European data protection law.
According to the GDPR, the app provider, acting as the data controller, is responsible for ensuring and demonstrating that consent was validly obtained (Article 6.1(a) and 7.1 \citeyearpar{noauthor_regulation_2016}).
In the context of decentralised systems as Solid, this entails the app provider understanding and documenting the matching process, not just its outcome.
Therefore, the matching algorithm must be transparent and demonstrate how the matching was conducted, allowing the app provider to assess the validity of consent.
However, recording the privacy preferences of the data subject may infringe their privacy as with knowledge of these preferences, the controller can submit targeted requests aligned with those preferences, even though they intend to use it for different purposes.

The liability in case of personal data misuse takes on a distinct framework if the Pod provider is regarded as a separate data controller.
In this scenario, the data subject gives consent to the Pod provider (Controller 1) for the storage and provision of personal data to third parties, under specific conditions.
Following the data subject's instructions, the Pod provider subsequently makes the data available to the app provider (Controller 2).
Each data controller must then ensure that the processing activities (storage, transfer, and further utilisation) are grounded in a valid legal basis~\citep{european_data_protection_board_guidelines_2020}.
Thus, the responsibility for errors in the matching process and for invalid consent could be jointly held by both controllers as per their mutual agreement.
While the GDPR does not stipulate a specific legal form for such arrangements, the EDPB recommends documenting this agreement in a binding document, such as a contract accessible to the data subject~\citep{european_data_protection_board_guidelines_2020}.
%perhaps need to summarise or signpost where this more legal analysis has an impact on the technical aspects of the thesis?