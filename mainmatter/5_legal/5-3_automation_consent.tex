\section{Can consent be automated?}
\label{sec:automation_consent}

To discuss consent automation, first one should look into the rationale of why it is such an important requirement in personal data protection law.
As per \cite{jarovsky_improving_2018}, the main rationale behind consent is to retain human autonomy and to enable data subjects to have agency regarding the processing of their data.
To achieve that, data subjects must (i) comprehend the circumstances that surround the processing of their data, (ii) decide which is their optimal choice among a variety of options, and (iii) express their choice, while knowing that they can change it at any point in the future.
However, if there are no technical and/or organisational measures in place to preserve individual autonomy in this process, meaningful, freely given consent cannot be achieved due to \textit{``issues of cognitive limitations, information overload, information insufficiency, lack of intervenability and lack of free choice''} \citep{jarovsky_improving_2018}.
\cite{solove_privacy_2012} also outlined a few shortcomings in the self-management of privacy, distinguishing between cognitive limitations related to human decision-making abilities and structural limitations that prevent an adequate cost-benefit analysis of consenting to simultaneous personal data processing activities. 
As such, presenting Solid users with a consent dialogue with the result of the matching for each access request that comes in will result in similar scalability issues for the users \citep{mcdonald_cost_2008}.

One possible solution to the previously identified consent-related issues is to automate some aspects of giving consent \citep{baarslag_automated_2017}.
However, this solution has often been criticised due to the complexity surrounding current personal data processing activities on the Web, which might compromise the validity of consent \citep{jarovsky_improving_2018,solove_murky_2023}.
Consenting is context-dependent and encompasses weighing the risks, likelihood of harms and benefits of several variables involved in a personal data processing activity.
As such, it is difficult to imagine how an automated system can weigh all the arguments in favour and against the processing of personal data, while maintaining the interests and autonomy of the data subject at the center of the decision making algorithm.

Thus, in this Section, the setting of OAC user policies in advance and the matching of such policies with requests for data access is analysed to check if such a system is sufficient to comply with the legal requirements for expressing valid consent.
Consenting is usually a binary choice -- the data subject either agrees with the conditions set by the data controller to process their data or they do not.
Nevertheless, different privacy laws implement this choice in distinct manners: in the United States the `opt-out' choice predominates, i.e., \textit{``organizations post a notice of their privacy practices and people are deemed to consent if they continue to do business with the organization or fail to opt out''}, while in the EU the `opt-in' option prevails, i.e., \textit{``people must voluntarily and affirmatively consent''} \citep{solove_murky_2023}.
As such, the latter involves (i) the data controller requesting consent and (ii) the data subject accepting or rejecting it.
If users set their policies in advance, this order is inverted.
Even though the GDPR does not regulate the interaction between data subjects and software to assist them in expressing consent, as previously mentioned, Recital 32 \citeyearpar{noauthor_regulation_2016} suggests the usage of \textit{``technical settings''} to indicate \textit{``acceptance of the proposed processing of his or her personal data''}.

Moreover, two levels of automation, both triggered by a data request, can be considered: (i) the result of the matching, between user policies and data request, is presented to the user for him/her to consent, or (ii) access to data is given automatically if the data request matches with the user's policies.
The former -- the consent dialogue, based on the policy matching algorithm -- improves the transparency of the processing activity and helps the data subject to make an informed choice, while the latter assists with the issues related to information overload and scalability.

% TODO: explain function creep and right to informational self-determination in the footnotes and add references
In the next Sections, the specific character of consent is going to be analysed to understand whether OAC can be used to automate access to data in Solid Pods in a GDPR-aligned manner.
According to Article 4.11 \citeyearpar{noauthor_regulation_2016}, consent must express a specific \textit{``indication of the data subject’s wishes''} that \textit{``signifies agreement to the processing of personal data relating to him or her''}.
However, the wording ``indication of wishes'' is rather vague in the sense that such wishes might be related to the categories of personal data, the purpose for processing, the processing operations, the identity of data controller(s) and/or third party recipients, or their interconnection.
Moreover, the European Court of Justice mentions in Case C-61/19 Orange Romania \citeyearpar{noauthor_orange_2020} states that the data subject’s wishes \textit{``must relate specifically to the processing of the data in question and cannot be inferred from an indication of the data subject’s wishes for other purposes''}.
The EDPB also included guidance on the specificity of consent in its consent guidelines \citep{european_data_protection_board_guidelines_2020}, in particular related to (i) using purpose as a safeguard against `function creep'\footnote{\cite{koops_concept_2021} defines `function creep' as \textit{``an imperceptibly transformative and therewith contestable change in a data-processing system's proper activity''} or, in simpler terms, \textit{``the expansion of a system or technology beyond its original purposes''}.}, (ii) the granularity of consent requests, and (iii) the requirement to provide information related to consent separately from other data processing matters.
Moreover, pursuant to Recital 42 \citeyearpar{noauthor_regulation_2016}, for consent to be informed, the data subject should be aware of, at least, the purpose for the personal data processing and the identity of the controller(s). 
However, the level of detail in which the purpose must be described is not further prescribed in the regulation.
According to \cite{kosta_consent_2013}, the specificity of consent is fulfilled when the relation between personal data and its processing, as well as all other conditions surrounding the processing activities, are explained.
Furthermore, consenting should be as specific as needed for safeguarding the data subject's right to informational self-determination\footnote{The right to informational self-determination was first formulated in German law and it has had a profound impact in European data protection law as it asserts that an individual should have the authority \textit{``to decide fundamentally for herself, when and within what limits personal data may be disclosed, [...]''} \citep{vivarelli_crisis_2020}.}.
As such, the following analysis will be focused on the purpose of processing personal data, as well as on the identity of the data controller, and how they relate to the specific character of consent.

\subsection{Distinguishing the processing operation from the purpose for processing}
\label{sec:processing_purposes}

The GDPR explicitly mentions that not only the purpose but also the processing operation needs to be specific to have valid consent.
Article 6.1(a) states that \textit{``Processing shall be lawful only if [...] the data subject has given consent to the processing of his or her personal data for one or more specific purposes''}, while Recital 43 pinpoints that \textit{``Consent is presumed not to be freely given if it does not allow separate consent to be given to different personal data processing operations despite it being appropriate in the individual case''}.
Hence it is important to distinguish between both.
While processing operations refer to the actions performed over data -- personal data in the case of GDPR --, the purpose expresses the motive or objective of the data controller for processing personal data.
This also means that several processing operations might be needed to reach a purpose and, on the other hand, distinct purposes can be reached through the same operation, with use of data being a case in point since it has a very broad scope.
As such, \textit{``collection, recording, organisation, structuring, storage, adaptation or alteration, retrieval, consultation, use, disclosure by transmission, dissemination or otherwise making available, alignment or combination, restriction, erasure or destruction''} are examples of processing operations set out on Article 4.2, while Article 5.1(e) mentions purposes such as \textit{``archiving purposes in the public interest, scientific or historical research purposes or statistical purposes''}.
Moreover, the distinction between both is also made explicit in Recital 32, which states that \textit{``Consent should cover all processing activities carried out for the same purpose or purposes. When the processing has multiple purposes, consent should be given for all of them''}.
As such, this provision can be interpreted as follows: (i) if a processing operation is used to reach more than one purpose, then consent must be obtained for each purpose, and (ii) if multiple processing operations are needed to reach a single purpose, then consent must be obtained for each processing operation.

WP 29, in its opinion on the definition of consent, affirmed that \textit{``There is a requirement of granularity of the consent with regard to the different elements that constitute the data processing: it can not be held to cover `all the legitimate purposes' followed by the data controller''} \citep{article_29_data_protection_working_party_opinion_2011}, i.e., consent should be specific in relation to a purpose.
On the other hand, there is no clear guidance on what should be the granularity of the processing operations.
Furthermore, these guidelines also provide an example of how consent fails to be specific -- data that is collected for the purpose of providing movie recommendations cannot be used to provide targeted advertisements as the former is more specific than the latter.
However, in a later guidance document, WP 29 also mentions that there are no tools to assess the specificity of data processing elements such as the processing operation or the purpose
\citep{article_29_data_protection_working_party_article_2016}.

% In its opinion on electronic health records (EHR\added{s}) \cite{article_29_data_protection_working_party_wp_nodate}, WP 29 provides that `specific' consent must relate to a well-defined, concrete situation in which the processing of medical data is envisaged. Again, the meaning of `concrete situation' is rather vague.
% According to the same opinion, if the reasons for which data are processed by the controller change at some point in time, the user must be notified and put in a position to consent to the new processing of personal data. The information supplied must discuss the repercussions of rejecting the proposed changes in particular.