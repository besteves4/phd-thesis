\section{DGAterms development and evaluation}
\label{sec:dgaterms_dev_ev}

The motivation and identified requirements for the development of the DGAterms, which were outlines in the previous Sections, are consolidated in the vocabularies's ORSD available in Table~\ref{tab:dgaterms_ORSD}.

\begin{table}[htp]
\centering
\caption{Ontology Requirement Specification Document of DGAterms.}
\label{tab:dgaterms_ORSD}
\scriptsize
\resizebox{\textwidth}{!}{%
\begin{tabular}{|l|l|l|l|l|l|l|l|}
\hline
\multicolumn{8}{|c|}{\cellcolor[HTML]{A0A0A0}\textbf{DGAterms -- Vocabulary to describe information flows in the Data Governance Act}} \\ \hline
\multicolumn{8}{|c|}{\cellcolor[HTML]{EFEFEF}\textbf{1. Purpose}} \\ \hline
\multicolumn{8}{| p{12.0cm} |}{The purpose of DGAterms is to provide concepts to describe conditions for the reuse of data held by public sector bodies, to publish and maintain public registers of entities and records of their activities, and to record data altruism consent and permission forms.} \\ \hline
\multicolumn{8}{|c|}{\cellcolor[HTML]{EFEFEF}\textbf{2. Scope}} \\ \hline
\multicolumn{8}{| p{12.0cm} |}{The scope of this vocabulary is limited to the flows of information depicted in DGA's Chapters II, III, and IV, and respective records of information that must be maintained by DGA entities in order to fulfil their obligations. DGAterms also promotes the usage of ODRL, OAC, DPV, DCAT, and DCMI Metadata Terms to model policies for the reuse of personal and non-personal data, to keep catalogs of entities, and to represent consent and permission terms.} \\ \hline
\multicolumn{8}{|c|}{\cellcolor[HTML]{EFEFEF}\textbf{3. Implementation Language}} \\ \hline
\multicolumn{8}{| p{12.0cm} |}{RDF, RDFS} \\ \hline
\multicolumn{8}{|c|}{\cellcolor[HTML]{EFEFEF}\textbf{4. Intended End-Users}} \\ \hline
\multicolumn{8}{| p{12.0cm} |}{Providers of single information points, data intermediation and altruism services, and maintainers of public registers of entities.} \\ \hline
\multicolumn{8}{|c|}{\cellcolor[HTML]{EFEFEF}\textbf{5. Intended Uses}} \\ \hline
\multicolumn{8}{| p{12.0cm} |}{
Use 1. Describing entities, and processes involved in the sharing of data compliant with the DGA. \newline 
Use 2. Expressing information regarding the representation of data-sharing policies and consent terms. \newline
Use 3. Generating records of altruistic and data intermediary activities that can be audited by national and European authorities. 
 } \\ \hline
\multicolumn{8}{|c|}{\cellcolor[HTML]{EFEFEF}\textbf{6. Ontology Requirements}} \\ \hline
\multicolumn{8}{|c|}{\cellcolor[HTML]{EFEFEF}\textbf{a. Non-Functional Requirements}}    \\ \hline
\multicolumn{8}{| p{12.0cm} |}{
NFR 1. The ontology is published online with HTML documentation, following W3C's specification format. } \\ \hline
\multicolumn{8}{|c|}{\cellcolor[HTML]{EFEFEF}\textbf{b. Functional  Requirements: Groups of Competency Questions}}  \\ \hline
\multicolumn{8}{|p{12.0cm}|}{
CQD1. Which legal basis can be used for the processing of personal and non-personal data regulated by the DGA? \newline
CQD2. Which processing operations can be performed on the data? \newline
CQD3. Who are the entities involved in DGA information flows? \newline 
CQD4. Which purposes are considered altruistic under the DGA? \newline 
CQD5. Which technical and organisational measures can be used to protect, provide, and ensure interoperability of data? \newline 
CQD6. Which data types can be shared by public sector bodies? \newline 
CQD7. What kinds of notices should be provided by DGA-related entities? \newline
CQD8. Which public registers should be maintained by authorities? \newline 
CQD9. What rights are provided to data subjects and data holders in the context of DGA?
}\\ \hline
\end{tabular}}
\vspace{-0.1in}
\end{table}

The methodology followed to produce the vocabulary is described in Section~\ref{sec:ontology_engineering}.
The vocabulary human-readable documentation and machine-readable file are available at \url{https://w3id.org/dgaterms} using content negotiation.
The HTML documentation includes a description of the terms defined in DGAterms, which was conducted and validated with domain experts, diagrams with graphical representations of the several taxonomies included in the vocabulary, RDF examples of policies, registers of entities, and records of activities that use DGAterms terms.
The vocabulary documentation also includes metadata, such as the identity of the creators and publishers of the vocabulary, the dates of creation and last modification, or the version number.

The source code is hosted at \url{https://w3id.org/dgaterms/repo}, under the CC-BY-4.0 license.
The repository can also be used by DGAterms implementers to suggest new inclusions to the vocabulary and to report bugs through GitHub Issues.

In terms of quality evaluation, the OOPS! tool was used to detect common errors in ontology development, such as missing domain or range properties or missing human-readable annotations.
No critical nor important issues were detected through this evaluation.
Moreover, FOOPS! was used to evaluate the alignment of the developed vocabularies with the FAIR principles.
The following results were obtained:
\begin{itemize}
    \item Findable -- 8/9
    \item Accessible -- 2/3
    \item Interoperable -- 2/3
    \item Reusable -- 8.83/9
    \item FOOPS! overall score -- 91\%
\end{itemize}
% TODO: perform evaluation
These outcomes are aligned with the scores obtained for OAC, PLASMA, and DUODRL.
% and largely exceed the ones computed for DUO (Findable -- 4.50/9; Accessible -- 2/3; Interoperable -- 3/3; Reusable -- 3.50/9; FOOPS! overall score -- 54\%).
Furthermore, DGAterms obtained a good score in all FAIR aspects.
In terms of improvements, DGAterms can be submitted to LOV to be recorded in a public registry of ontologies.
This will improve both the findability and the accessibility of the vocabulary.
