\section{Extending W3C vocabularies to cover DGA requirements}
\label{sec:extending_dga}

As an active contributor in the realm of data protection, the Semantic Web community possesses significant potential to aid with the compliance processes that such a legislation involve.
This potential is based on the opportunities for interoperability that such a Web of Linked Data can support.
In this context, Semantic Web technologies can be utilised:

\begin{itemize}
    \item to model conditions for the reuse of public data;
    \item by data subjects, data holders, and data users to declare data access and usage policies in a machine-readable format; and
    \item by organisations and service providers to fulfil their legal obligations such as maintaining records of the processing activities.
\end{itemize}

Hence, vocabularies like the DPV, ODRL, and DCAT play a pivotal role in these procedures, serving as interoperable frameworks for the expression of legally-aligned documentation.
DCAT and ODRL (including the work developed in OAC) can be used to publish records of activities as data catalogs and conditions for data access and usage as digital policies, respectively.
Meanwhile, DPV provides a comprehensive, openly accessible set of taxonomies for articulating machine-readable metadata regarding the handling and usage of personal data.
Using such solutions facilitates a transition from manual processes to automated ones by leveraging semantic technologies to uphold accuracy and scalability within the data-sharing ecosystems promoted by the DGA.
In the forthcoming Sections, terms from the previously mentioned standards and specifications are employed to represent some of the information items delineated in Section~\ref{sec:dga_flows}.
Additionally, the terms that cannot be expressed with existing vocabularies are provided in an open-source ad-hoc vocabulary -- DGAterms.
Examples to illustrate the practical applicability of the existing and developed semantic vocabularies are also supplied.

\subsection{Policies for the reuse and sharing of public data}
\label{sec:dga_policies}

In Section~\ref{sec:reuse}, it is highlighted that public sector bodies must provide single information point providers with details on their data resources and conditions for their re-usage.
This enables the providers to compile and maintain a searchable data asset list, facilitating data users' search and request for datasets for reuse.
Table~\ref{tab:conditions_public_data_vocabs} presents terms from DPV, DCAT, OAC, and DCMI Metadata Terms that can be repurposed to model some of the concepts identified in Table~\ref{tab:conditions_public_data}.

\begin{table}[ht] 
\centering
\caption{Information items that need to be represented to detail policies concerning the conditions of reuse of public sector bodies' datasets and respective reusable concepts from existing vocabularies.}
\label{tab:conditions_public_data_vocabs}
\begin{tabular}{c||l|l}
\textbf{Article} & \textbf{Information items} &\textbf{Terms from existing vocabularies} \\ \hline\hline
5.1 & Public sector body information & \texttt{dpv:hasName}, \texttt{dpv:hasContact} \\ \hline
5.1 & Competent body information & \texttt{dpv:hasName}, \texttt{dpv:hasContact} \\ \hline
5.2 & Categories of data & \texttt{dpv:hasData}, \texttt{dpv:Data} \\ \hline
5.2 & Purposes for usage and access & \texttt{dpv:hasPurpose}, \texttt{oac:Purpose} \\ \hline
5.3(a) & Nature of data & \begin{tabular}[c]{@{}l@{}}\texttt{dpv:hasData}, \texttt{dpv:AnonymisedData},\\ \texttt{dpv:PseudonymisedData}\end{tabular} \\ \hline
\begin{tabular}[c]{@{}c@{}}5.3(b),\\5.3(c)\end{tabular} & Processing environment & \begin{tabular}[c]{@{}l@{}}\texttt{dpv:ProcessingContext},\\ \texttt{dpv:hasLocation}\\ \texttt{dpv:WithinVirtualEnvironment}, \\ \texttt{dpv:WithinPhysicalEnvironment}\end{tabular} \\ \hline
5.5 & \begin{tabular}[c]{@{}l@{}}Technical and operational\\measures to prevent\\re-identification of data\\holders/subjects\end{tabular} & \texttt{dpv:Deidentification} \\ \hline
5.9 & Third party recipients & \texttt{dpv:ThirdParty} \\ \hline
8.2 & Data format & \texttt{dcat:mediaType}, \texttt{dcterms:format}\\ \hline
8.2 & Data size & \texttt{dcterms:extent} \\          
\end{tabular}
\end{table}

Expanding upon the existing \texttt{dpv:LegalEntity} term, four new classes of entities were introduced in DGAterms, as subclasses, to represent data users, public sector bodies, competent bodies, and single information point providers.
These classes are named \texttt{DataUser}, \texttt{PublicSectorBody}, \texttt{DataReuseCompetentBody}, and \texttt{SingleInformationPointProvider}, respectively.
Additionally, subclasses of \texttt{SingleInformationPointProvider} were created to represent EU, national, regional, local, and sectorial-level single information point providers, as outlined in DGA's Article 8~\citeyearpar{noauthor_regulation_2022}.
Furthermore, to classify the nature of data kept by public sector bodies, as detailed in Article 3.1~\citeyearpar{noauthor_regulation_2022}, four new subclasses of texttt{dpv:Data} -- \texttt{ConfidentialData}, \texttt{CommerciallyConfidentialData}, \texttt{StatisticallyConfidentialData} and \texttt{IntellectualPropertyData} -- were modelled in DGAterms as well.
These subclasses represent data protected through \texttt{CommercialConfidentialityAgreement}s or \texttt{StatisticalConfidentialityAgreement}s, as well as data protected by intellectual property rights.

Beyond the aforementioned additions, the following concepts were included in the DGAterms vocabulary:
\begin{itemize}
    \item \texttt{A5-9} for permissions to transfer, \texttt{A5-11} for model contractual clauses, and \texttt{A5-12} for adequacy decisions were added as subclasses of \texttt{dpv:DataTransferLegalBasis}.
    \item \texttt{DataReusePolicy}, \texttt{DataTransferNotice} and \texttt{ThirdCountryDataRequestNotice} concepts were introduced as subclasses of DPV's policy and notice concepts. These represent the conditions for data reuse and the notice provided to data owners.
    \item \texttt{DataAssetList} and \texttt{DataReuseRequestProcedure} were modelled as subclasses of \texttt{dpv:OrganisationalMeasure} to represent the searchable asset list maintained by \texttt{SingleInformationPointProvider}s and the procedure to request datasets, respectively.
\end{itemize}

To illustrate the application of both established and newly introduced terms, an instance of a \texttt{DataReusePolicy} concerning the dataset located at \url{http://example.com/dataset_001} is presented in Listing~\ref{list:public_sector_body}.
This policy delineates the terms of usage for the dataset, specifying that it can be reused until the end of 2024, more specifically for the purpose of \texttt{ScientificResearch}.
It is noteworthy that this policy, structured as an ODRL offer, outlines the conditions for utilising the dataset without granting any privileges to the data user.
Single information point providers can utilise this policy to maintain an updated catalog of available assets along with their respective usage conditions.
Furthermore, Listing~\ref{list:data_asset_list} offers an example of a \texttt{DataAssetList} produced by a \texttt{SingleInformationPointProvider}, employing both pre-existing and newly devised terms.
This list includes the aforementioned dataset, \url{http://example.com/dataset_001}, supplemented with additional metadata encompassing its data type, the governing policy (\url{http://example.com/offer_publicsectorbody}), data format and size, and any associated fees charged by the dataset publisher.

\begin{listing}[ht]
\caption{Data reuse policy, set by Public Sector Body X, that allows the reuse of a dataset until the end of 2024 for scientific research.}
\label{list:public_sector_body}
\begin{minted}{turtle}
ex:offer_publicsectorbody a odrl:Offer, dgaterms:DataReusePolicy ;
    odrl:uid ex:offer_publicsectorbody ;
    odrl:profile oac: ;
    odrl:permission [
        odrl:target ex:dataset_001 ;
        odrl:action dgaterms:Reuse ;
        odrl:assigner ex:publicsectorbodyX ;
        odrl:constraint [
            odrl:and [
                odrl:leftOperand odrl:dateTime ;
                odrl:operator odrl:lteq ;
                odrl:rightOperand "2024-12-31"^^xsd:date ], [
                odrl:leftOperand oac:Purpose ;
                odrl:operator odrl:isA ;
                odrl:rightOperand dgaterms:ScientificResearch ] ] ] .
ex:publicsectorbodyX a dgaterms:PublicSectorBody ;
    dpv:hasName "Public Sector Body X" ;
    dpv:hasContact "mailto:publicsectorbodyX@email.com" ;
    dgaterms:hasCompetentBody [
        a dgaterms:DataReuseCompetentBody ;
        dpv:hasName "Competent Body X" ;
        dpv:hasContact "mailto:competentbodyX@email.com" ] .
\end{minted}
\end{listing}

\begin{listing}[ht]
\caption{Data asset list maintained by the Single Information Point Provider A.}
\label{list:data_asset_list}
\begin{minted}{turtle}
ex:SIPPA_assets a dgaterms:DataAssetList, dcat:Catalog ;
    dcterms:description "Asset list maintained by SIPPA" ;
    dcterms:created "2023-12-04"^^xsd:date ;
    dcterms:publisher ex:SIPPA ;
    dcat:dataset ex:dataset_001 .
ex:SIPPA a dgaterms:SingleInformationPointProvider .
ex:dataset_001 a dcat:Dataset ;
    dcterms:publisher ex:publicsectorbodyX ;
    dpv:hasData dgaterms:StatisticallyConfidentialData ;
    dcterms:description "Dataset with statistically confidential data" ;
    dcterms:created "2023-12-04"^^xsd:date ; 
    odrl:hasPolicy ex:policy_001 ;
    dgaterms:hasFee "0€"^^xsd:string ;
    dcat:mediaType <https://iana.org/assignments/media-types/text/csv> ;
    dcterms:extent "5.6MB"^^xsd:string .
\end{minted}
\end{listing}

\subsection{Querying DGA-mandated public registers}
\label{sec:dga_registers}

As outlined in Section~\ref{sec:registers}, data intermediation service providers and data altruism organisations are required to provide detailed information about the purpose of their operations to a public register dedicated to such entities. 
This facilitates the creation of a database comprising information about these entities, which can be accessed by data users, holders, or subjects for retrieving or publishing data, e.g., for altruistic endeavors.

From the information items that need to be represented in a public register of data intermediation service providers, as specified in Table~\ref{tab:registers_intermediation}, 7 out of 14 can be defined using DPV, DCAT, and DCMI Metadata Terms declared in Table~\ref{tab:querying_registers}.
In addition, from the 15 identified items that are necessary for a data intermediary to provide such service, 7 can also be described with DPV, DCAT, and DCMI Metadata Terms declared in Table~\ref{tab:querying_registers}.

\begin{table}[htp]
\centering
\caption{Information items that need to be kept to register the activity of data intermediation service providers and respective terms from existing vocabularies that can be reused.}
\label{tab:querying_registers}
\resizebox{\textwidth}{!}{%
\begin{tabular}{c||l|l}
\textbf{Article} & \textbf{Information items} & \textbf{Terms from existing vocabularies} \\ \hline\hline
11.3 & \begin{tabular}[c]{@{}l@{}}Legal representative if not\\established in the EU\end{tabular} & \texttt{dpv:hasRepresentative} \\ \hline
11.6(a) & \begin{tabular}[c]{@{}l@{}}Name of the data intermediation\\services provider\end{tabular} & \texttt{dpv:hasName} \\ \hline
11.6(c) & \begin{tabular}[c]{@{}l@{}}Address of the data intermediation\\services provider\end{tabular} & \texttt{dpv:hasAddress} \\ \hline
11.6(c) & \begin{tabular}[c]{@{}l@{}}Address of the representative of the\\data intermediation services provider\end{tabular} & \texttt{dpv:hasAddress} \\ \hline
11.6.(d) & Public website & \texttt{dcat:landingPage} \\ \hline
11.6(e) & \begin{tabular}[c]{@{}l@{}}Data intermediation services\\provider’s contact persons\end{tabular} & \texttt{dcat:contactPoint} \\ \hline
11.6(e) & \begin{tabular}[c]{@{}l@{}}Data intermediation services\\provider’s contact details\end{tabular} & \texttt{dpv:hasContact} \\ \hline
11.6(f) & \begin{tabular}[c]{@{}l@{}}Description of the data\\intermediation service\end{tabular} & \texttt{dcterms:description} \\ \hline
11.6(g) & Date of the notification & \texttt{dcterms:issued} \\ \hline\hline
\textbf{Article} & \textbf{\begin{tabular}[c]{@{}l@{}}Conditions to provide\\intermediation services\end{tabular}} & \textbf{Terms from existing vocabularies} \\ \hline\hline
12(c) & Date and time of creation of data & \texttt{dcterms:created} \\ \hline
12(c) & Geolocation data & \texttt{dpv:hasLocation} \\ \hline
12(c) & Duration of the activity & \texttt{dpv:hasDuration} \\ \hline
12(d) & Data format & \texttt{dcat:mediaType}, \texttt{dcterms:format} \\ \hline
12(g,h,i,j,l) & \begin{tabular}[c]{@{}l@{}}Technical and organisational\\measures to protect, provision\\and ensure interoperability of data\end{tabular} & \texttt{dpv:hasTechnicalOrganisationalMeasure} \\ \hline
12(k) & \begin{tabular}[c]{@{}l@{}}Notification of unauthorised\\transfer, access or use of the\\non-personal data\end{tabular} & \texttt{dpv:IncidentReportingCommunication} \\ \hline
12(m) & \begin{tabular}[c]{@{}l@{}}Tools to facilitate exercising\\of data subjects' rights\end{tabular} & \texttt{dpv:isExercisedAt} 
\end{tabular}}
\end{table}

As for the data altruism organisations, from the information items that need to be represented in a public register of such organisations, as specified in Table~\ref{tab:registers_altruism}, 5 out of 9 can be defined using the DPV and DCAT terms declared in Table~\ref{tab:querying_registers_b}.
In addition, from the 11 identified items which are necessary for a data altruism organisation to keep records of its activity, 5 can also be described with the DPV and DCMI Metadata Terms declared in Table~\ref{tab:querying_registers_b}.

\begin{table}[ht]
\centering
\caption{Information items that need to be kept to record the activity of data altruism organisations and respective terms from existing vocabularies that can be reused.}
\label{tab:querying_registers_b}
\resizebox{\textwidth}{!}{%
\begin{tabular}{c||l|l}
\textbf{Article} & \textbf{Information items} & \textbf{Terms from existing vocabularies} \\ \hline\hline
19.3 & \begin{tabular}[c]{@{}l@{}}Legal representative if not \\ established in the EU\end{tabular} & \texttt{dpv:hasRepresentative} \\ \hline
19.4(a) & Name of the entity & \texttt{dpv:hasName} \\ \hline
19.4(e) & Address of the entity & \texttt{dpv:hasAddress} \\ \hline
19.4(e) & \begin{tabular}[c]{@{}l@{}}Address of the representative \\ of the entity\end{tabular} & \texttt{dpv:hasAddress} \\ \hline
19.4(f) & Public website & \texttt{dcat:landingPage} \\ \hline
19.4(g) & Entity’s contact persons & \texttt{dcat:contactPoint} \\ \hline
19.4(g) & Entity’s contact details & \texttt{dpv:hasContact} \\ \hline
19.4(i) & Nature of data & \texttt{dpv:hasData} \\ \hline
19.4(i) & Categories of personal data & \texttt{dpv:hasPersonalData}, pd extension \\ \hline\hline
\textbf{Article} & \textbf{Records of altruism activity} & \textbf{Terms from existing vocabularies} \\ \hline\hline
20.1(a) & Data users' contact details & \texttt{dpv:hasContact} \\ \hline
20.1(b) & Duration of the processing of data & \texttt{dpv:hasDuration} \\ \hline
20.2(b) & \begin{tabular}[c]{@{}l@{}}Description of the objectives of \\ general interest\end{tabular} & \texttt{dcterms:description} \\ \hline
20.2(c) & Technical means used for processing & \begin{tabular}[c]{@{}l@{}}\texttt{dpv:hasTechnicalMeasure}, \\ \texttt{dpv:PrivacyPreservingProtocol}\end{tabular} \\ \hline
20.2(d) & \begin{tabular}[c]{@{}l@{}}Summary of the results of the data \\ processing\end{tabular} & \texttt{dpv:hasOutcome} 
\end{tabular}}
\end{table}

In addition to these terms that can be reused from existing standards and specifications, DGAterms models the \texttt{DataIntermediationServiceProvider} concept to specify a data intermediation service provider (as a subclass of \texttt{dpv:LegalEntity}), and three new classes of entities to represent distinct types of intermediaries, i.e., \texttt{DataCooperativeServiceProvider}, \texttt{DataIntermediationServiceProviderForDataHolder}, and \texttt{DataIntermediationServiceProviderForDataSubject}.
The \texttt{DataAltruismOrganisation} concept is also introduced in DGAterms as a subclass of \texttt{dpv:NonProfitOrganisation}.
% Additionally, to be able to classify the nature of the entity, as specified in Article 11.6(b)~\citeyearpar{noauthor_regulation_2022}, five new properties were added to the vocabulary, \texttt{hasLegalStatus}, \texttt{hasForm}, \texttt{hasOwnershipStructure}, \texttt{hasSubsidiary}, and \texttt{hasRegistrationNumber}, to represent information regarding the legal status, form, ownership structure, subsidiary and registration number of an entity, respectively.
Details concerning the nature of the entity, as outlined in Article 11.6(b)~\citeyearpar{noauthor_regulation_2022} regarding its legal status, form, ownership structure, subsidiary relationships, and registration number, fall beyond the scope of DGAterms as they pertain to organisational-specific items.
However, as future work, upper ontologies like GIST \citeyearpar{semantic_arts_gist} or Schema.org \citep{guha_schemaorg_2015} can be examined and potentially extended to encompass these concepts if deemed necessary.
Moreover, a \texttt{PublicRegister} class was also modelled, as a new subclass of DPV's organisational measures, and its respective subclasses \texttt{PublicRegisterOfDataIntermediationServiceProviders} and \texttt{PublicRegisterOfDataAltruismOrganisations} to represent public registers of data intermediaries and altruistic organisations, respectively.

An illustration of a register of data intermediation service providers, incorporating both existing and newly introduced terms, is available in Listing~\ref{list:intermediary_register}.
The register \texttt{ex:publicregistry\_DI\_PT} presents a comprehensive list of intermediaries operating in Portugal.
In addition to storing metadata concerning the national authority \texttt{ex:nationalauthority\_PT} and creation dates, the register already includes a registered company, \texttt{ex:DISP\_Y}, which operates as a \texttt{DataCooperative}.
DPV's \texttt{hasName}, \texttt{hasContact}, and \texttt{hasAddress}, along with DCAT's \texttt{landingPage}, are used to detail the providers offering data intermediation services.
On the other hand, DCMI's \texttt{description}, \texttt{created}, and \texttt{publisher} predicates are essential in describing metadata pertaining to the register itself.

\begin{listing}[ht]
\caption{Example of a public register of data intermediation service providers.}
\label{list:intermediary_register}
\begin{minted}{turtle}
ex:publicregistry_DI_PT a dgaterms:RegisterOfDataIntermediationServiceProviders ;
    dcterms:description "Public register of intermediaries in PT" ;
    dcterms:created "2023-12-15"^^xsd:date ; 
    dcterms:modified "2023-12-23"^^xsd:date ;
    dcterms:publisher ex:nationalauthority_PT ;
    dgaterms:hasDataIntermediationServiceProvider ex:DISP_Y .
ex:nationalauthority_PT a dgaterms:DataIntermediationAuthority ;
    dpv:hasName "Data Intermediation Authority of Portugal" ;
    dpv:hasContact "mailto:nationalauthority_PT@email.com" ;
    dpv:hasJurisdiction "PT" .
ex:DISP_Y a dgaterms:DataCooperative ;
    dpv:hasName "Data Cooperative Y" ;
    dpv:hasAddress "Lisboa, Portugal" ;
    dcterms:description "Provider of anonymised geolocation data" ;
    dcat:landingPage <http://cooperativeA.com/> ;
    dcterms:date "2023-12-23"^^xsd:date .
\end{minted}
\end{listing}

Storing the public registers in RDF format, utilising existing and crafted semantic vocabularies such as DGAterms, facilitates seamless querying of these structures, using a language like SPARQL, for automated retrieval of information concerning data intermediation service providers (or of data altruism organisations).
An example query targeting data cooperatives is presented in Listing~\ref{list:cooperatives_sparql}, which returns the providers offering such services, along with their names and public websites.

\begin{listing}[ht]
\caption{SPARQL query to retrieve data cooperatives.}
\label{list:cooperatives_sparql}
\begin{minted}{sparql}
SELECT DISTINCT ?Provider ?Name ?Web WHERE {
    ?Provider a dgaterms:DataCooperative .
    ?Provider dpv:hasName ?Name .
    ?Provider dcat:landingPage ?Web . }
\end{minted}
\end{listing}

Furthermore, beyond public registers, as mentioned in Section~\ref{sec:registers}, intermediaries and altruism organisations must keep records of their respective activities for the competent authorities to verify their compliance with DGA-mandated requirements for providing such services.
In Listing~\ref{list:activity_log}, a representation of a record of data altruism activity is presented, using the modelled \texttt{RecordOfDataIAltruismActivity}.
These activity logs should be linked with the entities utilising the data and can be documented utilising DPV's \texttt{hasPersonalDataHandling}. 
This property can be used to supply details regarding the data processing, encompassing aspects such as duration, purpose, and categories of (personal) data involved.

\begin{listing}[ht]
\caption{Example of a record of data altruism activity logs.}
\label{list:activity_log}
\begin{minted}{turtle}
ex:altruism_logs a dgaterms:RecordOfDataIAltruismActivity ;
    dcterms:description "Logs of Data Altruism Organisation A" ;
    dcterms:created "2023-11-04"^^xsd:date ; 
    dcterms:modified "2023-11-13"^^xsd:date ;
    dcterms:publisher ex:altruism_A ;
    dcat:record ex:log_001 .
ex:altruism_A a dgaterms:DataAltruismOrganisation ;
    dpv:hasName "Data Altruism Organisation A" ;
    dpv:hasAddress "Lisboa, Portugal" ;
    dcat:landingPage <http://example.com/altruism_A> .
ex:log_001 a dcat:CatalogRecord ;
    dcterms:created "2023-11-13"^^xsd:date ;
    dgaterms:hasDataUser ex:userZ ;
    dgaterms:hasFee "1000€"^^xsd:string ;
    dpv:hasPersonalDataHandling [
        dcterms:description "Download and reuse anonymised health records to improve healthcare" ;
        dpv:hasProcessing dgaterms:Download,dgaterms:Reuse ;
        dpv:hasDuration 6226453 ;
        dpv:hasPurpose dgaterms:DataAltruism,dgaterms:ImproveHealthcare ;
        dpv:hasPersonalData pd:HealthRecord ;
        dpv:hasTechnicalMeasure dpv:Anonymisation ] .
ex:userZ a dgaterms:DataUser ;
    dpv:hasName "Data User Z" ;
    dpv:hasContact "mailto:user_z@email.com" .
\end{minted}
\end{listing}

\subsection{Uniform data altruism forms}
\label{sec:dga_forms}

As demonstrated by the examples presented in the preceding Sections, the identified vocabularies, along with the one developed in this Chapter, are suitable for automating the generation of compliance documentation for single information point providers, data altruism organisations, and data intermediation service providers alike.
Following these steps, the representation of consent forms for data subjects and permission forms for data holders should also be possible to automate with such technologies.
Through their (re)use, the DGA-mandated European data altruism forms can be ensured to be interoperable across the EU.
An instance of a consent form, utilising the DGAterms-modelled \texttt{EuropeanDataAltruismConsentForm}, completed by a data subject, is detailed in Listing~\ref{list:european_form_ds}.
In this example, an altruistic purpose for processing is used.
To this end, a taxonomy of \texttt{DataAltruism} concepts, modelled from DGA's Article 2.16~\citeyearpar{noauthor_regulation_2022}, are presented in DGAterms, with seven new purposes applicable to a data altruism context: \texttt{ImproveHealthcare}, \texttt{CombatClimateChange}, \texttt{ImproveTransportMobility} (utilised in Listing~\ref{list:european_form_ds}), \texttt{ProvideOfficialStatistics}, \texttt{ImprovePublicServices}, \texttt{ScientificResearch}, and \texttt{PublicPolicyMaking}.
Furthermore, additional purposes referenced throughout the DGA are also incorporated into the vocabulary developed in this Chapter.

\begin{listing}[ht]
\caption{Data altruism consent form, where the data subject consents to the usage of their location data for improving mobility.}
\label{list:european_form_ds}
\begin{minted}{turtle}
ex:consentForm_001 a dgaterms:EuropeanDataAltruismConsentForm ;
    dpv:hasIdentifier <http://example.com/consentForm_001> ;
    dpv:hasDataSubject ex:Anne ;
    dpv:isIndicatedBy ex:Anne ;
    dpv:isIndicatedAtTime "2022-12-14"^^xsd:date ;
    dpv:hasPersonalDataHandling [
        dpv:hasPurpose dgaterms:DataAltruism, dgaterms:ImproveTransportMobility ;
        dpv:hasLegalBasis eu-gdpr:A6-1-a ;
        dpv:hasPersonalData pd:Location ;
        dpv:hasProcessing dpv:Use, dpv:Store ;
        dpv:hasDataController [
            a dpv:DataController, dgaterms:DataAltruismOrganisation ;
            dpv:hasName "Company A" ] ] .
\end{minted}
\end{listing}

Similarly, Listing~\ref{list:european_form_dh} presents an example of a permission form issued by a data holder.

\begin{listing}[ht]
\caption{Permission for data altruism where data holder A allows the usage of their anonymised data for the purpose of providing official statistics.}
\label{list:european_form_dh}
\begin{minted}{turtle}
ex:permissionForm_001 a dpv:Permission ;
    dpv:hasIdentifier <http://example.com/permissionForm_001> ;
    dgaterms:hasDataHolder ex:dataHolderA ;
    dpv:isIndicatedBy ex:dataHolderA ;
    dpv:isIndicatedAtTime "2022-12-15"^^xsd:date ;
    dpv:hasPersonalDataHandling [
        dpv:hasPurpose dgaterms:DataAltruism, dgaterms:ProvideOfficialStatistics ;
        dpv:hasLegalBasis dgaterms:A2-6 ;
        dpv:hasData dpv:AnonymisedData ;
        dpv:hasProcessing dpv:Use, dpv:Store ;
        dpv:hasDataController [
            a dpv:DataController, dgaterms:DataAltruismOrganisation ;
            dpv:hasName "Company A" ] ] .
\end{minted}
\end{listing}

\subsection{DGAterms development and evaluation}
\label{sec:dgaterms_dev_ev}

The motivation and identified requirements for the development of the DGAterms, which were outlines in the previous Sections, are consolidated in the vocabularies's ORSD available in Table~\ref{tab:dgaterms_ORSD}.

\begin{table}[htp]
\centering
\caption{Ontology Requirement Specification Document of DGAterms.}
\label{tab:dgaterms_ORSD}
\scriptsize
\resizebox{\textwidth}{!}{%
\begin{tabular}{|l|l|l|l|l|l|l|l|}
\hline
\multicolumn{8}{|c|}{\cellcolor[HTML]{A0A0A0}\textbf{DGAterms -- Vocabulary to describe information flows in the Data Governance Act}} \\ \hline
\multicolumn{8}{|c|}{\cellcolor[HTML]{EFEFEF}\textbf{1. Purpose}} \\ \hline
\multicolumn{8}{| p{12.0cm} |}{The purpose of DGAterms is to provide concepts to describe conditions for the reuse of data held by public sector bodies, to publish and maintain public registers of entities and records of their activities, and to record data altruism consent and permission forms.} \\ \hline
\multicolumn{8}{|c|}{\cellcolor[HTML]{EFEFEF}\textbf{2. Scope}} \\ \hline
\multicolumn{8}{| p{12.0cm} |}{The scope of this vocabulary is limited to the flows of information depicted in DGA's Chapters II, III and IV, and respective records of information that must be maintained by DGA entities in order to fulfil their obligations. DGAterms also promotes the usage of ODRL, OAC, DPV, DCAT, and DCMI Metadata Terms to models policies for the reuse of personal and non-personal data, to keep catalogs of entities and to represent consent and permission terms.} \\ \hline
\multicolumn{8}{|c|}{\cellcolor[HTML]{EFEFEF}\textbf{3. Implementation Language}} \\ \hline
\multicolumn{8}{| p{12.0cm} |}{RDF, RDFS} \\ \hline
\multicolumn{8}{|c|}{\cellcolor[HTML]{EFEFEF}\textbf{4. Intended End-Users}} \\ \hline
\multicolumn{8}{| p{12.0cm} |}{Providers of single information points, data intermediation and altruism services, and maintainers of public registers of entities.} \\ \hline
\multicolumn{8}{|c|}{\cellcolor[HTML]{EFEFEF}\textbf{5. Intended Uses}} \\ \hline
\multicolumn{8}{| p{12.0cm} |}{
Use 1. Describing entities, and processes involved in the sharing of data compliant with the DGA. \newline 
Use 2. Expressing information regarding the representation of data-sharing policies and consent terms. \newline
Use 3. Generating records of altruistic and data intermediary activities which can be audited by national and European authorities. 
 } \\ \hline
\multicolumn{8}{|c|}{\cellcolor[HTML]{EFEFEF}\textbf{6. Ontology Requirements}} \\ \hline
\multicolumn{8}{|c|}{\cellcolor[HTML]{EFEFEF}\textbf{a. Non-Functional Requirements}}    \\ \hline
\multicolumn{8}{| p{12.0cm} |}{
NFR 1. The ontology is published online with HTML documentation, following W3C's specification format. } \\ \hline
\multicolumn{8}{|c|}{\cellcolor[HTML]{EFEFEF}\textbf{b. Functional  Requirements: Groups of Competency Questions}}  \\ \hline
\multicolumn{8}{|p{12.0cm}|}{
CQ1. Which legal basis can be used for the processing of personal and non-personal data regulated by the DGA? \newline
CQ2. Which processing operations can be performed on the data? \newline
CQ3. Who are the entities involved in DGA information flows? \newline 
CQ4. Which purposes are considered as altruistic under the DGA? \newline 
CQ5. Which technical and organisational measures can be used to protect, provide and ensure interoperability of data? \newline 
CQ6. Which data types can be shared by public sector bodies? \newline 
CQ7. What kinds of notices should be provided by DGA-related entities? \newline
CQ8. Which public registers should be maintained by authorities? \newline 
CQ9. What rights are provided to data subjects and data holders in the context of DGA?
}\\ \hline
\end{tabular}}
\vspace{-0.1in}
\end{table}

The methodology followed to produce the vocabulary is described in Section~\ref{sec:ontology_engineering}.
The vocabulary human-readable documentation and machine-readable file are available at \url{https://w3id.org/dgaterms} using content negotiation.
The HTML documentation includes a description of the terms defined in DGAterms, which was conducted and validated with domain experts, diagrams with graphical representations of the several taxonomies included in the vocabulary, RDF examples of policies, registers of entities, and records of activities that use DGAterms terms.
The vocabulary documentation also includes metadata, such as the identity of the creators and publishers of the vocabulary, the dates of creation and last modification, or the version number.

The source code is hosted at \url{https://w3id.org/dgaterms/repo}, under the CC-BY-4.0 license.
The repository can also be used by DGAterms implementers to suggest new inclusions to the vocabulary and to report bugs through GitHub Issues.

% vocab evaluation