\section{DGAterms}
\label{sec:dgaterms}

As an active contributor in the realm of data protection, the Semantic Web community possesses significant potential to aid with the compliance processes that such a legislation involve.
This potential is based on the opportunities for interoperability that such a Web of Linked Data can support.
In this context, Semantic Web technologies can be utilised:

\begin{itemize}
    \item to model conditions for the reuse of public data;
    \item by data subjects, data holders, and data users to declare data access and usage policies in a machine-readable format; and
    \item by organisations and service providers to fulfil their legal obligations such as maintaining records of the processing activities.
\end{itemize}

Hence, vocabularies like the DPV, ODRL, and DCAT play a pivotal role in these procedures, serving as interoperable frameworks for the expression of legally-aligned documentation.
DCAT and ODRL, both W3C Recommendations, can be used to publish records of activities as data catalogs and conditions for data access and usage as digital policies, respectively.
Meanwhile, DPV provides a comprehensive, openly accessible set of taxonomies for articulating machine-readable metadata regarding the handling and usage of personal data.
%As such, by integrating these standards and specifications, organizations can automate dataset discovery, articulate policies for data reuse and sharing, and ensure compliance with legal mandates such as notifying competent authorities under the DGA.
Using such solutions facilitates a transition from manual processes to automated ones by leveraging semantic technologies to uphold accuracy and scalability within the data-sharing ecosystems promoted by the DGA.

% requirements specification -- ORSD
% vocab implementation details
% Vocabulary publication and maintenance
% vocab evaluation