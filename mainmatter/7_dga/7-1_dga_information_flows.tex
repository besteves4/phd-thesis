\section{Information flows in the DGA}
\label{sec:dga_flows}

In this Section, a detailed description of the DGA and its core legislative objectives is provided, including the introduction of use cases where Semantic Web technologies and decentralised data environments can be leveraged to aid data intermediaries and data altruism organisations to implement their services while fulfilling their renewed legal duties, in particular, related to the reporting of their activities and how they use personal and non-personal data from data subjects and data holders, respectively.

In November 2020, the European Commission (EC) announced a package of new regulation proposals to legislate the European strategy for data \cite{noauthor_communication_2020}.
Among them, the proposal for a Regulation of the European Parliament and of the Council on European data governance, the Data Governance Act (DGA), was proposed to improve data availability and promote trust in data intermediation services across the European Union.
After the approval by the European Parliament and by the European Council, this new law will now be applicable 15 months after its entry into force date, on May 30th, 2022 \citeyearpar{noauthor_regulation_2022}.

% Figure of information flows and documents (from the paper)
% Comparison with GDPR flows
% Identification of important use cases that can be tackled and easily extended with the work already developed for GDPR