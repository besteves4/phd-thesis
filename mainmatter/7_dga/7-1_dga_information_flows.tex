\section{Information flows in the DGA}
\label{sec:dga_flows}

In this Section, a detailed description of the DGA and its core legislative objectives is provided, including the introduction of use cases where Semantic Web technologies and decentralised data environments can be leveraged to aid data intermediaries and data altruism organisations to implement their services while fulfilling their renewed legal duties, in particular, related to the reporting of their activities and how they use personal and non-personal data from data subjects and data holders, respectively.

In November 2020, the \cite{european_commission_communication_2020} introduced a series of regulatory proposals aimed at legislating the European strategy for data.
Among these proposals was the Data Governance Act \citeyearpar{noauthor_regulation_2022}, a regulation presented to enhance data accessibility and foster confidence in data intermediation services throughout the EU.
Following approval by both the European Parliament and the European Council, this legislation entered into force on 23 June 2022 and, following a 15 month grace period, has been applicable since September 2023.
Similarly to other data-related legislation within the EU, the DGA relays new rights and obligations to entities holding both personal and non-personal data.
It also regulates the operations of data users and two categories of data-related services related to data intermediation and altruism.
The principal objectives of this legislation include:
\begin{enumerate}
    \item Facilitating the reuse of protected public-sector data while maintaining its privacy and confidentiality, particularly in cases where such data is subject to the rights of others, including trade secrets, personal data protection, and data safeguarded by intellectual property rights.
    \item Regulating and maintaining a registry of data intermediation service providers, which facilitate data sharing among enterprises and support individuals to have a ``personal data-sharing intermediary'', designed to aid them in exercising their rights, e.g., under the GDPR.
    \item Allowing businesses and data subjects to voluntarily contribute data for altruistic purposes, such as medical research.
    \item Establishing a novel supervisory authority, the European Data Innovation Board, tasked with supervising the operations of data intermediation service providers and data altruism organisations.
\end{enumerate}



% Figure of information flows and documents (from the paper)
% Comparison with GDPR flows
% Identification of important use cases that can be tackled and easily extended with the work already developed for GDPR