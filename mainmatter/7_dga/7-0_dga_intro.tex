\chapter{Going beyond the GDPR -- Exploring the Data Governance Act}
\label{chap:dga}

\begin{tcolorbox}[colback=royallavender!40]
The content of this Chapter has already been partially included in the articles published during this Thesis \citep{esteves_semantifying_2022,esteves_semantics_2023,esteves_towards_2023}.
\end{tcolorbox}

\begin{tcolorbox}[colback=royallavender!10]
The source code produced during the development of this chapter is stored at:
\begin{itemize}
    \item \url{https://w3id.org/dgaterms/repo}
    \item \url{https://w3id.org/people/besteves/soda/repo}
\end{itemize}
\end{tcolorbox}

A new series of regulations concerning the usage and management of data is currently under consideration for adoption across the EU countries.
Within this framework, the Data Governance Act (DGA) \citeyearpar{noauthor_regulation_2022}, a regulation primarily focused on regulating the operations of data intermediation services and data altruism organisations, was approved by the Commission in May 2022 and is now mandated for implementation in all EU member states.
This Chapter outlines a range of requirements outlined by the DGA aimed at safeguarding the interests of both data subjects and data holders -- a new legal role inserted in the DGA to refer to entities providing non-personal data --, as well as regulating the jurisdiction of relevant authorities. 
Additionally, it presents various scenarios where the application of Semantic Web technologies, such as the developed work on OAC, ODRL, and DPV, could assist these stakeholders in meeting their respective new rights and responsibilities.

% This Chapter discusses the legal and ethical challenges of the impact of data-driven innovation in society, in particular, related to the emergence of PIMS as a service that helps individuals have more control over the processing of their data.
% While some studies have recently been published on the intersection of Solid and data protection requirements, as reviewed in Section~\ref{sec:sota_solid_data_protection}, plenty still has to be overcome to have a `legally-aligned' personal datastore.
% This interdisciplinary discussion relies on the collaborations fostered through the PROTECT project, and other EU-funded projects described in Section~\ref{sec:projects}, as well as through the participation in the W3C DPVCG work with data protection law experts.
% 
% Section~\ref{sec:motivation_legal} describes the emergence of decentralised personal information management systems as a way to give users more control over their personal data and the challenges that still need to be overcome in order to have to a GDPR-aligned personal datastore.
% 
% Section~\ref{sec:policies_consent} discusses the usage of OAC policies as a precursor of consent for Solid, which can enable compliance with several GDPR requirements including the transparent information obligations of Articles 13 and 14 and the conditions to obtain valid consent pursuant to Articles 4.11 and 7.
% 
% Section~\ref{sec:automation_consent} argues whether the automation of consent can be performed while maintaining the `informed', `freely given', `specific', and `unambiguous' character of GDPR consent.
% In particular, the specificity of purposes and processing operations, the distinction between data controllers and recipients, the compatibility of purposes, and the delegation of consent are further analysed through a `legal+tech' approach, relying on GDPR's requirements and on the OAC and PLASMA implementations.
% 
% Section~\ref{sec:biomedical_exception} discusses the special requirements of GDPR's special categories of data and research-related exceptions and, in particular, the requirements related to the sharing of health data for biomedical research or for the management of public health.
% 
% To conclude, Section~\ref{sec:ethical_challenges} debates the ethical challenges of controlling data and reclaiming control over it and explores how decentralised PIMS can help build confidence in data exchange practices and trust in the providers and developers of such systems.

\section{Information flows in the DGA}
\label{sec:dga_flows}

In this Section, a detailed description of the DGA and its core legislative objectives is provided, including the introduction of use cases where Semantic Web technologies and decentralised data environments can be leveraged to aid data intermediaries and data altruism organisations to implement their services while fulfilling their renewed legal duties, in particular, related to the reporting of their activities and how they use personal and non-personal data from data subjects and data holders, respectively.

In February 2020, the \cite{european_commission_communication_2020} introduced a series of regulatory proposals aimed at legislating the European strategy for data, encompassing a set of new regulatory proposals aimed at governing the utilisation of non-personal and public data, regulating digital services and markets, and fostering the creation of common European data spaces.
By prioritising data and ensuring its accessibility across sectors, this transformation must also maintain the interests of both data subjects and data holders, while supporting trusted entities to facilitate data sharing aligned with new regulations.
Among these proposals was the Data Governance Act \citeyearpar{noauthor_regulation_2022}, a regulation presented to enhance data accessibility and foster trust in data intermediation services throughout the EU.
Following approval by both the European Parliament and the European Council, this legislation entered into force on 23 June 2022 and, following a 15 month grace period, has been applicable since September 2023.
Similarly to other data-related legislation within the EU, the DGA relays new rights and obligations to entities holding both personal and non-personal data.
It also regulates the operations of data users and two categories of data-related services related to data intermediation and altruism.
The principal objectives of this legislation include:

\begin{enumerate}
    \item[(i)] Facilitating the reuse of protected public-sector data while maintaining its privacy and confidentiality, particularly in cases where such data is subject to the rights of others, including trade secrets, personal data protection, and data safeguarded by intellectual property rights.
    \item[(ii)] Regulating and maintaining a registry of data intermediation service providers, which facilitate data sharing among enterprises and support individuals to have a ``personal data-sharing intermediary'', designed to aid them in exercising their rights, e.g., under the GDPR.
    \item[(iii)] Allowing businesses and data subjects to voluntarily contribute data for altruistic purposes, such as medical research.
    \item[(iv)] Establishing a novel supervisory authority, the European Data Innovation Board, tasked with supervising the operations of data intermediation service providers and data altruism organisations.
\end{enumerate}

The main hurdles to overcome in order to achieve these objectives are associated with:

\begin{enumerate}
    \item[(i)] \textit{Availability/Discovery of datasets}: in the absence of technical assistance for creating data spaces and reliable data-sharing platforms, individuals and organisations will not have tools to share their data for the common good, nor will they have adequate support to exercise their data-related rights. On the other hand, data users lack the necessary tools to search for the data they require.
    \item[(ii)] \textit{Establishment of data access and usage conditions}: with the unavailability of standards and metadata vocabularies to articulate machine-readable policies, setting conditions for the usage and access to personal, non-personal, and public-sector data -- rooted not just in legal frameworks but also in ethical, organisational, and social norms -- will lead to interoperability challenges among entities providing and seeking data access.
    \item[(iii)] \textit{Reporting duties}: without maintaining structured records of their activities, providers of data intermediation services and organisations engaged in data altruism will depend on manual methodologies to generate documentation showcasing their accountable and responsible data-handling practices.
\end{enumerate}

Thus, to tackle the challenges at hand, the first task defined in this Thesis is related to the identification of relevant flows of information between DGA-regulated entities, as well as what specific information items need to be shared or kept by which entities.
Since the DGA both advocates for data availability and legislates data sharing, a delineation of information flows among data-sharing entities can be specified.
In addition, as a result of these interactions, certain registries and records of activities must be maintained for transparency and accountability, in line with previous EU law, e.g., the GDPR.
Figure~\ref{fig:dga_flow} outlines the entities and their corresponding information flows identified within this context.

\begin{figure}[ht]
\centering
\includegraphics[width=\textwidth]{figures/chapter-7/information-flows.png}
\caption[Flows of information between DGA entities.]{Flows of information between DGA entities, adapted from \cite{esteves_semantics_2023}. The concepts in a black box are the (legal) entities and the ones in an orange box are the legal documentation that needs to be created and maintained by said entities. The direction of the arrows represents the direction of the information flow between entities. A short description of each flow is specified on the right side of the Figure.}
\label{fig:dga_flow}
\end{figure}

\subsection{Reuse of protected data held by public sector bodies}
\label{sec:reuse}

DGA's Chapter II legislates the reuse of data stored by public sector bodies.
To keep such data, these entities need to have in place safeguards to protect their commercial and statistical confidentiality, intellectual property rights of third parties, and personal data-related rights.
Moreover, they have to publish the dataset reuse conditions, and the respective data request procedure, in a transparent and publicly accessible manner.
The reuse must be contracted, with a maximum duration of one year, including the categories of data being used as well as the purpose for the reuse. 
Public sector bodies also have the right to obtain guidance and technical support from a competent body, which must be appointed by each EU member state.
These appointed entities are responsible for providing guidance on data formats and storage, assisting in the implementation of privacy-preserving methods to protect personal data integrity, and supporting activities to obtain consent from data subjects and permission from data holders.

\subsection{Data intermediaries}
\label{sec:intermediation}

DGA is also the first of its kind to regulate the activity of data intermediation service providers.
Such entities \textit{``aim[s] to establish commercial relationships for the purposes of data sharing between an undetermined number of data subjects and data holders on the one hand and data users on the other, through technical, legal or other means''} \citeyearpar{noauthor_regulation_2022}.
Article 12 includes a list of conditions for the provision of this service, e.g., providers should have tools to convert data into specific formats, use standards to promote interoperability across sectors, gather data subjects' consent and data holders' permission terms, as well as update or withdraw these terms, and maintain records of their activity.
Before providing this service, intermediation providers must be registered in the public register maintained by the EC, and their activity must be supervised at the national level by a nominated competent authority.

\subsection{Data altruism}
\label{sec:altruism}

Data altruism as a term is also introduced by the DGA.
This term relates to the sharing of personal and non-personal data, based on data subjects' consent and data holders' permission, for the `common good'.
This entails purposes such as improving healthcare, combating climate change, or performing scientific research.
Moreover, the proposed visions for an European Health Data Space \citeyearpar{noauthor_proposal_2022} and the Data Act \citeyearpar{noauthor_dataact_2022} also emphasise the altruistic reuse of data, with the Health Data Spaces proposal in particular focusing on the challenges brought on by the access and sharing of electronic health data for scientific research and public interest purposes.
In this context, each EU member state can establish its altruism policy and has to appoint a competent authority to oversee the activity of altruistic organisations (it can be the same as the one that supervises intermediation providers).
Said authority also has to keep up to date, public registries of altruistic organisations, including details on their identity, contact information, legal status, and main goals.
Data altruism organisations themselves have to maintain records of their activities, in particular, to produce annual reports to share with the relevant competent authority.
To facilitate this activity, a European data altruism consent form will be facilitated \beatriz{by the EC} to \textit{``allow the collection of consent or permission across Member States in a uniform format''} \citeyearpar{noauthor_regulation_2022}.

\subsection{European Data Innovation Board}
\label{sec:edib}

Aligned with the previously outlined regulatory objectives, the DGA established the formation of the European Data Innovation Board (EDIB), a supranational authority tasked with overseeing the operations of data intermediation service providers and data altruism organisations.
This Board comprises representatives from regulatory bodies overseeing data intermediation and altruism activities across all EU member states, from the EDPB and EDPS, from the European Union Agency for Cybersecurity (ENISA) and the European Commission, as well as experts in standardisation, portability, interoperability, and other pertinent stakeholders.
Article 30 \citeyearpar{noauthor_regulation_2022} delineates thirteen specific responsibilities of the EDIB, encompassing initiatives such as fostering uniform practices of data altruism throughout the EU and proposing guidelines for the development of common European data spaces.

% Figure of information flows and documents (from the paper)
% Comparison with GDPR flows
% Identification of important use cases that can be tackled and easily extended with the work already developed for GDPR
\section{DGAterms}
\label{sec:dgaterms}

As an active contributor in the realm of data protection, the Semantic Web community possesses significant potential to aid with the compliance processes that such a legislation involve.
This potential is based on the opportunities for interoperability that such a Web of Linked Data can support.
In this context, Semantic Web technologies can be utilised:

\begin{itemize}
    \item to model conditions for the reuse of public data;
    \item by data subjects, data holders, and data users to declare data access and usage policies in a machine-readable format; and
    \item by organisations and service providers to fulfil their legal obligations such as maintaining records of the processing activities.
\end{itemize}

Hence, vocabularies like the DPV, ODRL, and DCAT play a pivotal role in these procedures, serving as interoperable frameworks for the expression of legally-aligned documentation.
DCAT and ODRL, both W3C Recommendations, can be used to publish records of activities as data catalogs and conditions for data access and usage as digital policies, respectively.
Meanwhile, DPV provides a comprehensive, openly accessible set of taxonomies for articulating machine-readable metadata regarding the handling and usage of personal data.
%As such, by integrating these standards and specifications, organizations can automate dataset discovery, articulate policies for data reuse and sharing, and ensure compliance with legal mandates such as notifying competent authorities under the DGA.
Using such solutions facilitates a transition from manual processes to automated ones by leveraging semantic technologies to uphold accuracy and scalability within the data-sharing ecosystems promoted by the DGA.

% requirements specification -- ORSD
% vocab implementation details
% Vocabulary publication and maintenance
% vocab evaluation
\section{Lessons learned for the (Personal) Data Spaces future}
\label{sec:dgaterms}

contributions towards the EHDS
% \input{mainmatter/5_legal/5-4_biomedical_exceptiosn}
% \section{Ethical challenges of controlling data and reclaiming control over it}
\label{sec:ethical_challenges}

Advancements in data-driven innovations are poised to drive further economic and societal progress~\citep{jacobides_platforms_2019}.
The analysis, sharing, and reuse of data have led to transformative changes in business models and government processes, enabling them to capitalise on these practices. 
As discussed in the previous Sections, these changes propelled policy initiatives implemented by various governments globally.
In particular, the EU is actively engaged in this transformation, exemplified by the \cite{european_commission_communication_2020} commitment\footnote{The European Commission's strategy and related documents are available at \url{https://ec.europa.eu/info/strategy/priorities-2019-2024/europe-fit-digital-age_en} (accessed on 10 March 2024)} to shaping ``A Europe fit for the Digital Age''. 
Whether it is a prominent Big Tech firm headquartered in the United States, a major data intermediary in the EU, or a state-controlled entity in China, contemporary data practices face scrutiny from diverse sectors of society, spanning individuals, non-governmental organisations (NGOs), academics, and governmental bodies.
Such distrust in digital services has been called into question~\citep{waldman_industry_2021}, prompting individuals to ponder who should they trust their data with.

Amidst this trust crisis, technology has emerged as a potential solution, in particular self-sovereign PIMS \citep{chomczyk_penedo_selfsovereign_2021}, as discussed in Section~\ref{sec:motivation_legal}.
These models empower users to directly control their data, dictating the terms of access and usage, and have been gaining the support of policymakers, in particular in Europe, with the European Commission supporting the creation of common European data spaces~\citeyearpar{noauthor_commission_2022}.
Moreover, it could be argued that the EU is strategically investing in these technologies to foster more democratic and participatory data practices, and enhance confidence in data-intensive operations by advocating for technologically robust systems that reduce reliance on the reputation of individual firms, thus mitigating power imbalances between data subjects and controllers~\citep{european_commission_communication_2020}.

The literature exploring the concept of trust is extensive, yet complex due to varying interpretations.
\cite{de_filippi_blockchain_2020} distinguished trust from confidence, noting that trust is rooted in personal vulnerability and risk-taking, while confidence is based on internalised expectations stemming from knowledge or past experiences.
As such, in this Section, the interest of data subjects in technologies that provide insights into how their information is integrated into real personal data handling processes is studied as a vehicle of trust, given their general apprehension regarding the processing actions of data controllers over personal data.
As visible in the previous Sections, the personal data regulatory framework in the EU is designed to address imbalances or vulnerabilities between multiple parties by revealing potential risks and resulting harms, aiming to leverage consent as a catalyst for the data-driven economy \citep{chomczyk_penedo_towards_2022}. 
Simultaneously, they aim to furnish essential information to individuals making decisions, facilitating informed choices \cite{benshahar_more_2014}.
Furthermore, from an ethical standpoint, several norms emerge that should guide the conduct of individuals with whom information is shared to ensure trustworthiness.
These norms encompass sincerity, competence, and the appropriateness of the entrusted task \citep{hawley_how_2019}.

Considering the myriad of factors influencing both trust and confidence, the analysis in this Section focuses on (i) transparency as a crucial prerequisite for the functioning of decentralised PIMS, (ii) the relevant EU regulatory framework on personal data, and (iii) an ethical debate concerning data control, as outlined in \citeauthor{bodo_mediated_2021}'s framework for mediated technological trust.
The emphasis on transparency stems from three primary considerations:
\begin{itemize}
    \item from a regulatory viewpoint, transparency stands as a fundamental principle within personal data protection regimes, often integrated alongside lawfulness and fairness, as exemplified in GDPR's Article 5.1(a);
    \item transparency encompasses both its \textit{ex-ante} and \textit{ex-post} components, with the latter including the issue of explainability \citep{felzmann_transparency_2019};
    \item transparency offers the potential to demystify the `black box' nature of many AI systems, enabling the identification of potential biases towards vulnerable populations \citep{pasquale_black_2015}.
\end{itemize}

As illustrated by case law from supervisory authorities, the intricate nature of data processing activities has proven challenging for data controllers to articulate in straightforward terms, especially when relying on limited attention resources from data subjects \citep{european_data_protection_board_guidelines_2020}.
The dearth of actionable information, to understand data handling practices, poses a risk to fostering trust among involved parties.
As a result, individuals are endeavoring to reassert control over their data and restrict its usage by such entities, also by looking at new data governance schemes such as PIMS or other data intermediaries \citep{craglia_digitranscope_2021,papagiannakopoulou_leveraging_2014}. 

As such, the concept of `control' gains particular importance as users require someone to trust in order to reclaim control over their data in the digital era.
Emerging data governance models are coupled with legal frameworks to assist data subjects in asserting their agency.
For instance, in the data cooperative model (which is regulated by the DGA), cooperatives act as trustees overseeing data on behalf of data subjects, thus enabling data subjects to maintain democratic control over their data. 
In such governance frameworks, establishing a relationship of trust between cooperatives managing data and data subjects is paramount.
In certain instances, trustees may need to consult with data subjects, providing agreements and contracts to inform them. 
Meanwhile, data subjects can articulate their preferences and determine how to share their data and for what purposes~\citep{craglia_digitranscope_2021}.

Data cooperatives and other intermediaries (will) play a pivotal role in empowering data subjects to maintain control over their data and reassert their ethical standing in the digital era.
Specifically, personal data sovereignty offers a significant return to more democratic and egalitarian governance, allowing individuals to reclaim control over their personal data \citep{craglia_digitranscope_2021, giannopoulou_digital_2023}.
In theory, these systems should restore personal autonomy and uphold classical liberal values by fostering trust-based relationships. %TODO: add citation
Furthermore, drawing from our current democratic experiences can offer valuable lessons to avoid repeating the same mistakes made in the past two centuries.
During this time, a substantial portion of the population, particularly in the Global South, suffered from neglected rights due to inadequate governance safeguards. %TODO: add citation
For instance, democratic failures in Latin America over the last 50 years, stemming from regime changes, economic crises, or environmental catastrophes, have led to the absence of robust governance mechanisms to address such challenges.
One illustrative example is the impact of the last Argentinian military dictatorship, which significantly altered the identities of numerous individuals who were abducted as children and placed with new families, effectively erasing their true identities.
In response, collective organisations emerged to address this injustice, recognising the vulnerable position these individuals were placed in and their limited ability to resist and reclaim their true identities~\citep{gesteira_mas_2014}.

Despite the critical role of trust in upholding the autonomy and agency of data subjects \citep{benshahar_more_2014}, the methods currently employed to foster trust remain contentious, and unresolved societal issues persist in digital services and emerging digital intermediaries \citep{carovano_regulating_2023}.
Given the practical nature of the issues at hand, including how to practically approach trust, establish trust relationships between data subjects and data intermediaries, and identify the necessary conditions for fostering trust, a public Think-In event was organised in the context of the PROTECT project.
In these events, individuals were convened to explore the implications of governing personal data spaces through decentralised PIMS or trusted data intermediaries.
With the ``citizens' Think-In'' approach, there is a public discussion focused on the opinion of individuals, which encourages direct participation from attendees.
In particular, through small-scale group discussions, a Think-In offers a platform for individuals from diverse backgrounds to deliberate and exchange views on current societal issues stemming from advancements in Science, Technology, Engineering, and Mathematics (STEM) fields\footnote{Information regarding the organised PROTECT Think-Ins and respective results is available at \url{https://w3id.org/people/besteves/phd/thinkin} (accessed on 11 March 2024)}. %TODO: add citation of think-ins

While the comprehensive outcomes of the Think-In process will not be included in this Thesis as a contribution, it is worth noting that the general public exhibited sensitivity toward the ethical considerations regarding whom to trust and the significance of transparency in such contexts.
Citizens emphasised the importance of preventing the GDPR from turning into a mere `tick-box' compliance exercise, similar to the current format of privacy notices which result from deploying template privacy notices for distinct data processing activities.
Furthermore, there was a call for increased disclosure and oversight concerning the practical and beneficial utilisation of personal data, highlighting the importance of meaningful transparency in fostering trust among parties involved in such sensitive data exchanges.

To conclude, the insights derived from the Citizens' Think-In discussion offer a valuable foundation for considering the integration of transparency into data access agreement terms for personal data vaults, presented in both machine-readable and human-readable formats.
As such, the proposed vocabulary work, described in Chapter~\ref{chap:vocabularies}, represents a first step to offer said transparency for data subjects, containing both machine-readable and human-readable descriptions of concepts.
This approach enables data subjects to better comprehend and manage the expression of policy terms, and empowers data controllers and data subjects to navigate the intricate data-sharing landscape of the platform economy with greater control vested in the data subject.

% How can OAC be extended to create DGA-aligned policies
% Development of a Data Governance Act (DGA) vocabulary, to create OAC-based policies for the sharing of data for altruistic purposes and keep registries of available datasets.