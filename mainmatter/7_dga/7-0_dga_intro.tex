\chapter{Going beyond the GDPR -- Exploring the Data Governance Act}
\label{chap:dga}

\begin{tcolorbox}[colback=royallavender!40]
The content of this Chapter has already been partially included in the articles published during this Thesis \citep{esteves_semantifying_2022,esteves_semantics_2023,esteves_towards_2023}.
\end{tcolorbox}

\begin{tcolorbox}[colback=royallavender!10]
The source code produced during the development of this chapter is stored at:
\begin{itemize}
    \item \url{https://w3id.org/dgaterms/repo}
    \item \url{https://w3id.org/people/besteves/soda/repo}
\end{itemize}
\end{tcolorbox}

After witnessing the influence of the GDPR on managing the personal data of European citizens, the European Commission has shifted its attention to establishing a unified data strategy.
This strategy aims to encourage the (re)use and exchange of data among citizens, businesses, and governments, all while ensuring control remains with the original data generators \citep{european_commission_communication_2020}.
In this context, a new series of regulations concerning the usage and management of data is currently under consideration for adoption across the EU countries.
Within this framework, the Data Governance Act (DGA) \citeyearpar{noauthor_regulation_2022}, a regulation primarily focused on regulating the operations of data intermediation services and data altruism organisations, was approved by the Commission in May 2022 and is now mandated for implementation in all EU member states.

This Chapter outlines a range of requirements outlined by the DGA aimed at safeguarding the interests of both data subjects and data holders -- a new legal role inserted in the DGA to refer to entities providing non-personal data --, as well as regulating the jurisdiction of relevant authorities. 
Additionally, it presents various scenarios where the application of Semantic Web technologies, such as the developed work on OAC, ODRL, and DPV, could assist these stakeholders in meeting their respective new rights and responsibilities.
More specifically, it aims to achieve three main goals: (i) generating machine-readable policies for the reuse of public data, (ii) defining consent and permission terms for data altruism, and (iii) establishing standardised registries of data altruism organisations and intermediation service providers and records of their activities.
By leveraging these semantic vocabularies, the aim is not only to enhance machine-readability and interoperability but also to streamline the modelling of data-sharing policies and consent forms across various scenarios, as well as facilitating the creation of a shared semantic framework for maintaining public registries of data intermediaries and altruism organisations, along with documenting their activities.
Given the accessibility and extensibility of these vocabularies, adapting them to meet specific requirements outlined in the DGA becomes a straightforward task.

% Section~\ref{sec:motivation_legal} describes the emergence of decentralised personal information management systems as a way to give users more control over their personal data and the challenges that still need to be overcome in order to have to a GDPR-aligned personal datastore.
% 
% Section~\ref{sec:policies_consent} discusses the usage of OAC policies as a precursor of consent for Solid, which can enable compliance with several GDPR requirements including the transparent information obligations of Articles 13 and 14 and the conditions to obtain valid consent pursuant to Articles 4.11 and 7.
% 
% Section~\ref{sec:automation_consent} argues whether the automation of consent can be performed while maintaining the `informed', `freely given', `specific', and `unambiguous' character of GDPR consent.
% In particular, the specificity of purposes and processing operations, the distinction between data controllers and recipients, the compatibility of purposes, and the delegation of consent are further analysed through a `legal+tech' approach, relying on GDPR's requirements and on the OAC and PLASMA implementations.
% 
% Section~\ref{sec:biomedical_exception} discusses the special requirements of GDPR's special categories of data and research-related exceptions and, in particular, the requirements related to the sharing of health data for biomedical research or for the management of public health.
% 
% To conclude, Section~\ref{sec:ethical_challenges} debates the ethical challenges of controlling data and reclaiming control over it and explores how decentralised PIMS can help build confidence in data exchange practices and trust in the providers and developers of such systems.

\section{Information flows in the DGA}
\label{sec:dga_flows}

Figure of information flows and documents (from the paper)
Comparison with GDPR flows
Identification of important use cases that can be tackled and easily extended with the work already developed for GDPR
\section{DGAterms}
\label{sec:dgaterms}

As an active contributor in the realm of data protection, the Semantic Web community possesses significant potential to aid with the compliance processes that such a legislation involve.
Such potential is based on the opportunities for interoperability that such a Web of Linked Data can support.
In this context, Semantic Web technologies can be utilised:
\begin{itemize}
    \item to model conditions for the reuse of public data;
    \item by data subjects, data holders and data users to declare data access and usage policies in a machine-readable format; and
    \item by organisations and service providers to maintain records of the processing activities.
\end{itemize}

% requirements specification -- ORSD
% vocab implementation details
% Vocabulary publication and maintenance
% vocab evaluation
\section{Lessons learned for the (Personal) Data Spaces future}
\label{sec:dgaterms}

contributions towards the EHDS