\section{Projects}
\label{sec:projects}

The following projects funded the work presented in this Thesis:

\paragraph{PROTECT ITN:} Protecting Personal Data Amidst Big Data Innovation (PROTECT) is an EU-funded Innovative Training Network project with the goal of developing \textit{``new ways of empowering users of digital services to understand the risks they take when they go online and to offer new ways to enable companies to incorporate data protection into digital services''} and train \textit{``a new generation of 14 early stage researchers who will integrate and apply arguments, analyses, and tools from across the fields of law, ethics and knowledge engineering''}\footnote{Extracted from \url{https://cordis.europa.eu/project/id/813497} (accessed on 15 March 2024).}. This project has received funding from the European Union’s Horizon 2020 research and innovation programme under the Marie Skłodowska-Curie grant agreement No. 813497.

\paragraph{AURORA:} Achieving a new European Energy Awareness (AURORA) is an EU-funded Innovation Action project whose main objective is to \textit{``empower several thousand citizens across five locations in Denmark, England, Portugal, Slovenia, and Spain to make more informed energy decisions''}\footnote{Extracted from \url{https://cordis.europa.eu/project/id/101036418} (accessed on 15 March 2024).}. This project has received funding from the European Union’s Horizon 2020 research and innovation programme under grant agreement No. 101036418.

\paragraph{INESData:} Infraestructura para la Investigación de Espacios de Datos (INESData) is an EU-funded project with the main goal of creating and installing a data governance structure and technological components for common data spaces. This project has received funding from the European Union’s NextGenerationEU funding programme.

\paragraph{COST DKG:} COST Action on Distributed Knowledge Graphs (DKG), with grant agreement No. CA19134, whose main goal is to \textit{``create a research community for deployable Distributed Knowledge Graph technologies that are standards-based, and open, embrace the FAIR principles, allow for access control and privacy protection, and enable the decentralised publishing of high quality data''}\footnote{Extracted from \url{https://www.cost.eu/actions/CA19134/} (accessed on 15 March 2024).}.