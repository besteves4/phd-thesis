\section{Motivation}
\label{sec:motivation}

GDPR came into full effect on the 25th of May 2018 with the main objective of providing the European Union’s natural persons with the right to protection over their personal data, especially in relation to its fair, transparent, and lawful processing and sharing, including a series of rights regarding portability and erasure of data or objection to processing.
This Regulation revolves around the relationship between \textit{`data subjects'} -- the natural personal to which the personal data refers/identifies -- and \textit{`data controllers'} -- the legal entities processing said personal data.
As previously mentioned, a large part of controllers' compliance obligations are related to the data subject’s rights defined in Chapter III of the GDPR.
Information related to these rights should be provided in a concise, transparent, comprehensible, and easily accessible manner, as well as in clear and plain language.
In particular, the so-called \textit{`Right to be informed'}, described in Articles 13 and 14, establishes the information that should be provided to the data subject at the time when the data is first collected, e.g., information on the identity of the controllers, purposes for processing, information on data transfers or existence of automated decision-making.
Companies usually deal with this requirement by providing a description of their personal data-handling services in their privacy policies, which are usually difficult to comprehend due to their length and usage of legal terms.
%This information is therefore naturally connected to the terms that should be provided by data-handling services in their privacy policies, whether personal data is collected directly from the data subject or obtained through other data sources.
Moreover, data subjects should also be provided with information regarding their other rights:
%These information items should also provide information about the other data subject’s rights: 
the right of access to the personal data being processed; the right of rectification of inaccurate personal data; the right to be forgotten, i.e., the data controller has to erase the personal data requested by the subject; the right to restrict the processing of personal data; the right to be notified about the rectification, erasure or restriction of processing; the right to data portability; the right to object to any processing, including profiling; and the right to not be subjected to automated decision-making, including profiling.

As such, machine-readable policy languages seem perfectly fitted to convey these information requirements -- they have been on the Web scene since the 1990s, with the primary goal of establishing the conditions to access Web resources and they already can encode some privacy terms, such as the purpose for processing or recipients.
On the other hand, they are not enough to invoke specific legal terms related to the information that must be shared by data controllers, such as the legal basis for processing or the existing data subject rights.
In this context, a new wave of Semantic Web privacy and data protection vocabularies and ontologies has appeared, which can be used to represent this information, no doubt due to the proliferation of the GDPR and other data privacy-related laws.
Thus, such policy languages and vocabularies can be proven useful to assist data controllers in achieving GDPR alignment for their Web services and to help data subjects in the management of their rights, whether related to the transparent information requirements or their other GDPR-based rights.

More than that, the Semantic Web domain itself is of extreme importance for the representation of these privacy terms as it drives the development of open standards and specifications with interoperability and extensibility in mind.
This effort was and is being led by the World Wide Web Consortium (W3C), openly and collaboratively, with the cooperation of academia and industry.
In this context, two main lines of work are pursued: (i) the development of common formats for data interoperability to ensure seamless integration of data from distinct sources; and (ii) the promotion of a structured language with the ability to document how data relates to real-world objects.
Semantic Web technologies can therefore be applied to a wide range of application fields: data integration, improved search, content management and discovery, domain modelling, or semantic annotation~\citeyearpar{noauthor_semantic_2012}.
Moreover, the term \textit{`Semantic Web'} was first formulated by Sir Tim Berners-Lee, Web inventor and founder of the W3C, with the goal of having a \textit{`Web of Data'}, an extension of the Web of Documents so that data can be shared and reused in a granular manner across applications, companies and the Web community in general~\citep{berners-lee_semantic_2001}.
Solid emerged then as a natural solution to deliver this promise as a Web standards-based decentralised storage environment for data with an integrated, granular access control mechanism~\citep{sambra_solid_2016}.
Such a system allows its users to choose who has access to their data and what applications to use, fulfilling GDPR's requirements of improving data portability and control for data subjects.
However, in order for it to be regarded as a tool that is aligned with the GDPR, the information requirements and other previously mentioned rights need to be considered and integrated into the Solid platform, in the form of user policies, notices, data access agreements, registers of users and applications and logging of Solid-related activities.