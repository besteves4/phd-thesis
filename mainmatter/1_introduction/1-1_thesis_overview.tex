\section{Thesis Overview}
\label{sec:thesis_overview}

The content of this Thesis is based on the research work, and corresponding published articles, developed from January 2020 to December 2023.
\textcolor{blue}{ChatGPT was not used for the development of this Thesis.}

\subsection*{Part I: Introduction}

This Chapter presents the motivation of this Thesis and resulting publications, a set of definitions, as well as the projects and research stays accomplished during the Thesis.

Chapter 2 provides a state of the art on (i) decentralising the access to personal data with Solid, (ii) representing personal data processing information in a machine-readable format, and (iii) using policy languages to specify access control conditions.

Chapter 3 describes the objectives, hypotheses, assumptions, restrictions, research questions, contributions, and methodology followed throughout this Thesis.

\subsection*{Part II: GDPR-Aligned Vocabularies for Personal Datastores}

Chapter 4 describes the developed vocabularies, including (i) an ODRL profile for Access Control (OAC), (ii) a metadata language for Solid (PLASMA), and (iii) rights exercising records using DPV, and includes the ontologies quality evaluation, including the detection of common pitfalls, alignment with FAIR principles and validation of competency questions with SPARQL queries.
% TODO: maybe mention the prototypes here

Chapter 5 includes a legal and ethical discussion, including collaborations with the law and ethics experts in PROTECT, and other EU-funded projects, as well as with the law experts in the W3C DPVCG.
% TODO: ADD TO CHAPTER 4 - an analysis of the alignment of this Thesis with the International Organization for Standardization/International Electrotechnical Commission (ISO/IEC) 27560 standard.

\subsection*{Part III: Algorithms \& Use Cases}

Chapter 6 describes the policy matching algorithm and presents a proof of concept implementation that uses and extends the developed vocabularies to deal with the specific requirements of health data sharing.

%Chapter 7 presents the developed user interfaces and services for policy generation and metadata-keeping.

%Chapter 8 presents a use case validation of this work by applying the developed resources to a health data sharing use case and as a building block for the creation of policies for the new Data Governance Act (DGA).
Chapter 7 presents a use case validation of this work by applying the developed resources as a building block for the creation of policies for the new Data Governance Act (DGA).

\subsection*{Part IV: Conclusions}

Chapter 8 describes the conclusions of this Thesis and presents future work.