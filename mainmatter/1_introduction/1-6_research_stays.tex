\section{Research stays}
\label{sec:research_stays}

The research stays done in the context of this Thesis are outlined below.

\paragraph{01/10/2021 -- 01/12/2021 (2 months):} Research stay at \textit{EmPushy, Dublin, Ireland}\footnote{\url{https://www.empushy.com/} (accessed on 15 March 2024)}, supervised by Dr. Kieran Fraser. During this stay, the Annotation of Push-Notifications (APN) ontology was created to annotate push-notification datasets and train models to identify the presence of personal data in notifications' text, the intent of the notification, its persuasiveness, and so on. As this ontology is out of the scope of this Thesis, its description is omitted from this document. This work resulted in the publication of two conference papers, (PC1) and (PC2)~(\cite{esteves_extracting_2022, esteves_now_2022}, respectively). An analysis of which data to track in EmPushy's tools, and respective GDPR requirements to fulfil, was also performed with the EmPushy team. This stay was funded by the PROTECT ITN.

\paragraph{01/02/2022 -- 31/07/2022 (6 months -- half-time):} Virtual research stay at \textit{Inrupt, Inc., Boston, United States of America}\footnote{\url{https://www.inrupt.com/} (accessed on 15 March 2024)}, supervised by Pat McBennett and Nicolas Mondada. During this stay, an overview of relevant vocabularies related to the Solid ecosystem was performed. Moreover, this Thesis work on the ODRL profile for Access Control (OAC) was improved with the requirements brought by Inrupt's use cases and a Solid application (Solid ODRL access control Policies Editor -- SOPE) was developed to generate and store OAC policies in Solid Pods. This work resulted in the publication of the (PW3) workshop paper~\citep{esteves_using_2022}. This stay was funded by the PROTECT ITN.

\paragraph{01/09/2022 -- 01/12/2022 (3 months):} Research stay at \textit{ADAPT Centre, Trinity College Dublin, Dublin, Ireland}\footnote{\url{https://www.adaptcentre.ie/} (accessed on 15 March 2024)}, supervised by Prof. Dr. Harshvardhan J. Pandit and Prof. Dr. Dave Lewis. During this stay, the Policy LAnguage for Solid’s Metadata-based Access control (PLASMA) was developed. We also contributed to the development of the Data Privacy Vocabulary (DPV) specifications, including writing documentation and use cases, in particular, related to the exercising of data subjects' rights and the new DGA law. This work resulted in the publication of one conference and two workshop papers, (PC4), (PW6), and (PW7) (\cite{esteves_semantics_2023}, \cite{esteves_towards_2023} and \cite{esteves_using_2023}, respectively). This stay was funded by the PROTECT ITN.

\paragraph{01/03/2023 -- 18/03/2023 (~3 weeks):} Short-Term Scientific Mission at \textit{KNoWS, IDLab, Ghent University, Ghent, Belgium}\footnote{\url{https://knows.idlab.ugent.be/} (accessed on 15 March 2024)}, supervised by Prof. Dr. Ruben Verborgh. The main objective of this stay was to discuss and establish technical and legal requirements to align Solid with data protection principles and understand current issues and solutions that need to be dealt with and reused to implement such requirements in decentralised data-sharing environments. This mission was funded by the DKG COST Action.