\section{Evaluation and concluding remarks}
\label{sec:evaluation_conclusions}

\subsection{Ontology quality evaluation}
\label{sec:ontology_evaluation}

This Section describes the outcomes of DUODRL's quality evaluation, including the detection of common pitfalls with OOPS!~\citep{poveda-villalon_oops_2014} and the alignment with FAIR principles with FOOPS!~\citep{garijo_foops_2021}. 
% and validation of competency questions with SPARQL queries \citep{harris_sparql_2013}.

In terms of quality evaluation, the OOPS! tool was used to detect common errors in ontology development, such as missing domain or range properties or missing human-readable annotations.
No critical nor important issues were detected through this evaluation.
Moreover, FOOPS! was used to evaluate the alignment of the developed vocabularies with the FAIR principles.
The following results were obtained:
\begin{itemize}
    \item Findable -- 8/9
    \item Accessible -- 2/3
    \item Interoperable -- 2/3
    \item Reusable -- 8.83/9
    \item FOOPS! overall score -- 91\%
\end{itemize}
% TODO: perform evaluation
These outcomes are aligned with the scores obtained for OAC and PLASMA, and largely exceed the ones computed for DUO (Findable -- 4.50/9; Accessible -- 2/3; Interoperable -- 3/3; Reusable -- 3.50/9; FOOPS! overall score -- 54\%).
Furthermore, DUODRL obtained a good score in all FAIR aspects.
In terms of improvements, DUODRL can be submitted to LOV for it to be recorded in a public registry of ontologies.
This will improve both the findability and the accessibility of the vocabulary.
% Using Table~\ref{tab:foops_evaluation}, it can be observed that both PLASMA and OAC rate much higher that other reused ontologies in terms of findability and reusability as the used URIs and version URIs are persistent and resolvable, both include the recommended ontology and ontology terms metadata, a resolvable data usage license and provenance metadata.

\subsection{Comparison with existing access control systems}
\label{sec:comparison_evaluation}

\subsection{Lessons learned from the implemented PoC}
\label{sec:conclusions_poc}