\chapter{Design of a Policy-based Algorithm for Access to Decentralised Personal Datastores}
\label{chap:matching}

\begin{tcolorbox}[colback=royallavender!40]
The content of this Chapter has already been partially included in the articles published during this Thesis \citep{esteves_odrl_2021,esteves_using_2022,pandit_enhancing_2024}.
\end{tcolorbox}

\begin{tcolorbox}[colback=royallavender!10]
The source code produced during the development of this chapter is stored at:
\begin{itemize}
    \item \url{https://w3id.org/people/besteves/sope/repo}
\end{itemize}
\end{tcolorbox}

This Chapter describes an architecture for a legally-aligned, decentralised personal datastores ecosystem, including the description of a policy matching algorithm and data access agreement generator prototype that uses and extends the developed vocabularies to deal with the specific requirements of health data sharing.
This Chapter builds upon the vocabularies described in Chapter~\ref{chap:vocabularies} to bring decentralised datastore systems, such as Solid, closer to being compliant with data protection law in Europe, by improving its transparency and accountability mechanisms through interoperable, machine-readable information which can be recorded and consulted by data subjects, data controllers and other interested parties.

Section~\ref{sec:architecture} describes a detailed decomposition of the architectural building blocks for a legally-aligned, decentralised personal datastores ecosystem using the C4 graphical notation model \citep{brown_c4_2015}.
System context, container, and component architectural diagrams are provided to report the proposed system, and a sequence diagram is provided to demonstrate how they work together.

% Section~\ref{sec:policies_consent} discusses the usage of OAC policies as a precursor of consent for Solid, which can enable compliance with several GDPR requirements including the transparent information obligations of Articles 13 and 14 and the conditions to obtain valid consent pursuant to Articles 4.11 and 7.
% 
% Section~\ref{sec:automation_consent} argues whether the automation of consent can be performed while maintaining the `informed', `freely given', `specific', and `unambiguous' character of GDPR consent.
% In particular, the specificity of purposes and processing operations, the distinction between data controllers and recipients, the compatibility of purposes, and the delegation of consent are further analysed through a `legal+tech' approach, relying on GDPR's requirements and on the OAC and PLASMA implementations.
% 
% Section~\ref{sec:biomedical_exception} discusses the special requirements of GDPR's special categories of data and research-related exceptions and, in particular, the requirements related to the sharing of health data for biomedical research or for the management of public health.
% 
% To conclude, Section~\ref{sec:ethical_challenges} debates the ethical challenges of controlling data and reclaiming control over it and explores how decentralised PIMS can help build confidence in data exchange practices and trust in the providers and developers of such systems.

\section{Architecture for the deployment of a policy matching algorithm for access control}
\label{sec:architecture}

Design and implementation of a policy matching algorithm and data-sharing agreement generator prototype for access to data stored in Solid Pods.

High level architecture with Pods and data requests from o ther users and so on
Low level architecture with components of policy matching algorithm
UML diagram flow
\section{Design of a policy matching algorithm for generating data access agreements}
\label{sec:algorithm}

In this Section, a detailed overview is given of the proposed algorithms for offer instantiation and policy matching towards the generation of a common data access agreement.
Moreover, a policy editor to define and store OAC policies in Solid Pods is described, as well as a REST API service that uses the results of the policy matching algorithm to facilitate the exercising of data subject rights.

% Table comparing features of OAC with existing Solid solutions + solutions in SotA ?
% Offer, Request, Agreement
% Data, Access, Purpose, Legal Basis

\subsection{Development of an OAC policy editor}
\label{sec:sope}

This Section features an ODRL editor designed to define and save RDF policies, to enable the granting of access to personal data stored in Solid Pods.
RDF policies are articulated using the OAC specification developed in Section~\ref{sec:oac}.
As such, with such an OAC editor, data subjects can create intricate, detailed policies that adhere to GDPR stipulations concerning personal data processing without the burden of knowing ODRL or even RDF.

\paragraph{SOPE -- the Solid ODRL access control Policies Editor}
SOPE is a Solid-based app for data subjects to define and store ODRL policies, based on the OAC specification, on Solid Pods.
Detailed instructions on how to install, launch, and use the app are available on the source code repository\footnote{\url{https://w3id.org/people/besteves/sope/repo}}.
To use it, data subjects must already be in the possession of a Solid Pod and a WebID to be able to log into their Pod and store policies there, using SOPE.
Once logged in, users can select the type of policy they want to model, as well as choose the types of personal data and purposes to which the policy applies.
Additionally, users can also select what type of access they which to provide for said data type and purpose, as well as model further constraints, such as types of recipients that can receive the results of the personal data processing or identity providers used by data requesters to authenticate.
Finally, the prototype interface allows users to generate and store the ODRL policy's RDF in the Pod, without the need for users to be knowledgeable about OAC, ODRL, or DPV.
SOPE also stores policy logs and updates the policy registry in the user's Pod, using the PLASMA vocabulary to model such information, however, in the future, such documentation should be generated by the server to ensure interoperability (by not depending on the app's own implementation, which could diverge from app to app).
All generated information, e.g., policies, logs, and registry, is stored in a private location within the Pod -- only authorised users, apps, or services will have access to it if desired by the data subject.
\beatriz{Figure XX } presents a screenshot of SOPE's interface.

\paragraph{SOPE coverage, maintenance, and future work}
SOPE is published and archived according to the methodology described in Section~\ref{sec:code_preservation}.
Furthermore, SOPE's source code is hosted at \url{https://w3id.org/people/besteves/sope/repo}, under the CC-BY-4.0 license.
A live demonstration of the app's features is also available at \url{https://w3id.org/people/besteves/demo/eswc22}.
The repository can also be used by SOPE users to suggest new features to be added to the app and to report bugs through GitHub Issues.
\beatriz{Table XX } illustrates the current coverage of SOPE concerning the terms implemented in the app from DPV's taxonomies of purposes, personal data categories, processing operations, and recipients.
As future work, SOPE can be extended to include all terms present in the previously mentioned DPV's taxonomies, as well as to cover all constraints defined in the OAC profile, e.g., restrict legal bases or specify the technical and organisational measures used by data controllers to ensure the secure processing of personal data.
Moreover, with such an extension, SOPE could also be used by data controllers to detail their privacy policies.
Additionally, user studies should be performed to assess the design choices included in the editor, as well as to understand what type of additional controls people want to have on top of what is legally mandated, e.g., temporal constraints or duties for the data controller to fulfil prior to data access.

\subsection{Data subject policies as \texttt{odrl:Offer}s}
\label{sec:algorithm-offer}

Data subjects can express policies for specific resources, containers of resources, specific personal data types, or even for data they have not produced yet.
As such, at the time of instantiation, the incoming data request should be used to filter the data subject's policies that apply to that particular request.
This is to ensure that only the needed policies are shared and no more, aligned with the `data minimisation' principle in GDPR's Article 5.1(c) \citeyearpar{noauthor_regulation_2016}, as user policies should also be considered personal data and be treated as such.
Therefore, the resulting \texttt{odrl:Offer} instance contains the union of all pertinent user policies, which can be used to match against incoming data requests in order to generate data access agreements for certain resources or data.
Furthermore, this policy can be used as metadata that accompanies the data being accessed/shared, so that data controllers can keep a copy of the conditions under which they can use the data.
In essence, such a system improves trust and accountability in decentralised data-sharing ecosystems as the policy can travel with the data, with provenance information on who generated the data, who generated the policy, and who attached it to the data.
Even though malicious agents can perform prohibited actions, such as making copies of data when only read-access to data was allowed, with the documentation of access conditions being stored on the datastore of the data subjects, they can use these policies and metadata to file a complaint in the court of law for data misuse.
Moreover, when formulating offers, each distinct rule merged into the policy is preserved as an individual rule to facilitate the matching with data requests.
This preservation of individual rules also enables their individual annotation with provenance metadata, e.g., their origin or whether they are negotiable or non-negotiable rules.

Taking this into consideration, the instantiation of the \texttt{odrl:Offer} should follow the subsequent algorithm:
\begin{enumerate}
    \item For a given personal datastore, retrieve all user preferences and requirements recorded in a datastore policy registry.
    \item Filter out duplicated policies.
    \item Filter out policies that do not match with any terms of the data request.
    \item For each retrieved policy, fetch relevant \texttt{odrl:Permission} and \texttt{odrl:Prohibition} rules and merge them in a single \texttt{odrl:Offer}.
    \item Permissions and/or prohibitions associated with an OAC requirement or an OAC preference should be associated with the term \texttt{dpv:Required}, in case said rule is a requirement, or the term \texttt{dpv:Optional}, in case it is a preference, using the \texttt{dpv:hasContext} property.
    \item A link to each `original' policy is maintained in the final \texttt{odrl:Offer} by using the \texttt{dcterms:source} property.
    \item Add provenance information to the \texttt{odrl:Offer}, e.g. \texttt{dcterms:issued} for when the offer was instantiated and \texttt{dcterms:creator} for the issuer of the policy.
\end{enumerate}

The previously described Listing~\ref{list:oac_offer} presents an example of a result of the offer instantiation algorithm, generated from two existing, relevant policies, as indicated by the \texttt{dcterms:source} property, based on an OAC requirement and OAC preference policies, as expressed by the \texttt{dpv:hasContext} property.

\subsection{Policy matching outcomes as \texttt{odrl:Agreement}s}
\label{sec:algorithm-agreement}

As mentioned in the previous Sections, the instantiated user policies are one of the few bits, representing the will of the data subject, that must be fed to the policy matching algorithm to generate the data access conditions to certain data or resources.
In addition to those, data request policies, expressing the needs and purposes of the data requesters, as well as other contextual information, e.g., date or time of the day, must also be passed on to the algorithm in order to reach a data access agreement that benefits and satisfies both parties.

The recorded outcomes, resulting from the matching process where access to data needs to be either permitted or prohibited, are instances of \texttt{odrl:Agreement}.
Within these agreements, specific ODRL terms play a crucial role in specifying who has granted or denied access (\texttt{odrl:assigner}), to whom (\texttt{odrl:assignee}), for what resources (\texttt{odrl:Asset}), and the associated conditions for access (\texttt{odrl:Rule}).
Moreover, the rules referenced within an agreement mirror the specific rules outlined for a dataset, i.e., through an \texttt{odrl:Offer}, and within a request, i.e., through an \texttt{odrl:Request}).
As such, by being derived from these rules, an agreement should explicitly reference them to assist in the explainability of the algorithm, e.g., in case the data subject wants to know why a certain data access-related decision was made.
An example representation of a data access agreement, depicted as an \texttt{odrl:Agreement} between two parties to read the Pod's data subject age data for academic research, is provided in Listing \ref{list:oac_agreement}.

Taking this into consideration, the generation of the \texttt{odrl:Agreement} should follow the subsequent algorithm:
\begin{enumerate}
    \item Retrieve the data requester's \texttt{odrl:Request} and the user policy's \texttt{odrl:Offer}.
    \item Match the \texttt{odrl:Offer} with the \texttt{odrl:Request}.
    \item Record the outcome of the matching algorithm, where the \texttt{odrl:target} property specifies the data to be accessed, and the \texttt{odrl:assignee} and \texttt{odrl:assigner} properties identify the data subject and the requester, respectively.
    \item If the matching result is positive, i.e., the request and the offer are compatible, then access is permitted by employing a permission with constraints on the requested purpose for access, along with any additional constraints such as legal bases, identity providers or recipients. If access is denied, similar information is included in the policy as a prohibition.
    \item Utilise the \texttt{dcterms:references} property to associate the agreement with the \texttt{odrl:Offer} and \texttt{odrl:Request} that were used to generate it.
    \item Include provenance and other relevant information, such as the \texttt{dcterms:issued} property, to document the creation and acceptance of the agreement among the parties.
\end{enumerate}

The matching process described in step 2 of agreement generation algorithm operates by comparing and assessing the compatibility between the conditions described in user policies and in data access requests.
In the ODRL-based system proposed in this Thesis, this process involves comparing the data subjects' \textit{odrl:Offer}, stored in their Pods, with an \textit{odrl:Request} of a data requester, app, service or agent.
As such, when considering two sets of concepts representing an offer and a request, the matching algorithm may employ two distinct and incompatible approaches to determining access to data.
The first approach, the one proposed in this Thesis to cater to the specific requirements of GDPR consent described in Chapter~\ref{chap:legal}, which is also the more commonly used semantically, involves treating classes as sets and determining access based on set membership.
In this approach, if a class $P$ is a superclass of $C$, a request for accessing $P$ would also allow access to $C$ because every member of $C$ is inherently a member of $P$.
However, a request for accessing $C$ would not grant access to $P$ since not all members of $P$ are necessarily members of $C$.
This method has also been previously utilised in matching policies for GDPR compliance by \cite{bonatti_realtime_2020}.

The second approach is based on determining the applicability of a concept according to its specificity.
In this method, when considering a class $P$ and its subclass $C$, a request for accessing $P$ would not extend access to $C$ since $C$ is more specific.
Conversely, a request for accessing $C$ would grant access to $P$ as $C$ is less specific.
Employing subsumption as a criterion, in the first approach, access is granted when the user offer subsumes the data request, whereas, in the second approach, access is granted when the data request subsumes the user offer.
Hence, both mentioned approaches can be adapted in a decentralised data access ecosystem by reversing the direction of the subsumption in the policy matching algorithm.

Another factor to consider for the matching algorithm involves resolving permissions and prohibitions in terms of their evaluation order and potential conflicts.
Policies can be interpreted in various incompatible ways, such as prioritising permissions and granting access upon the first one that is fulfilled -- a permissive model.
Contrarily, prioritising prohibitions and denying access upon the first fulfilled prohibition is considered a prohibitive model.
In cases where both a permission and a prohibition apply to the same data or resource, conflict resolution is based on the prevalence of one over the other.
In decentralised data environments such as Solid, the matching algorithm follows a prohibitive model, where prohibitions outweigh permissions.
This means that if a request either fails to satisfy a permission or satisfies a prohibition, data access is not granted.
As such, compatibility between offers and requests is only achieved when all permissions are satisfied and all prohibitions remain unsatisfied.

In light of these considerations, the policy matching algorithm described in this Thesis involves examining subsumption or satisfiability between instances of \texttt{odrl:Offer} and \texttt{odrl:Request}.
The algorithm essentially verifies whether the conditions outlined in the user offer are met by the data request policy in the case of permissions, or breached in the case of prohibitions.
If any prohibitions are identified, it indicates that certain conditions of the proposed data request are incompatible with the policies set by the user to govern the access to their personal data.
Conversely, if no prohibitions are found and all permissions are met, the conditions are deemed compatible and access to data can be provided.
In this context, Algorithm~\ref{alg:matching} offers pseudo-code outlining the steps of the proposed policy matching process.

\begin{algorithm}
\caption{Pseudo-code of the proposed OAC-based matching algorithm.}
\label{alg:matching}
\begin{algorithmic}
\For{$prohibition \gets odrl{:}Offer$}
    \If{$offer{:}target \cap request{:}target \neq\emptyset$}
        $decision \gets DENY$
    \EndIf
    \If{$odrl{:}assignee \in offer{:}prohibition$}
        \If{$offer{:}assignee \equiv request{:}assignee$}
            $decision \gets DENY$
        \EndIf
    \EndIf
    \If{$odrl{:}action \in offer{:}prohibition$}
        \If{$offer{:}action \cap request{:}action \neq\emptyset$}
            $decision \gets DENY$
        \EndIf
    \EndIf
    \For{$constraint \gets prohibition$}
        \If{$oac{:}Purpose \gets constraint$}
            \If{$offer{:}Purpose \cap request{:}Purpose \neq\emptyset$}
                $decision \gets DENY$
            \EndIf
        \ElsIf{$oac{:}Recipient \gets constraint$}
            \If{$offer{:}Recipient \cap request{:}Recipient \neq\emptyset$}
                $decision \gets DENY$
            \EndIf
        \ElsIf{$oac{:}LegalBasis \gets constraint$}
            \If{$offer{:}LegalBasis \cap request{:}LegalBasis \neq\emptyset$}
                $decision \gets DENY$
            \EndIf
        \ElsIf{$oac{:}TOM \gets constraint$}
            \If{$offer{:}TOM \cap request{:}TOM \neq\emptyset$}
                $decision \gets DENY$
            \EndIf
        \ElsIf{$oac{:}Technology \gets constraint$}
            \If{$offer{:}Technology \cap request{:}Technology \neq\emptyset$}
                $decision \gets DENY$
            \EndIf
        \ElsIf{$oac{:}IdP \gets constraint$}
            \If{$offer{:}IdP \cap request{:}IdP \neq\emptyset$}
                $decision \gets DENY$
            \EndIf
        \EndIf
    \EndFor
\EndFor

\For{$permission \gets odrl{:}Offer$}
    \If{$offer{:}target \cap request{:}target =\emptyset$}
        $decision \gets DENY$
    \EndIf
    \If{$odrl{:}assignee \in offer{:}permission$}
        \If{$offer{:}assignee \not\equiv request{:}assignee$}
            $decision \gets DENY$
        \EndIf
    \EndIf
    \If{$odrl{:}action \in offer{:}permission$}
        \If{$offer{:}action \cap request{:}action =\emptyset$}
            $decision \gets DENY$
        \EndIf
    \EndIf
    \For{$constraint \gets permission$}
        \If{$oac{:}Purpose \gets constraint$}
            \If{$request{:}Purpose \not\subseteq offer{:}Purpose$}
                $decision \gets DENY$
            \EndIf
        \ElsIf{$oac{:}Recipient \gets constraint$}
            \If{$request{:}Recipient \not\subseteq offer{:}Recipient$}
                $decision \gets DENY$
            \EndIf
        \ElsIf{$oac{:}LegalBasis \gets constraint$}
            \If{$request{:}LegalBasis \not\equiv offer{:}LegalBasis$}
                $decision \gets DENY$
            \EndIf
        \ElsIf{$oac{:}TOM \gets constraint$}
            \If{$request{:}TOM \not\subseteq offer{:}TOM$}
                $decision \gets DENY$
            \EndIf
        \ElsIf{$oac{:}Technology \gets constraint$}
            \If{$request{:}Technology \not\subseteq offer{:}Technology$}
                $decision \gets DENY$
            \EndIf
        \ElsIf{$oac{:}IdP \gets constraint$}
            \If{$request{:}IdP \not\subseteq offer{:}IdP$}
                $decision \gets DENY$
            \EndIf
        \EndIf 
    \EndFor 
\EndFor

\If{$ \nexists DENY$}
    $decision \gets GRANT$
\EndIf
\end{algorithmic}
\end{algorithm}

The proposed algorithm mirrors the previously described prohibitive approach to matching, where the prohibitions outlined in the user offer are examined and ensured to be met before any permissions are considered.
The denial of the access request occurs during prohibition checking if any of the following constraints in the user offer are found to be incompatible with the data request:

\begin{enumerate}
    \item offer target has a data type matching ($\cap\neq\emptyset$) the target in the data request;
    \item offer assignee matches\footnote{Representing permissions and prohibitions of intricate legal entities such as subsidiaries or company groups accurately is not feasible using equality ($=$) or subset ($\subseteq$) relations. Hence, in this Thesis, the equivalence relation ($\equiv$) is used to signify that the data requester must adhere to the legal interpretation of equality -- defining this equality is beyond the scope of this Thesis.} ($\equiv$) the assignee of the data request; 
    \item offer action has an access mode matching ($\cap\neq\emptyset$) the action in the data request;
    \item offer has a purpose matching ($\cap\neq\emptyset$) the purpose in the data request;
    \item offer has a recipient matching ($\cap\neq\emptyset$) the recipient in the data request;
    \item offer has a legal basis matching ($\cap\neq\emptyset$) the legal basis in the data request;
    \item offer has a technical and organisational measure matching ($\cap\neq\emptyset$) the technical and organisational measure in the data request;
    \item offer has a technology matching ($\cap\neq\emptyset$) the technology in the data request; and
    \item offer has an identity provider matching ($\cap\neq\emptyset$) the identity provider in the data request.
\end{enumerate}

If no prohibitions are identified, the next step is to verify the permissions.
The access request will be denied during permission checking if any of the following constraints in the offer are incompatible with the data request:

\begin{enumerate}
    \item request target does not have a data type matching ($\cap=\emptyset$) the target in the offer;
    \item offer assignee does not match ($\not\equiv$) the assignee of the data request; 
    \item request action does not have an access mode matching ($\cap=\emptyset$) the action in the offer;
    \item request purpose is not compatible or a subset ($\not\subseteq$) of the offer purpose, e.g., DPV's \texttt{ResearchAndDevelopment} in a data request does not match DPV's \texttt{AcademicResearch} purpose in an offer as \texttt{ResearchAndDevelopment} is a superclass of \texttt{AcademicResearch} and, as such, less specific;
    \item request recipient is not compatible or a subset ($\not\subseteq$) of the offer recipient;
    \item offer legal basis does not match ($\not\equiv$) the legal basis of the data request;
    \item request technical and organisational measure is not compatible or a subset ($\not\subseteq$) of the offer technical and organisational measure;
    \item request technology is not compatible or a subset ($\not\subseteq$) of the offer technology; and
    \item request identity provider is not compatible or a subset ($\not\subseteq$) of the offer identity provider.
\end{enumerate}

The described procedures are applied to all permissions and prohibitions outlined in the data subject's offer.
If all permissions and prohibitions are met without any violations, access to the data can be authorised.

Table~\ref{tab:oac-matching-examples} showcases a set of data access agreement's outcomes illustrating the functioning of the matching algorithm concerning permissions and prohibitions, focusing on data type and purpose constraints.
In a semantic-based architectural design, evaluating equivalence ($\equiv$), intersection ($\cap$), and subset ($\subseteq$) necessitates additional considerations beyond the mere interpretation of \texttt{owl:sameAs} or \texttt{rdfs:subClassOf} properties. 
For instance, comparing \textit{Academic Research} as a purpose with a data request for \textit{Research and Development} purpose, using subset ($\subseteq$) for permissions or intersection ($\cap$) for prohibition, mandates both purposes to be articulated in a manner enabling such `hierarchical' or `set-based' interpretations.
In such a case, the matching algorithm entails interpreting \textit{Academic Research} as a \textit{narrower concept} or a \textit{subset} of \textit{Research and Development}, a relationship that can be denoted through various semantic properties, from distinct vocabularies, such as \texttt{rdfs:subClassOf}, \texttt{skos:broader}, \texttt{dcterms:isPartOf}, or even an ad-hoc property such as \texttt{ex:specialisationOf}.
Further complexity emerges when considering the compatibility of purposes, as such relationships cannot be specified in a hierarchical manner.

\begin{table}[ht]
\centering
\caption{Examples of outcomes of the policy matching algorithm.}
\label{tab:oac-matching-examples}
\resizebox{\textwidth}{!}{%
\begin{tabular}{c|c|c||c|c||c|c}
\multicolumn{3}{c||}{Offer} & \multicolumn{2}{c||}{Request} & \multicolumn{2}{c}{Outcome} \\
\hline
Rule & Purpose & Data & Purpose & Data & Decision & Reason \\
\hline\hline
Prohibition & \begin{tabular}[c]{@{}c@{}}Academic\\research\end{tabular} & Contact & \begin{tabular}[c]{@{}c@{}}Research and\\development\end{tabular} & Age & DENY & request purpose $\cap$ offer purpose $\neq\emptyset$ \\
\hline
Prohibition & \begin{tabular}[c]{@{}c@{}}Academic\\research\end{tabular} & \begin{tabular}[c]{@{}c@{}}Age\\range\end{tabular} & Payment & Age & DENY & request data $\cap$ offer data $\neq\emptyset$ \\
\hline
Prohibition & \begin{tabular}[c]{@{}c@{}}Academic\\research\end{tabular} & Contact & Payment & Age & GRANT & \begin{tabular}[c]{@{}c@{}}request purpose $\cap$ offer purpose $=\emptyset$\\request data $\cap$ offer data $=\emptyset$\end{tabular} \\
\hline
Permission & \begin{tabular}[c]{@{}c@{}}Academic\\research\end{tabular} & Age & \begin{tabular}[c]{@{}c@{}}Commercial\\research\end{tabular} & Age & DENY & request purpose $\not\subseteq$ offer purpose \\
\hline
Permission & \begin{tabular}[c]{@{}c@{}}Research and\\development\end{tabular} & Age & \begin{tabular}[c]{@{}c@{}}Academic\\research\end{tabular} & Age range & GRANT & \begin{tabular}[c]{@{}c@{}}request purpose $\subseteq$ offer purpose\\request data $\cap$ offer data $\neq\emptyset$\end{tabular} \\
\end{tabular}}
\end{table}

Hence, any implementation of an OAC-based, policy matching algorithm must be aware of such relationships and carefully consider when is makes sense to employ equivalence, intersection, and subset methodologies using established semantic web interpretations such as the \texttt{rdf:type} and \texttt{rdfs:subClassOf} properties.
As such, to facilitate the consistent application and interpretation of the algorithm, a standardised specification of vocabulary terms is essential.
This specification should clarify how concepts are expressed and how they are to be interpreted within the policy matching process.
For instance, it should specify that any purpose term in an offer or request policy \textit{MUST} be an instance of \texttt{Purpose} and \textit{MUST} be associated with at least one concept in the purpose taxonomy using \texttt{rdf:type} or \texttt{rdfs:subClassOf} properties.
By adhering to such a standardised specification, the matching algorithm can rely on these assertions to accurately interpret the constraints included in both offer and request policies.
To achieve this, in this Thesis, the adoption of DPV's taxonomies is strongly encouraged when defining both offer and request terms to ensure accuracy and explainability of the desired outcomes in the policy matching algorithm.

% Future work -- supported by the formal semantics work -- explain how the algorithm took certain decisions, e.g. why the prohibitions were prioritised over the permissions
\section{Proof of concept implementation for health data sharing}
\label{sec:poc_health}

In this Section, a proof of concept implementation of the proposed algorithms for offer instantiation and policy matching towards the generation of a data access agreement is described for a specific use case involving health data sharing.

\subsection{Background \& Motivation}
\label{sec:poc_background}

In this Thesis, a health data sharing use case was selected to showcase the strengths of the proposed algorithm since the exchange of health-related data presents significant potential for advancing research and leveraging advanced computational and statistical techniques to drive progress in healthcare.
However, due to its sensitive nature and potentially significant impact if misused, the sharing and utilisation of health-related data are highly regulated at both legal and institutional levels, e.g., \textit{``data concerning health''} is a special category of personal data under the GDPR, i.e., Article 9.1~\citeyearpar{noauthor_regulation_2016}, and as such its processing is prohibited unless one of the legal grounds of Article 9.2 applies.
Currently, institutions such as hospitals handle each health data request through a dedicated committee tasked with evaluating and making decisions regarding the release of such data under their care.
To facilitate this process, the Global Alliance for Genomics and Health\footnote{\url{https://www.ga4gh.org/} (accessed on 2 April 2024)} (GA4GH) was established as an international consortium focused on the development of standards and the promotion of responsible sharing of genomics and health data.
Among its various resources, aimed at different aspects and processes of health-related data sharing, GA4GH has introduced a machine-readable ontology known as the Data Use Ontology\footnote{\url{http://purl.obolibrary.org/obo/duo} (accessed on 2 April 2024)} (DUO).
DUO~\citep{lawson_data_2021,rehm_ga4gh_2021} was designed to express Data Use Limitations (DULs), i.e., conditions and constraints defined by data providers which should be respected by data requesters to use said data.

DUO is an OWL ontology aligned with the Open Biological and Biomedical Ontology\footnote{\url{https://obofoundry.org/} (accessed on 2 April 2024)} (OBO).
By utilising OBO's upper level ontologies, DUO ensures semantic interoperability with a range of biomedical ontologies belonging to the OBO family of ontologies.
As such, DUO's main purpose is related to annotating datasets with DUL codes to specify usage conditions, articulate data usage requests, and automatically identify or discover compatible datasets by comparing requests with datasets' usage conditions.

\paragraph{Challenges of the DUO specification}
DUO expresses DULs as concepts with human-readable definitions, utilising the \texttt{obo:IAO\_0000115} property, e.g., as \texttt{skos:definition} is used to define SKOS concepts.
This limits their utility to humans or machines that operate solely on known concepts.
Furthermore, DUO concepts lack linkage to relevant legal concepts, leading to ambiguity regarding the implications of their usage in strongly legislated jurisdictions like the EU, where the GDPR~\citeyearpar{noauthor_regulation_2016} introduces additional accountability and compliance requirements that must be acknowledged and adhered to.
While existing documentation mentions that the applicability of laws falls under the responsibility of the adopter and that DUO terms have not been evaluated for GDPR compliance, it is crucial for data subjects and data controllers to ensure compatibility with existing regulations.
As such, the absence of such support from the DUO specification poses a risk of hindering interoperability as additional approaches need to be taken to fulfil legal requirements.
Moreover, with the EU push to have a common `Health Data Space'\footnote{\url{https://ec.europa.eu/health/ehealth-digital-health-and-care/european-health-data-space_en} (accessed on 2 April 2024)}, machine-readability and automation will play a pivotal role in facilitating legally-aligned health data exchange.

\paragraph{Proposed improvements over the DUO specification}
To achieve \textit{true machine-readability}, DUO concepts must be represented with permissions, prohibitions, constraints, and duties to form machine-readable rules, by leveraging semantic standards for the expression of asset usage conditions.
By formalising the DULs embedded within the descriptions of each DUO concept as a set of rules, these become explicit and can be attached and sent alongside the data for future inspection.

To assess the compatibility of a data request with the dataset's DULs, both the conditions set by the data provider and those articulated by the data requester for data use should be formulated as policies.
These policies can then be matched to determine if the intended use aligns with the dataset's conditions, using the same approach as the one presented in Section~\ref{sec:algorithm}.
While DUO is currently being utilised in this manner, as evidenced in systems like the Data Use Oversight System\footnote{\url{https://duos.broadinstitute.org/} (accessed on 2 April 2024)} (DUOS), the matching relies on hierarchical compatibility between data request and data use conditions established through a subclass relationship, i.e., a data request $DR$ is a subclass/superclass of a data use condition $DUC$.
This approach has limitations in terms of its capacity and expressiveness for delineating fine-grained rules to utilise in automated systems, as not all relevant pieces of information can be explicitly captured in distinct concepts.
For instance, DUO's \texttt{DUO\_0000006} indicates through a sole human-readable label that use is allowed for health/medical/biomedical research purposes, not including the study of population origins or ancestry, while the same information can be much more explicitly declared using ODRL policies with permission and prohibition rules with purpose constraints.

More significantly, in order to automate the generation of data access/usage agreements effectively, a set of criteria should be considered when selecting a vocabulary to articulate such conditions:
\begin{enumerate}
    \item[(i)] the level of expressiveness to define rules and policies, encompassing the ability to express actions, purposes, or other constraints as distinct concepts that can be autonomously specified and evaluated, and combined in distinct ways to represent various types of policies;
    \item[(ii)] the capability to associate and verify their conformity and adherence to legal requirements, e.g., such as the GDPR; and
    \item[(iii)] the capacity to specify access/usage conditions in a machine-readable format and utilise them for assessing the accuracy and comprehensiveness of information that should be in such a data agreement.
\end{enumerate}
% FROM THE PAPER: Such solutions have existed for a while now -- for example, Answer Set Programming (ASP) and logic-based semantic reasoners have been utilised in a variety of domains -- including for representing information and using it for checking legal compliance for GDPR (see Section.\ref{sec:sota-legal}).

With the aforementioned motivation in mind, this Thesis proposes an approach to explicitly represent DUO concepts using RDF, leveraging ODRL and the efforts declared in previous Sections related with an OAC-based architecture for decentralised access to data.
The choice of using ODRL, beyond being a W3C Recommendation for the expression of policies, is supported by the following motives:
\begin{enumerate}
    \item[(i)] it is RDF-based, ensuring machine-readability;
    \item[(ii)] it encompasses concepts that model domain-specific and legally relevant terms to depict constraints, such as spatial and temporal operators, as well as support various types of policies like offers, requests, and agreements; % Additionally, it offers flexibility in utilizing these concepts in a manner analogous to the conventional contents and structures of legal agreements;
    \item[(iii)] its usage can be validated, and efforts are underway within the W3C ODRL CG to actively develop a formal semantics specification~\citep{fornara_odrl_2023};
    \item[(iv)] it facilitates the development of extensions through ODRL profiles, offering the flexibility to tailor ODRL to specific requirements -- as proved by this Thesis' work on OAC (Section~\ref{sec:oac}), which connects ODRL with legal requirements using DPV and can be extended to cater for legal requirements of health data sharing; and
    \item[(v)] backwards compatibility can be ensured -- existing DUO-based systems can adopt the practices suggested in this Section and continue being compatible with the DUO specification. Also, DUO users can select which aspects of this solution they want to incorporate in their system.
\end{enumerate}

As such, based on the algorithms described in Section~\ref{sec:algorithm}, in this Section, (i) DUO concepts are modelled as ODRL policies, (ii) such policies are instantiated as ODRL offers for the access to health-related datasets, (iii) offers are matched with incoming data requests to generate permissive or prohibitive data agreements, and (iv) work on OAC is recycled to deal with GDPR obligations for the processing of health data.

% modelling of DUODRL
% extension of the policy matching algorithm to deal with DUODRL requirements
% poc implementation
% \section{Special categories of data and research exceptions}
\label{sec:biomedical_exception}

As discussed in preceding Sections, the stringent requirements for obtaining consent under the GDPR place significant burdens on data subjects.
Requiring separate agreements for each app provider and for each specific purpose leads to repetitive consent requests.
In the biomedical field, individuals may have less involvement in decision-making regarding their data compared to other sectors -- the benefits from participation in biomedical research are often not immediate and may not directly impact the individual's personal circumstances.
Consequently, individuals may be less inclined to make the effort of checking their Pod for new requests, reading information notices, and accepting/rejecting access requests, compared to sectors like information society services which include social media and streaming services.
As such, in the context of research, various provisions indicate a more flexible approach to consent requirements or even suggest moving away entirely from reliance on consent.

Recital 33 \citeyearpar{noauthor_regulation_2016} states that broad consent is permissible for research purposes under specific conditions -- \textit{``data subjects should be allowed to give their consent to certain areas of scientific research when in keeping with recognised ethical standards for scientific research''} and while having the \textit{``opportunity to give their consent only to certain areas of research or parts of research projects''}.
However, the terms `areas of research' or `parts of research projects' are domain-specific concepts that are not further defined in the GDPR.
The Global Alliance for Genomics and Health\footnote{\url{https://www.ga4gh.org/} (accessed on 9 March 2024)} develops distinct components and processes for health data sharing, including the Data Use Ontology (DUO)\footnote{\url{http://purl.obolibrary.org/obo/duo} (accessed on 9 March 2024)}, a vocabulary that can be used to describe data use conditions and limitations for research data generated in the health, clinical and biomedical domain \citep{lawson_data_2021,rehm_ga4gh_2021}.
While such vocabulary contains concepts of health-related research purposes, links to other ontologies with disease taxonomies, and incorporates concepts for modeling projects and obligations related to data usage, e.g., need for ethical approval, collaboration with the study's investigator, or the obligation to return the study's results, it does not take into consideration data protection-related requirements, e.g., legal grounds for processing.
% TODO: connect with the DUODRL chapter

However, this provision outlined in Recital 33 \citeyearpar{noauthor_regulation_2016} is not legally binding, is not mirrored in the actual text of the GDPR and it was strictly interpreted by the EDPS in its opinion on data protection and scientific research \citep{european_data_protection_supervisor_preliminary_2020}.
Furthermore, while the EDPS asserts that Recital 33 does not supersede the provisions mandating specific consent, it also suggests an assessment based on the data subject's rights, the sensitivity of the data, the nature and objective of the research, and relevant ethical safeguards. Concurrently, the EDPS also notes that if purposes cannot be precisely specified, data controllers could compensate by enhancing transparency and implementing safeguards.
Outside of the EU, the UK government advocated for an influential role for broad consent in medical research within its proposal to amend the UK's Data Protection Act \citep{uk_government_consultation_2022}.
This proposal was generally well-received, though some concerns were expressed regarding its potential for ambiguity and possible misuse.

Furthermore, the European Commission has proposed a regulation instrument for the health data domain, the European Health Data Space \citeyearpar{noauthor_proposal_2022}, which aims to depart from relying on consent for the secondary use of personal data in biomedical research.
According to this proposal, a `data holder', or a \textit{``any natural or legal person, which is an entity or a body in the health or care sector, or performing research in relation to these sectors [...] who has the right or obligation [...] to make available, including to register, provide, restrict access or exchange certain data''} \citeyearpar{noauthor_proposal_2022}, is mandated to disclose both personal and non-personal data under specific conditions and for a limited set of purposes, including scientific research (Article 34.1(e) \citeyearpar{noauthor_proposal_2022}), without requiring the consent of the data subject.
Additionally, Article 33.5 of the EHDS proposal \citeyearpar{noauthor_proposal_2022} appears to override national laws mandating consent by stipulating that \textit{``where the consent of the natural person is required by national law, health data access bodies shall rely on the obligations laid down in this Chapter to provide access to electronic health data''}.
The final version of this proposal, including the role of consent and its scope (broad or specific), is yet to be determined.
However, this proposal has drawn criticism from both the EDPB and EDPS in a joint opinion document, which calls for further clarification on how national laws requiring consent will interact with the proposed European legislation \citeyearpar{noauthor_edpbedps_2022}.

Biomedical research presents challenges within the GDPR due to its unique combination of a stringent regulatory framework, as it involves processing health data, which falls under the GDPR's special categories of data, alongside a set of exemptions designed to facilitate research due to its societal significance.

\subsection{A stricter regime for health data processing}
\label{sec:stricter_regime}

GDPR's Article 9.1 \citeyearpar{noauthor_regulation_2016} prohibits the processing of special categories of data, including health data.
However, there are ten exceptions to this rule, one of which is explicit consent from the data subject.
Nevertheless, the term `explicit' lacks clarity, as it's unclear what distinguishes it from `regular' consent, which requires a clear affirmative action or statement by the data subject.
Further clarification is needed in the GDPR regarding the additional steps a controller should take to obtain explicit consent from a data subject \citep{european_data_protection_board_guidelines_2020}.
The EDPB offers various examples of how explicit consent can be expressed.
These include providing a written statement, or in the digital context, actions such as filling out an electronic form, sending an email, uploading a scanned document bearing the data subject's signature, or using an electronic signature.
Another method mentioned is two-stage verification, where the data subject may receive an email from the controller requesting consent to process specific medical data.
Upon agreement, the data subject is asked to respond via email with the phrase `I agree', followed by receiving a verification link or an SMS message for confirmation \citep{european_data_protection_board_guidelines_2020}.

Within decentralised frameworks such as Solid, various approaches can be employed for the purpose of expressing explicit consent.
Depending on the Solid server chosen by users to host their Pod, an inbox container, akin to email inboxes found in other systems, may be automatically created when the user sets up the Pod.
This container can serve as a platform to receive such requests as it is equipped with a specialised access control authorisation, allowing only the data subject to read its contents while permitting other users to write to it. 
However, due to the lack of standardisation across the Solid ecosystem, the presence of this container cannot always be guaranteed, or it may be named differently, leading to interoperability issues.
A more sophisticated solution involves adopting a graph-centric interpretation of a Pod, wherein each Solid Pod functions as a hybrid, contextualised knowledge graph \citep{dedecker_whats_2022}.
In this context, `hybrid' denotes support for both documents and RDF statements, while `contextualised' signifies the ability to associate each document and statement with metadata such as policies or provenance data.
By accurately recording metadata, including context and provenance, multiple interfaces of the Pod can be generated as needed by various applications chosen by the data subject.
In this scenario, requests can be seamlessly integrated into the graph without requiring hardcoded specifications in the application for where the requests should be written.
These requests can then be visualised by the data subject using a Solid application or service compatible with this graph-centric approach.
Additionally, the research conducted by \cite{braun_selfverifying_2022} can be utilised to sign and validate resources carrying the `I agree' statement of the data subject.

In summary, expressing explicit consent through pre-set polices poses challenges.
Matching user policies, predefined in advance, with data requests is unlikely to meet the explicit nature of consent. 
Although matching can enhance transparency and assist individuals in decision-making, a separate action of explicitly approving the use of personal data is required to meet the explicit requirement of consent.

\subsection{A series of derogations for research purposes}
\label{sec:derogations}

In this Section, three distinct aspects, relevant to the domain of health research, are discussed: (i) the compatibility between data collection purposes and secondary reuse for research, (ii) exceptions from the right to information, and (iii) alternative exceptions, aside from consent, for processing special categories of data.

\paragraph{Secondary use for research}
As previously discussed in Section \ref{sec:consent_compatibility}, related to the `purpose limitation' principle and the assessment of compatibility, the GDPR states that \textit{``data shall be collected for specific, explicit and legitimate purposes and not further processed in a manner that is incompatible with those purposes''} (Article 5.1(b) \citeyearpar{noauthor_regulation_2016}).
As such, there is an assumption of compatibility between the purpose of collection and subsequent reuse, provided that the personal data processing for scientific research purposes appropriately implements safeguards to protect the rights and freedoms of the data subject (as outlined in Article 89.1 \citeyearpar{noauthor_regulation_2016}).
It is crucial to highlight that the prohibition against processing personal data for incompatible purposes differs from the requirement of purpose specificity, and an exception does not alleviate the need for a specific purpose.
Moreover, regardless of compatibility, the data controller must rely on consent or another legal ground to process personal (health) data.
However, there is one provision in the GDPR preamble that questions the distinction between these two requirements -- \textit{``The processing of personal data for purposes other than those for which the personal data were initially collected should be allowed only where the processing is compatible with the purposes for which the personal data were initially collected. In such a case, no legal basis separate from that which allows the collection of the personal data is required''} (Recital 50 \citeyearpar{noauthor_regulation_2016}).
This appears to challenge the separation between the `purpose limitation' and the `lawfulness' principles.
This intersection, and its implications for decentralised data-sharing ecosystems such as Solid, needs to be further investigated.

\paragraph{Exceptions to the information obligations}
In Section \ref{sec:specific_consent}, particularly in the ``Identifying the data controller'' paragraph, the information obligations outlined in Articles 13 and 14 of the GDPR \citeyearpar{noauthor_regulation_2016} are explored, with a focus on the timing of when information must be provided to the data subject.
In particular, Article 14 provides an exception for cases where personal data are processed for research purposes and have not been obtained directly from the data subject.
This exception may be relevant to Solid, considering that not all personal data stored in Solid Pods originates directly from the data subject -- it may be generated by app providers, Pod providers, other users, or agents.
Furthermore, according to Article 14.5, if (i) providing information is impossible or would require disproportionate effort, or if doing so is likely to render impossible or seriously impair the achievement of the processing objectives, and (ii) the conditions and safeguards specified in Article 89 \citeyearpar{noauthor_regulation_2016} are met, the information requirements outlined in Article 14 are inapplicable.
The compliance of Solid-based data exchanges with these conditions and safeguards in place will need to be evaluated on a case-by-case basis, depending on the context and the data access request.
However, it is probable that these conditions will be fulfilled only in exceptional cases rather than as a standard practice, and if they are met, the data controller \textit{``shall take appropriate measures to protect the data subject's rights and freedoms and legitimate interests, including making the information publicly available''}. 
As such, further research is needed to explore the role of Solid's notification system, as well as other mechanisms, to act as appropriate measures to safeguard the rights of the data subject.

\paragraph{Alternative legal bases beyond consent}
In addition to explicit consent, GDPR's Article 9.2 \citeyearpar{noauthor_regulation_2016} outlines other exceptions to the prohibition on processing special categories of data.
Article 9.2(j) is particularly pertinent to this discussion because it pertains to the processing of personal data for health research.
This point permits the processing of health-related data when it is necessary for scientific research in accordance with Article 89.1 \citeyearpar{noauthor_regulation_2016}, as long as it is based on European or national law.
Such processing must be proportionate to the intended purpose, uphold the essence of the right to data protection, and include appropriate and specific measures to safeguard the fundamental rights and interests of the data subject.
Consequently, the applicability of this exception hinges on the identification of a European or national law that can justify the processing of personal data.
If the processing falls within the scope of such legislation, explicit consent from the data subject is not required.

As such, from this Section is possible to conclude that the exemptions for processing personal data for scientific research hinge on the adoption and use of suitable safeguards.
According to GDPR's Article 89.1, these safeguards center around upholding the `data minimisation' principle and include practices like pseudonymisation and methods that prevent the identification of data subjects.
Subsequent research could explore whether PIMS, such as the Solid with an OAC-based matching system, could serve as a safeguard in this context.


% \section{Ethical challenges of controlling data and reclaiming control over it}
\label{sec:ethical_challenges}

Advancements in data-driven innovations are poised to drive further economic and societal progress~\citep{jacobides_platforms_2019}.
The analysis, sharing, and reuse of data have led to transformative changes in business models and government processes, enabling them to capitalise on these practices. 
As discussed in the previous Sections, these changes propelled policy initiatives implemented by various governments globally.
In particular, the EU is actively engaged in this transformation, exemplified by the \cite{european_commission_communication_2020} commitment\footnote{The European Commission's strategy and related documents are available at \url{https://ec.europa.eu/info/strategy/priorities-2019-2024/europe-fit-digital-age_en} (accessed on 10 March 2024)} to shaping ``A Europe fit for the Digital Age''. 
Whether it is a prominent Big Tech firm headquartered in the United States, a major data intermediary in the EU, or a state-controlled entity in China, contemporary data practices face scrutiny from diverse sectors of society, spanning individuals, non-governmental organisations (NGOs), academics, and governmental bodies.
Such distrust in digital services has been called into question~\citep{waldman_industry_2021}, prompting individuals to ponder who should they trust their data with.

Amidst this trust crisis, technology has emerged as a potential solution, in particular self-sovereign PIMS \citep{chomczyk_penedo_selfsovereign_2021}, as discussed in Section~\ref{sec:motivation_legal}.
These models empower users to directly control their data, dictating the terms of access and usage, and have been gaining the support of policymakers, in particular in Europe, with the European Commission supporting the creation of common European data spaces~\citeyearpar{noauthor_commission_2022}.
Moreover, it could be argued that the EU is strategically investing in these technologies to foster more democratic and participatory data practices, and enhance confidence in data-intensive operations by advocating for technologically robust systems that reduce reliance on the reputation of individual firms, thus mitigating power imbalances between data subjects and controllers~\citep{european_commission_communication_2020}.

The literature exploring the concept of trust is extensive, yet complex due to varying interpretations.
\cite{de_filippi_blockchain_2020} distinguished trust from confidence, noting that trust is rooted in personal vulnerability and risk-taking, while confidence is based on internalised expectations stemming from knowledge or past experiences.
As such, in this Section, the interest of data subjects in technologies that provide insights into how their information is integrated into real personal data handling processes is studied as a vehicle of trust, given their general apprehension regarding the processing actions of data controllers over personal data.
As visible in the previous Sections, the personal data regulatory framework in the EU is designed to address imbalances or vulnerabilities between multiple parties by revealing potential risks and resulting harms, aiming to leverage consent as a catalyst for the data-driven economy \citep{chomczyk_penedo_towards_2022}. 
Simultaneously, they aim to furnish essential information to individuals making decisions, facilitating informed choices \cite{benshahar_more_2014}.
Furthermore, from an ethical standpoint, several norms emerge that should guide the conduct of individuals with whom information is shared to ensure trustworthiness.
These norms encompass sincerity, competence, and the appropriateness of the entrusted task \citep{hawley_how_2019}.

Considering the myriad of factors influencing both trust and confidence, the analysis in this Section focuses on (i) transparency as a crucial prerequisite for the functioning of decentralised PIMS, (ii) the relevant EU regulatory framework on personal data, and (iii) an ethical debate concerning data control, as outlined in \citeauthor{bodo_mediated_2021}'s framework for mediated technological trust.
The emphasis on transparency stems from three primary considerations:
\begin{itemize}
    \item from a regulatory viewpoint, transparency stands as a fundamental principle within personal data protection regimes, often integrated alongside lawfulness and fairness, as exemplified in GDPR's Article 5.1(a);
    \item transparency encompasses both its \textit{ex-ante} and \textit{ex-post} components, with the latter including the issue of explainability \citep{felzmann_transparency_2019};
    \item transparency offers the potential to demystify the `black box' nature of many AI systems, enabling the identification of potential biases towards vulnerable populations \citep{pasquale_black_2015}.
\end{itemize}

As illustrated by case law from supervisory authorities, the intricate nature of data processing activities has proven challenging for data controllers to articulate in straightforward terms, especially when relying on limited attention resources from data subjects \citep{european_data_protection_board_guidelines_2020}.
The dearth of actionable information, to understand data handling practices, poses a risk to fostering trust among involved parties.
As a result, individuals are endeavoring to reassert control over their data and restrict its usage by such entities, also by looking at new data governance schemes such as PIMS or other data intermediaries \citep{craglia_digitranscope_2021,papagiannakopoulou_leveraging_2014}. 

As such, the concept of `control' gains particular importance as users require someone to trust in order to reclaim control over their data in the digital era.
Emerging data governance models are coupled with legal frameworks to assist data subjects in asserting their agency.
For instance, in the data cooperative model (which is regulated by the DGA), cooperatives act as trustees overseeing data on behalf of data subjects, thus enabling data subjects to maintain democratic control over their data. 
In such governance frameworks, establishing a relationship of trust between cooperatives managing data and data subjects is paramount.
In certain instances, trustees may need to consult with data subjects, providing agreements and contracts to inform them. 
Meanwhile, data subjects can articulate their preferences and determine how to share their data and for what purposes~\citep{craglia_digitranscope_2021}.

Data cooperatives and other intermediaries (will) play a pivotal role in empowering data subjects to maintain control over their data and reassert their ethical standing in the digital era.
Specifically, personal data sovereignty offers a significant return to more democratic and egalitarian governance, allowing individuals to reclaim control over their personal data \citep{craglia_digitranscope_2021, giannopoulou_digital_2023}.
In theory, these systems should restore personal autonomy and uphold classical liberal values by fostering trust-based relationships. %TODO: add citation
Furthermore, drawing from our current democratic experiences can offer valuable lessons to avoid repeating the same mistakes made in the past two centuries.
During this time, a substantial portion of the population, particularly in the Global South, suffered from neglected rights due to inadequate governance safeguards. %TODO: add citation
For instance, democratic failures in Latin America over the last 50 years, stemming from regime changes, economic crises, or environmental catastrophes, have led to the absence of robust governance mechanisms to address such challenges.
One illustrative example is the impact of the last Argentinian military dictatorship, which significantly altered the identities of numerous individuals who were abducted as children and placed with new families, effectively erasing their true identities.
In response, collective organisations emerged to address this injustice, recognising the vulnerable position these individuals were placed in and their limited ability to resist and reclaim their true identities~\citep{gesteira_mas_2014}.

Despite the critical role of trust in upholding the autonomy and agency of data subjects \citep{benshahar_more_2014}, the methods currently employed to foster trust remain contentious, and unresolved societal issues persist in digital services and emerging digital intermediaries \citep{carovano_regulating_2023}.
Given the practical nature of the issues at hand, including how to practically approach trust, establish trust relationships between data subjects and data intermediaries, and identify the necessary conditions for fostering trust, a public Think-In event was organised in the context of the PROTECT project.
In these events, individuals were convened to explore the implications of governing personal data spaces through decentralised PIMS or trusted data intermediaries.
With the ``citizens' Think-In'' approach, there is a public discussion focused on the opinion of individuals, which encourages direct participation from attendees.
In particular, through small-scale group discussions, a Think-In offers a platform for individuals from diverse backgrounds to deliberate and exchange views on current societal issues stemming from advancements in Science, Technology, Engineering, and Mathematics (STEM) fields\footnote{Information regarding the organised PROTECT Think-Ins and respective results is available at \url{https://w3id.org/people/besteves/phd/thinkin} (accessed on 11 March 2024)}. %TODO: add citation of think-ins

While the comprehensive outcomes of the Think-In process will not be included in this Thesis as a contribution, it is worth noting that the general public exhibited sensitivity toward the ethical considerations regarding whom to trust and the significance of transparency in such contexts.
Citizens emphasised the importance of preventing the GDPR from turning into a mere `tick-box' compliance exercise, similar to the current format of privacy notices which result from deploying template privacy notices for distinct data processing activities.
Furthermore, there was a call for increased disclosure and oversight concerning the practical and beneficial utilisation of personal data, highlighting the importance of meaningful transparency in fostering trust among parties involved in such sensitive data exchanges.

To conclude, the insights derived from the Citizens' Think-In discussion offer a valuable foundation for considering the integration of transparency into data access agreement terms for personal data vaults, presented in both machine-readable and human-readable formats.
As such, the proposed vocabulary work, described in Chapter~\ref{chap:vocabularies}, represents a first step to offer said transparency for data subjects, containing both machine-readable and human-readable descriptions of concepts.
This approach enables data subjects to better comprehend and manage the expression of policy terms, and empowers data controllers and data subjects to navigate the intricate data-sharing landscape of the platform economy with greater control vested in the data subject.

% Design and implementation of a policy matching algorithm and data-sharing agreement generator prototype for access to data stored in Solid Pods.