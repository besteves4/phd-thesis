\section{Assumptions}
\label{sec:assumptions}

The work presented in this Thesis is done under the following set of assumptions:

\begin{enumerate}
    \item [\textbf{A1.}] We assume that decentralised personal datastores function as a personal information management system and as such should be regulated according to personal data processing law, whereas broader datastores can be used for storing all types of data and therefore imply a myriad of additional legislation, which is not in the scope of this research. 
    \item [\textbf{A2.}] We assume that is viable and advantageous for users to manage their data and privacy preferences through decentralised services and applications.
    % Re-written as restrictions
    % \item [\textbf{A3.}] We bound our research to decentralised datastores based on the technology stack of the Semantic Web.
    % \item [\textbf{A4.}] We bound our research to the personal data protection domain in the European Union as it presents a fully-fledged legal regime, the GDPR, which puts the data subjects at the centre of the flow of their personal data.
    % \item [\textbf{A5.}] We bound our research to personal data that is digitally available in an accessible format, e.g. RDF, CSV or PNG files.
\end{enumerate}
