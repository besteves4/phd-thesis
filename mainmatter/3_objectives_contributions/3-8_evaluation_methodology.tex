\section{Evaluation Methodology}
\label{sec:eval_methodology}

As previously mentioned, the work for this Thesis involves two distinct research fields, law and ontology engineering.
As such, distinct evaluation methods were followed to assess the hypotheses identified in Section~\ref{sec:hypotheses}, in particular:

\begin{enumerate}
    \item [\textbf{E1.}] (for H1.) The goal of this evaluation is to determine the alignment of the developed models to represent consent terms and fine-grained policies for the processing of personal data with the EU's General Data Protection Regulation. To this end, the proposed vocabularies were validated by legal experts through collaboration with members of W3C DPVCG and legal scholars from the PROTECT ITN. Moreover, alignment with the ISO/IEC 27560 standard on consent records and receipts was also verified.
    \item [\textbf{E2.}] (for H2a., H2b., and H2c.) The goal of this evaluation is to assess whether the developed methods can be used to establish access control conditions, describe metadata related to decentralised personal datastores, and represent information related to data subject's rights. To this end, the quality of the proposed ontologies was evaluated by detecting common pitfalls and alignment with FAIR principles, and their ability to answer the competency questions through SPARQL queries was also verified.
    \item [\textbf{E3.}] (for H3.) The goal of this evaluation is to test whether the developed ontologies and policy-based algorithms can be used to define data access agreements that fulfil the data subjects' privacy preferences. To this end, a proof of concept implementation for policy matching towards the generation of data access agreements was built to evaluate the proposed algorithms against a specific real-world use case involving health data sharing. Furthermore, proof of concept prototypes to generate policies and exercise the GDPR's right of access were also built, to assess the applicability of the developed vocabularies in the development of decentralised applications, in addition to a data altruism protocol which verifies the extensibility of the proposed vocabularies to cover other data protection laws, e.g., the EU's Data Governance Act.
\end{enumerate}
