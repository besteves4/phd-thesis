\section{Contributions}
\label{sec:contributions}

The contributions of this Thesis are outlined below.
In order to centralise their access, a Web page with links to all contributions is available at \url{https://w3id.org/people/besteves/phd/contributions}, including links to open-access versions of the publications mentioned in Section~\ref{sec:publications}.

\subsection{Main contributions}
\label{sec:contr_main}

\begin{enumerate}
    \item [\textbf{C1.}] \textbf{Development of Vocabularies}
    \begin{enumerate}
        \item [\textbf{C1.1.}] \textbf{OAC}: Development of an ODRL Profile for Access Control (OAC), to define access control policies that express permissions and/or prohibitions associated with data stored in a decentralised storage environment, such as Solid Pods.
        \item [\textbf{C1.2.}] \textbf{PLASMA}: Development of a Policy LAnguage for Solid’s Metadata-based Access control (PLASMA), to provide consistent taxonomies to describe the entities, infrastructure, legal roles, policies, notices, registries, and logs necessary to understand and establish responsibilities and accountability within the Solid ecosystem.
        \item [\textbf{C1.3.}] \textbf{Rights Exercising}: Development of vocabulary-based patterns to describe rights exercising metadata using DPV, to provide uniform recording of data subject rights exercising activities.
        \item [\textbf{C1.4.}] \textbf{DUODRL}: Development of ODRL rules for the Data Use Ontology (DUO), to create policies for the sharing of health data.
        \item [\textbf{C1.5.}] \textbf{DGAterms}: Development of a Data Governance Act (DGA) vocabulary, to create OAC-based policies for the sharing of data for altruistic purposes and keep registries of available datasets.
    \end{enumerate}
    \item [\textbf{C2.}] \textbf{Policy matching algorithm}: Design and implementation of a policy matching algorithm and data-sharing agreement generator prototype for access to data stored in Solid Pods.
\end{enumerate}

\subsection{Minor contributions}
\label{sec:contr_minor}

\begin{enumerate}
    \item [\textbf{C3.}] \textbf{Analysis of data protection-related challenges for decentralised datastores}: A complete literature review was performed for existing work on Solid, machine-readable policy languages and data protection vocabularies in Chapter \ref{chap:sota}. From this review, a series of gaps and challenges in the literature were identified and described in Section \ref{sec:challenges}.
    \item [\textbf{C4.}] \textbf{Proof of concept prototypes}
    \begin{enumerate}
        \item [\textbf{C4.1.}] \textbf{SOPE}: Development of the Solid ODRL access control Policies Editor (SOPE), to generate and store OAC policies in Solid Pods and visualise existing policies.
        \item [\textbf{C4.2.}] \textbf{SoDA}: Development of a Solid Data Altruism application (SoDA), to implement data altruism as a service using the Solid protocol and ODRL policies to grant access to personal data for altruistic purposes in a privacy-friendly manner.
        \item [\textbf{C4.3.}] \textbf{Service for exercising data subject rights}: Design and implementation of a service to generate rights exercising metadata.
        \item [\textbf{C4.4.}] \textbf{Service to search for data protection-related concepts}: A REST API service to find references to specific concepts in the collection of identified ontologies and languages is available at \url{https://protect.oeg.fi.upm.es/sota/searcher}.
    \end{enumerate}
\end{enumerate}

\subsection{Contributions to W3C Community Groups}
\label{sec:contr_w3c}

\begin{enumerate}
    \item [\textbf{C5.}] \textbf{Contributions to W3C DPVCG}: When aligned with the groups' purpose of having metadata to describe personal data handling activities, the concepts present in the developed vocabularies were submitted for integration in the DPV's specifications. Contributions to the DPV primer were also submitted.
    \item [\textbf{C6.}] \textbf{Contributions to W3C ODRL CG}: OAC was submitted to be considered as an official ODRL profile for Access Control. The ODRL-related work developed in this Thesis was also considered for the under-development specification of a formal semantics document for ODRL.
\end{enumerate}