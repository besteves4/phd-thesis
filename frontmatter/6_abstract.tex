% called by main.tex
%

\section*{Abstract}
\addcontentsline{toc}{section}{Abstract}
\label{sec::abstract}

%<Abstract in English: maximum of 4000 characters, plain text (without symbols), structured summary of the thesis (introduction or motivation, objectives, findings and conclusions)>

%With the widespread of technologies in every aspect of our day-to-day life, the amount of data available has reached a critical level and the legal and ethical implications of its exploration have been under debate for quite a few years.
With the widespread of technologies in every aspect of our day-to-day life, the amount of data available worldwide is growing rapidly and the legal and ethical implications of its exploration have been under debate for quite a few years.
When the General Data Protection Regulation (GDPR) came into full effect on the 25th of May 2018, companies had to deal with the impact of this new legislation on the processing of personal data and users were left in overload with the amount of complex technical information on their renewed data rights.
The main goal of this thesis is to find ways to help users of Web services deal with this overload of privacy policies, offering services that match their preferences and respect their rights.
%The main goal of this thesis is to find ways to help users deal with this overload of privacy policies, offering services that match their preferences and respect the GDPR data subject's rights.

In this context, the use and extension of data protection vocabularies and machine-readable policy languages are suitable for the representation of individual privacy preferences and requirements, fine-grained policies for the processing of personal data and other machine-readable information related to GDPR rights and obligations, including the logging of processing activities for future auditing and the exercising of user's personal data-related rights.
Furthermore, these specifications can also be used to establish a policy matching mechanism where fine-grained GDPR-aligned access control policies are used to manage and determine access to decentralised personal datastores, such as Solid Pods.
%Using these policies, a policy matching mechanism, in which data subjects and data controllers communicate until a consensus is reached on what are the acceptable privacy terms for both parties for the disclosure of personal data to occur, a policy matching mechanism was developed on top of the Solid stack of technologies. 
Solid is a decentralised data environment that detaches the storage of data from the processing of said data performed by data-driven applications.
Such an architecture allows Web users to regain data ownership over their personal data and regain trust in the services using it as the users are the ones specifying who can access their data.
The policy matching algorithm and the developed vocabularies are also extended to deal with the requirements of sharing health data and to manage the requirements of the newly enforced Data Governance Act, a data-related law that regulates the activities of data altruism organisations and intermediation service providers.
%This implementation is also complemented by the development of user interfaces for data subjects to establish their privacy preferences, without having previous knowledge of how machine-readable policy languages and vocabularies work, and to assist in the exercising of their GDPR data subject rights in decentralized data environments.

\beatriz{add sentence with conclusions of the thesis?}

\newpage
\section*{Resumen}
\addcontentsline{toc}{section}{Resumen}
\label{sec::resumen}


<Resumen en español: máximo de 4000 caracteres, texto plano (sin símbolos), resumen estructurado de la tesis (introducción o motivación, objetivos, hallazgos y conclusiones)>

